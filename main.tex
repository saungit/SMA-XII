%%%%%%%%%%%%%%%%%%%%%%%%%%%%%%%%%%%%%%%%%
% The Legrand Orange Book
% LaTeX Template
% Version 2.2 (30/3/17)
%
% This template has been downloaded from:
% http://www.LaTeXTemplates.com
%
% Original author:
% Mathias Legrand (legrand.mathias@gmail.com) with modifications by:
% Vel (vel@latextemplates.com)
%
% License:
% CC BY-NC-SA 3.0 (http://creativecommons.org/licenses/by-nc-sa/3.0/)
%
% Compiling this template:
% This template uses biber for its bibliography and makeindex for its index.
% When you first open the template, compile it from the command line with the 
% commands below to make sure your LaTeX distribution is configured correctly:
%
% 1) pdflatex main
% 2) makeindex main.idx -s StyleInd.ist
% 3) biber main
% 4) pdflatex main x 2
%
% After this, when you wish to update the bibliography/index use the appropriate
% command above and make sure to compile with pdflatex several times 
% afterwards to propagate your changes to the document.
%
% This template also uses a number of packages which may need to be
% updated to the newest versions for the template to compile. It is strongly
% recommended you update your LaTeX distribution if you have any
% compilation errors.
%
% Important note:
% Chapter heading images should have a 2:1 width:height ratio,
% e.g. 920px width and 460px height.
%
%%%%%%%%%%%%%%%%%%%%%%%%%%%%%%%%%%%%%%%%%

%----------------------------------------------------------------------------------------
%	PACKAGES AND OTHER DOCUMENT CONFIGURATIONS
%----------------------------------------------------------------------------------------

\documentclass[11pt,fleqn]{book} % Default font size and left-justified equations

%----------------------------------------------------------------------------------------

\input{structure} % Insert the commands.tex file which contains the majority of the structure behind the template

\begin{document}

%----------------------------------------------------------------------------------------
%	TITLE PAGE
%----------------------------------------------------------------------------------------

\begingroup
\thispagestyle{empty}
\begin{tikzpicture}[remember picture,overlay]
\node[inner sep=0pt] (background) at (current page.center) {\includegraphics[width=\paperwidth]{background}};
\draw (current page.center) node [fill=ocre!30!white,fill opacity=0.6,text opacity=1,inner sep=1cm]{\Huge\centering\bfseries\sffamily\parbox[c][][t]{\paperwidth}{\centering The Search for a Title\\[15pt] % Book title
{\Large A Profound Subtitle}\\[20pt] % Subtitle
{\huge Dr. John Smith}}}; % Author name
\end{tikzpicture}
\vfill
\endgroup

%----------------------------------------------------------------------------------------
%	COPYRIGHT PAGE
%----------------------------------------------------------------------------------------

\newpage
~\vfill
\thispagestyle{empty}

\noindent Copyright \copyright\ 2013 John Smith\\ % Copyright notice

\noindent \textsc{Published by Publisher}\\ % Publisher

\noindent \textsc{book-website.com}\\ % URL

\noindent Licensed under the Creative Commons Attribution-NonCommercial 3.0 Unported License (the ``License''). You may not use this file except in compliance with the License. You may obtain a copy of the License at \url{http://creativecommons.org/licenses/by-nc/3.0}. Unless required by applicable law or agreed to in writing, software distributed under the License is distributed on an \textsc{``as is'' basis, without warranties or conditions of any kind}, either express or implied. See the License for the specific language governing permissions and limitations under the License.\\ % License information

\noindent \textit{First printing, March 2013} % Printing/edition date

%----------------------------------------------------------------------------------------
%	TABLE OF CONTENTS
%----------------------------------------------------------------------------------------

%\usechapterimagefalse % If you don't want to include a chapter image, use this to toggle images off - it can be enabled later with \usechapterimagetrue

\chapterimage{chapter_head_1.pdf} % Table of contents heading image

\pagestyle{empty} % No headers

\tableofcontents % Print the table of contents itself

\cleardoublepage % Forces the first chapter to start on an odd page so it's on the right

\pagestyle{fancy} % Print headers again

%----------------------------------------------------------------------------------------
%	PART
%----------------------------------------------------------------------------------------

\part{Part One}

%----------------------------------------------------------------------------------------
%	CHAPTER 1
%----------------------------------------------------------------------------------------

\chapterimage{chapter_head_2.pdf} % Chapter heading image

\chapter{Kekongruenan Antar Bangun Datar}

\section{Pendahuluan}\index{Paragraphs of Text}

Definisi kekongruenan tidak lepas dari kesebangunan karena kekongruenan
merupakan kasus khusus kesebangunan. Jadi definisinya sebagai berikut.
Dua segibanyak (polygon) dikatakan kongruen jika ada korespondensi satu-satu
antara titik-titik sudut kedua segi banyak tersebut sedemikian hingga berlaku: 

1. sudut-sudut yang bersesuaian sama besar, dan

2. semua perbandingan panjang sisi-sisi yang bersesuaian adalah satu.

Syarat kedua ini dapat diringkas menjadi 2`. sisi-sisi yang bersesuaian sama panjang. 

%------------------------------------------------

\section{Contoh}\index{Contoh}
\includegraphics[width = 8cm, height= 5cm]{Pictures/1.png}
 
Pada gambar di atas telah dibuat korespondensi satu-satu antar titik-titik sudut pada kedua bangun sehingga sudut-sudut yang bersesuaian sama besar dan sisi-sisi yang bersesuaian sama panjang Berarti (sesuai definisi) dapat disimpulkan segiempat
ABCD kongruen dengan segiempat EFGH atau ditulis segiempat ABCD $latex\cong $ EFGH.

Sekali lagi, perhatikan bahwa korespondensi yang menjadikan dua bangun datar kongruen tidak terpengaruh oleh posisi kedua bangun. Jadi sekali telah ditemukan korespondensi satu-satu antar kedua bangun maka posisi apapun tetap kongruen. 

\includegraphics[width = 8cm, height= 5cm]{Pictures/2.png}

Perhatikan gambar di atas. Kedua bangun pada posisi I, II, III, mupun IV tetap
kongruen walaupun posisi kedua bangun tersebut berubah-ubah. Jika dicermati lebih
lanjut, keempat posisi itu mewakili proses translasi, refleksi, rotasi, dan kombinasi
dari ketiganya. Secara bahasa sederhana, dua bangun dikatakan kongruen jika kedua
bangun tersebut sama dalam hal bentuk dan ukurannya. 

\paragraph{}


Selanjutnya perhatikan segiempat dan segilima berikut. 

\includegraphics[width = 8cm, height= 5cm]{Pictures/3.png}

Berdasar gambar di atas, segiempat dapat disusun dari dua segitiga dan segilima
dapat disusun dari tiga segitiga. Secara umum segi-n dapat disusun dari n – 2 segitiga.
Hal tersebut merupakan gambaran bahwa setiap segibanyak dapat disusun dari segitiga-segitiga. Oleh karena itu sifat-sifat kesebangunan dan kekongruenan pada
segitiga perlu untuk dibicarakan secara khusus. 

%------------------------------------------------



%----------------------------------------------------------------------------------------
%	CHAPTER 2
%----------------------------------------------------------------------------------------

\chapter{Teorema}

\section{Teorema}\index{Teorema}

Secara sederhana sesuai dengan pengertian kekongruenan, dua segitiga dikatakan
kongruen jika sudut-sudut yang bersesuaian sama besar dan sisi-sisi yang bersesuaian
sama panjang. Ada satu postulat dan tiga teorema yang terkait dengan kekongruenan
segitiga. Kita ingat bahwa postulat tidak dibuktikan sedangkan teorema perlu
dibuktikan. Tetapi pada modul ini kita tidak membahas bukti teorema karena telah
dibahas pada modul BERMUTU tahun sebelumnya. 

\subsection{Postulat kekongruenan s.sd.s (sisi-sudut-sisi}\index{Teorema!Postulat kekongruenan s.sd.s (sisi-sudut-sisi)}


\begin{theorem}[Postulat kekongruenan s.sd.s (sisi-sudut-sisi)]

Diberikan dua segitiga $\vartriangle $ABC dan $vartriangle $DEF dimana m$\angle$A = m$\angle$D, AB = DF maka $\vartriangle $ABC $\cong$ $\vartriangle $DEF
\end{theorem}
\includegraphics[width = 8cm, height= 4cm]{Pictures/4.png}
\subsection{Teorema kekongruenan sd.s.sd (sudut-sisi-sudut)}\index{Theorems!Teorema kekongruenan sd.s.sd (sudut-sisi-sudut)}
\begin{theorem}
Diberikan dua segitiga $\vartriangle $ABC dan $vartriangle $DEF dimana m$\angle$A = m$\angle$D, AC = DF, m$\angle$A = m$\angle$D maka $\vartriangle $ABC $\cong$ $\vartriangle $DEF
\end{theorem}
\includegraphics[width = 8cm, height= 4cm]{Pictures/5.png}

%------------------------------------------------

\subsection{Teorema Teorema kekongruenan s.s.s (sisi-sisi-sisi)}\index{Theorems!Teorema kekongruenan s.s.s (sisi-sisi-sisi)}
\begin{theorem}
Diberikan dua segitiga $\vartriangle $ABC dan $vartriangle $DEF dimana, AB = DE,  m$\angle$A = m$\angle$D,dan  m$\angle$C = m$\angle$F , BC = EF  maka $\vartriangle $ABC $\cong$ $\vartriangle $DEF
\end{theorem}
\includegraphics[width = 8cm, height= 4cm]{Pictures/6.png}

\subsection{Teorema kekongruenan s.sd.sd (sisi-sudut-sudut)}\index{Theorems!Teorema kekongruenan s.sd.sd (sisi-sudut-sudut)}
\begin{theorem}
Diberikan dua segitiga $\vartriangle $ABC dan $vartriangle $DEF dimana, AB = DE, AC = DF,dan , BC = EF  maka $\vartriangle $ABC $\cong$ $\vartriangle $DEF
\end{theorem}
\includegraphics[width = 8cm, height= 4cm]{Pictures/7.png}

\section{Definitions}\index{Definitions}

This is an example of a definition. A definition could be mathematical or it could define a concept.

\begin{definition}[Definition name]
Given a vector space $E$, a norm on $E$ is an application, denoted $||\cdot||$, $E$ in $\mathbb{R}^+=[0,+\infty[$ such that:
\begin{align}
& ||\mathbf{x}||=0\ \Rightarrow\ \mathbf{x}=\mathbf{0}\\
& ||\lambda \mathbf{x}||=|\lambda|\cdot ||\mathbf{x}||\\
& ||\mathbf{x}+\mathbf{y}||\leq ||\mathbf{x}||+||\mathbf{y}||
\end{align}
\end{definition}

%------------------------------------------------

\section{Notations}\index{Notations}

\begin{notation}
Given an open subset $G$ of $\mathbb{R}^n$, the set of functions $\varphi$ are:
\begin{enumerate}
\item Bounded support $G$;
\item Infinitely differentiable;
\end{enumerate}
a vector space is denoted by $\mathcal{D}(G)$. 
\end{notation}

%------------------------------------------------

\section{Remarks}\index{Remarks}

This is an example of a remark.

\begin{remark}
The concepts presented here are now in conventional employment in mathematics. Vector spaces are taken over the field $\mathbb{K}=\mathbb{R}$, however, established properties are easily extended to $\mathbb{K}=\mathbb{C}$.
\end{remark}

%------------------------------------------------

\section{Corollaries}\index{Corollaries}

This is an example of a corollary.

\begin{corollary}[Corollary name]
The concepts presented here are now in conventional employment in mathematics. Vector spaces are taken over the field $\mathbb{K}=\mathbb{R}$, however, established properties are easily extended to $\mathbb{K}=\mathbb{C}$.
\end{corollary}

%------------------------------------------------

\section{Propositions}\index{Propositions}

This is an example of propositions.

\subsection{Several equations}\index{Propositions!Several Equations}

\begin{proposition}[Proposition name]
It has the properties:
\begin{align}
& \big| ||\mathbf{x}|| - ||\mathbf{y}|| \big|\leq || \mathbf{x}- \mathbf{y}||\\
&  ||\sum_{i=1}^n\mathbf{x}_i||\leq \sum_{i=1}^n||\mathbf{x}_i||\quad\text{where $n$ is a finite integer}
\end{align}
\end{proposition}

\subsection{Single Line}\index{Propositions!Single Line}

\begin{proposition} 
Let $f,g\in L^2(G)$; if $\forall \varphi\in\mathcal{D}(G)$, $(f,\varphi)_0=(g,\varphi)_0$ then $f = g$. 
\end{proposition}

%------------------------------------------------

\section{Examples}\index{Examples}

This is an example of examples.

\subsection{Equation and Text}\index{Examples!Equation and Text}

\begin{example}
Let $G=\{x\in\mathbb{R}^2:|x|<3\}$ and denoted by: $x^0=(1,1)$; consider the function:
\begin{equation}
f(x)=\left\{\begin{aligned} & \mathrm{e}^{|x|} & & \text{si $|x-x^0|\leq 1/2$}\\
& 0 & & \text{si $|x-x^0|> 1/2$}\end{aligned}\right.
\end{equation}
The function $f$ has bounded support, we can take $A=\{x\in\mathbb{R}^2:|x-x^0|\leq 1/2+\epsilon\}$ for all $\epsilon\in\intoo{0}{5/2-\sqrt{2}}$.
\end{example}

\subsection{Paragraph of Text}\index{Examples!Paragraph of Text}

\begin{example}[Example name]
\lipsum[2]
\end{example}

%------------------------------------------------

\section{Exercises}\index{Exercises}

This is an example of an exercise.

\begin{exercise}
This is a good place to ask a question to test learning progress or further cement ideas into students' minds.
\end{exercise}

%------------------------------------------------

\section{Problems}\index{Problems}

\begin{problem}
What is the average airspeed velocity of an unladen swallow?
\end{problem}

%------------------------------------------------

\section{Vocabulary}\index{Vocabulary}

Define a word to improve a students' vocabulary.

\begin{vocabulary}[Word]
Definition of word.
\end{vocabulary}

%----------------------------------------------------------------------------------------
%	PART
%----------------------------------------------------------------------------------------

\part{Part Two}

%----------------------------------------------------------------------------------------
%	CHAPTER 3
%----------------------------------------------------------------------------------------

\chapterimage{chapter_head_1.pdf} % Chapter heading image

\chapter{Presenting Information}

\section{Table}\index{Table}

\begin{table}[h]
\centering
\begin{tabular}{l l l}
\toprule
\textbf{Treatments} & \textbf{Response 1} & \textbf{Response 2}\\
\midrule
Treatment 1 & 0.0003262 & 0.562 \\
Treatment 2 & 0.0015681 & 0.910 \\
Treatment 3 & 0.0009271 & 0.296 \\
\bottomrule
\end{tabular}
\caption{Table caption}
\end{table}

%------------------------------------------------

\section{Figure}\index{Figure}

\begin{figure}[h]
\centering\includegraphics[scale=0.5]{placeholder}
\caption{Figure caption}
\end{figure}

%----------------------------------------------------------------------------------------
%	BIBLIOGRAPHY
%----------------------------------------------------------------------------------------

\chapter*{Bibliography}
\addcontentsline{toc}{chapter}{\textcolor{ocre}{Bibliography}}
\section*{Books}
\addcontentsline{toc}{section}{Books}
\printbibliography[heading=bibempty,type=book]
\section*{Articles}
\addcontentsline{toc}{section}{Articles}
\printbibliography[heading=bibempty,type=article]

%----------------------------------------------------------------------------------------
%	INDEX
%----------------------------------------------------------------------------------------

\cleardoublepage
\phantomsection
\setlength{\columnsep}{0.75cm}
\addcontentsline{toc}{chapter}{\textcolor{ocre}{Index}}
\printindex

%----------------------------------------------------------------------------------------

\end{document}
