%%%%%%%%%%%%%%%%%%%%%%%%%%%%%%%%%%%%%%%%%
% The Legrand Orange Book
% LaTeX Template
% Version 2.2 (30/3/17)
%
% This template has been downloaded from:
% http://www.LaTeXTemplates.com
%
% Original author:
% Mathias Legrand (legrand.mathias@gmail.com) with modifications by:
% Vel (vel@latextemplates.com)
%
% License:
% CC BY-NC-SA 3.0 (http://creativecommons.org/licenses/by-nc-sa/3.0/)
%
% Compiling this template:
% This template uses biber for its bibliography and makeindex for its index.
% When you first open the template, compile it from the command line with the 
% commands below to make sure your LaTeX distribution is configured correctly:
%
% 1) pdflatex main
% 2) makeindex main.idx -s StyleInd.ist
% 3) biber main
% 4) pdflatex main x 2
%
% After this, when you wish to update the bibliography/index use the appropriate
% command above and make sure to compile with pdflatex several times 
% afterwards to propagate your changes to the document.
%
% This template also uses a number of packages which may need to be
% updated to the newest versions for the template to compile. It is strongly
% recommended you update your LaTeX distribution if you have any
% compilation errors.
%
% Important note:
% Chapter heading images should have a 2:1 width:height ratio,
% e.g. 920px width and 460px height.
%
%%%%%%%%%%%%%%%%%%%%%%%%%%%%%%%%%%%%%%%%%

%----------------------------------------------------------------------------------------
%	PACKAGES AND OTHER DOCUMENT CONFIGURATIONS
%----------------------------------------------------------------------------------------

\documentclass[11pt,fleqn]{book} % Default font size and left-justified equations

%----------------------------------------------------------------------------------------

\input{structure} % Insert the commands.tex file which contains the majority of the structure behind the template

\begin{document}

%----------------------------------------------------------------------------------------
%	TITLE PAGE
%----------------------------------------------------------------------------------------

\begingroup
\thispagestyle{empty}
\begin{tikzpicture}[remember picture,overlay]
\node[inner sep=0pt] (background) at (current page.center) {\includegraphics[width=\paperwidth]{background}};
\draw (current page.center) node [fill=ocre!30!white,fill opacity=0.6,text opacity=1,inner sep=1cm]{\Huge\centering\bfseries\sffamily\parbox[c][][t]{\paperwidth}{\centering The Search for a Title\\[15pt] % Book title
{\Large A Profound Subtitle}\\[20pt] % Subtitle
{\huge Dr. John Smith}}}; % Author name
\end{tikzpicture}
\vfill
\endgroup

%----------------------------------------------------------------------------------------
%	COPYRIGHT PAGE
%----------------------------------------------------------------------------------------

\newpage
~\vfill
\thispagestyle{empty}

\noindent Copyright \copyright\ 2013 John Smith\\ % Copyright notice

\noindent \textsc{Published by Publisher}\\ % Publisher

\noindent \textsc{book-website.com}\\ % URL

\noindent Licensed under the Creative Commons Attribution-NonCommercial 3.0 Unported License (the ``License''). You may not use this file except in compliance with the License. You may obtain a copy of the License at \url{http://creativecommons.org/licenses/by-nc/3.0}. Unless required by applicable law or agreed to in writing, software distributed under the License is distributed on an \textsc{``as is'' basis, without warranties or conditions of any kind}, either express or implied. See the License for the specific language governing permissions and limitations under the License.\\ % License information

\noindent \textit{First printing, March 2013} % Printing/edition date

%----------------------------------------------------------------------------------------
%	TABLE OF CONTENTS
%----------------------------------------------------------------------------------------

%\usechapterimagefalse % If you don't want to include a chapter image, use this to toggle images off - it can be enabled later with \usechapterimagetrue

\chapterimage{chapter_head_1.pdf} % Table of contents heading image

\pagestyle{empty} % No headers

\tableofcontents % Print the table of contents itself

\cleardoublepage % Forces the first chapter to start on an odd page so it's on the right

\pagestyle{fancy} % Print headers again

%----------------------------------------------------------------------------------------
%	PART
%----------------------------------------------------------------------------------------

\part{Part One}

%----------------------------------------------------------------------------------------
%	CHAPTER 1
%----------------------------------------------------------------------------------------

\chapterimage{chapter_head_2.pdf} % Chapter heading image

\chapter{Geometri Bidang Datar}
\section{Kesebangunan antar bangun datar}\index{Paragraphs of Text}

Kesebangunan dan kekongruenan biasanya digunakan untuk membandingkan dua buah bangun datar (atau lebih) dengan bentuk yang sama. dua buah bangun datar dapat dikatakan sebangun apabila panjang setiap sisi pada kedua bangun datar tersebut memiliki nilai perbandingan yang sama. sedangkan kongruen memiliki konsep yang lebih mendetail, apabila dua buah (atau lebih) bangun datar memiliki bentuk, ukuran, serta besar sudut yang sama barulah mereka dapat disebut sebagai bangun datar yang kongruen.Perhatikan gambar berikut:

\includegraphics[width=3cm,height=3cm]{Kesebangunan.jpg}


Kesebangunan Pada Persegi Panjang

Perhatikan gambar dua buah persegi panjang di bawah ini.keduanya merupakan bangun datar yang sebangun karena memiliki kesamaan sifat yang dapat dijelaskan sebagai berikut:

\includegraphics[width=3cm,height=3cm]{persegi.jpg}


\textbf{1.Perbandingan antara sisi terpanjang dengan sisi terpendek memiliki nilai yang sama.}

Perbandingan sisi terpanjang PQ dengan sisi terpendek QR  = 39 : 13  = 1 : 3
Perbandingan sisi terpanjang KL dengan sisi terpendek LM   = 24 : 8    = 1 : 3
Perbandingan sisi terpanjang RS dengan sisi terpendek QP   = 39 : 13  = 1 : 3
Perbandingan sisi terpanjang MN dengan sisi terpendek NK = 24 : 8    = 1 : 3

Dari perhitungan diatas dapat dilihat bahwa sisi terpanjang dan terpendek pada kedua persegi panjang diatas  memiliki perbandingan yang sama yaitu 1 : 3.


\textbf{2.Besar sudut pada kedua persegi panjang tersebut memiliki nilai yang sama besar.}

Sudut P = Sudut K; Sudut Q = Sudut L; Sudut R = Sudut M; Sudut S = Sudut N

Karena kedua persegi panjang tersebut hanya memiliki bentuk dan sudut yang sama besar namun tidak memiliki ukuran yang sama, maka dua bangun datar tersebut tidak bisa disebut kongruen.

\textbf{Contoh Soal Kesebangunan pada Persegi Panjang}

Ada dua buah persegi panjang dengan ukuran yang berbeda ABCD dan KLMN. Persegi panjang ABCD memiliki panjang 16cm dan lebar 4cm. Bila persegi panjang ABCD sebangun dengan persegi panjang KLMN yang memiliki panjang 32cm, maka berapakah lebar dari persegi panjang KLMN?

Karena kedua persegi panjang tersebut sebangun, maka berlaku rumus:

AB/KL = BC/LM
16/32 = 4/LM
   LM = 32x4/16
   LM = 124/16
   LM = 8 cm

Maka lebar dari persegi panjang KLMN adalah 8 cm.


Kesebangunan pada Segitiga
Kesebangunan pada segitiga agak lebih sulit dicapai karena ada tiga buah sisi yang harus sama perbandingannya. 

Contoh segitiga yang sebangun:


\includegraphics[width=3cm,height=3cm]{segitiga.jpg}


Segitiga tersebut dapat dikatakan sebangun karena perbandingan sisi-sisinya sama besar:

Sisi AC sesuai dengan sisi PR = AC/PR = 4/2 = 2/1
Sisi AB sesuai dengan sisi PQ = AB/PQ = 8/4 = 2/1
Sisi BC sesuai dengan sisi QR = BC/QR = 6/3 = 2/1

Maka AC/PR = AB/PQ = BC/QR = 2/1


Besar sudut yang bersesuaian memiliki besar yang sama:

Sudut A = sudut P; sudut B = sudut Q; sudut C = sudut R

\textbf{Contoh Soal Kesebangunan pada Persegi Panjang}


\includegraphics[width=3cm,height=3cm]{soal.jpg}


Diketahui segitiga ABC sebangun dengan segitiga KLM, maka berapakah panjang LM dan MK?

Jawab:

AB/KL = BC/LM
18/6  = 15/LM
   3  = 15/LM
   LM = 15/3
   LM = 5 cm

Dari hasil tersebut kita dapat mengetahui bahwa perbandingan sisi pada kedua segitiga tersebut adalah:

18 : 6 = 3 : 1
15 : 5 = 3 : 1
12 : MK = 3 : 1
MK = 12/3
MK = 4 cm
\\

\textbf{Contoh Kesebangunan pada Trapesium}

Perhatikan gambar di bawah ini!

\includegraphics[width=3cm,height=3cm]{soal1.jpg}

Buktikan bahwa,

Soal 6 Rumus

Jika DC = 20 cm, AB = 34 cm, DE = 9 cm dan AE = 15 cm, tentukan EF!

Pembahasan Untuk membuktikan rumus yang ditentukan, kita harus menggambar garis DH yang sejajar dengan garis BC, seperti berikut.
Karena garis EG sejajar dengan garis AH, maka segitiga DEG sebangun dengan segitiga DAH. Akibatnya,

\includegraphics[width=3cm,height=3cm]{rumus1.jpg}

Untuk DC = 20 cm, AB = 34 cm, DE = 9 cm dan AE = 15 cm, maka

\includegraphics[width=3cm,height=3cm]{rumus2.jpg}

Jadi, diperoleh panjang EF adalah 25,25 cm.

\section{Kekongruenan Antar Bangun Datar}\index{Paragraphs of Text}

Definisi kekongruenan tidak lepas dari kesebangunan karena kekongruenan
merupakan kasus khusus kesebangunan. Jadi definisinya sebagai berikut.
Dua segibanyak (polygon) dikatakan kongruen jika ada korespondensi satu-satu
antara titik-titik sudut kedua segi banyak tersebut sedemikian hingga berlaku: 

1. sudut-sudut yang bersesuaian sama besar, dan

2. semua perbandingan panjang sisi-sisi yang bersesuaian adalah satu.

Syarat kedua ini dapat diringkas menjadi 2`. sisi-sisi yang bersesuaian sama panjang. 

%------------------------------------------------

\section{Contoh}\index{Contoh}
\includegraphics[width = 8cm, height= 5cm]{Pictures/1.png}
 
Pada gambar di atas telah dibuat korespondensi satu-satu antar titik-titik sudut pada kedua bangun sehingga sudut-sudut yang bersesuaian sama besar dan sisi-sisi yang bersesuaian sama panjang Berarti (sesuai definisi) dapat disimpulkan segiempat
ABCD kongruen dengan segiempat EFGH atau ditulis segiempat ABCD $latex\cong $ EFGH.

Sekali lagi, perhatikan bahwa korespondensi yang menjadikan dua bangun datar kongruen tidak terpengaruh oleh posisi kedua bangun. Jadi sekali telah ditemukan korespondensi satu-satu antar kedua bangun maka posisi apapun tetap kongruen. 

\includegraphics[width = 8cm, height= 5cm]{Pictures/2.png}

Perhatikan gambar di atas. Kedua bangun pada posisi I, II, III, mupun IV tetap
kongruen walaupun posisi kedua bangun tersebut berubah-ubah. Jika dicermati lebih
lanjut, keempat posisi itu mewakili proses translasi, refleksi, rotasi, dan kombinasi
dari ketiganya. Secara bahasa sederhana, dua bangun dikatakan kongruen jika kedua
bangun tersebut sama dalam hal bentuk dan ukurannya. 

\paragraph{}


Selanjutnya perhatikan segiempat dan segilima berikut. 

\includegraphics[width = 8cm, height= 5cm]{Pictures/3.png}

Berdasar gambar di atas, segiempat dapat disusun dari dua segitiga dan segilima
dapat disusun dari tiga segitiga. Secara umum segi-n dapat disusun dari n – 2 segitiga.
Hal tersebut merupakan gambaran bahwa setiap segibanyak dapat disusun dari segitiga-segitiga. Oleh karena itu sifat-sifat kesebangunan dan kekongruenan pada
segitiga perlu untuk dibicarakan secara khusus. 

\section{Teorema}\index{Teorema}

Secara sederhana sesuai dengan pengertian kekongruenan, dua segitiga dikatakan
kongruen jika sudut-sudut yang bersesuaian sama besar dan sisi-sisi yang bersesuaian
sama panjang. Ada satu postulat dan tiga teorema yang terkait dengan kekongruenan
segitiga. Kita ingat bahwa postulat tidak dibuktikan sedangkan teorema perlu
dibuktikan. Tetapi pada modul ini kita tidak membahas bukti teorema karena telah
dibahas pada modul BERMUTU tahun sebelumnya. 

\subsection{Postulat kekongruenan s.sd.s (sisi-sudut-sisi}\index{Teorema!Postulat kekongruenan s.sd.s (sisi-sudut-sisi)}


\begin{theorem}[Postulat kekongruenan s.sd.s (sisi-sudut-sisi)]

Diberikan dua segitiga $\vartriangle $ABC dan $vartriangle $DEF dimana m$\angle$A = m$\angle$D, AB = DF maka $\vartriangle $ABC $\cong$ $\vartriangle $DEF
\end{theorem}
\includegraphics[width = 8cm, height= 4cm]{Pictures/4.png}
\subsection{Teorema kekongruenan sd.s.sd (sudut-sisi-sudut)}\index{Theorems!Teorema kekongruenan sd.s.sd (sudut-sisi-sudut)}
\begin{theorem}
Diberikan dua segitiga $\vartriangle $ABC dan $vartriangle $DEF dimana m$\angle$A = m$\angle$D, AC = DF, m$\angle$A = m$\angle$D maka $\vartriangle $ABC $\cong$ $\vartriangle $DEF
\end{theorem}
\includegraphics[width = 8cm, height= 4cm]{Pictures/5.png}

%------------------------------------------------

\subsection{Teorema Teorema kekongruenan s.s.s (sisi-sisi-sisi)}\index{Theorems!Teorema kekongruenan s.s.s (sisi-sisi-sisi)}
\begin{theorem}
Diberikan dua segitiga $\vartriangle $ABC dan $vartriangle $DEF dimana, AB = DE,  m$\angle$A = m$\angle$D,dan  m$\angle$C = m$\angle$F , BC = EF  maka $\vartriangle $ABC $\cong$ $\vartriangle $DEF
\end{theorem}
\includegraphics[width = 8cm, height= 4cm]{Pictures/6.png}

\subsection{Teorema kekongruenan s.sd.sd (sisi-sudut-sudut)}\index{Theorems!Teorema kekongruenan s.sd.sd (sisi-sudut-sudut)}
\begin{theorem}
Diberikan dua segitiga $\vartriangle $ABC dan $vartriangle $DEF dimana, AB = DE, AC = DF,dan , BC = EF  maka $\vartriangle $ABC $\cong$ $\vartriangle $DEF
\end{theorem}
\includegraphics[width = 8cm, height= 4cm]{Pictures/7.png}

\section{Kekongruenan Segitiga}\index{Kekongruenan Segitiga}

Pada bagian ini, pembahasan bangun-bangun yang kongruen difokuskan pada bangun segitiga. Untuk menunjukkan apakah dua segitiga kongruen atau tidak, cukup ukur setiap sisi dan sudut pada segitiga. Kemudian,bandingkan sisi-sisi dan sudut-sudut yang bersesuaian. Perhatikan tabel syarat kekongruenan dua segitiga berikut.


\includegraphics[width = 13cm, height= 12cm]{Pictures/a21.png}

\subsection{Sifat-Sifat Dua Segitiga yang Sebangun dan Kongruen}
\includegraphics[width = 13cm, height= 8cm]{Pictures/a25.png}

Setelah kita memahami pengertian kesebangunan dan kekongruenan secara umum,sekarang kita akan mendalami sifat-sifat kesebangunan dan kekongruenan, khusus mengenai segitiga. Namun sebelumnya perlu diingat bahwa dua bangun yang kongruen pasti sebangun sementara dua bangun yang sebangun belum tentu kongruen. Oleh karena itu dalam pembahasan ini akan dimulai dari sifat kekongruenan.

Secara sederhana sesuai dengan pengertian kekongruenan, dua segitiga dikatakan kongruen jika sudut-sudut yang bersesuaian sama besar dan sisi-sisi yang bersesuaian sama panjang. Ada satu postulat dan tiga teorema yang  terkait dengan kekongruenan segitiga. Kita ingat bahwa postulat tidak dibuktikan sedangkan teorema perlu dibuktikan. Tetapi pada modul ini kita tidak membahas bukti teorema karena telah dibahas pada modul BERMUTU tahun sebelumnya. 

Contoh : 

\includegraphics[width = 13cm, height= 10cm]{Pictures/a26.png}

\subsection{Contoh Soal 1}
\includegraphics[width = 13cm, height= 8cm]{Pictures/a22.png}

\includegraphics[width = 13cm, height= 8cm]{Pictures/a23.png}

\subsection{Contoh Soal 2}
\includegraphics[width = 13cm, height= 8cm]{Pictures/a24.png}
%------------------------------------------------



%----------------------------------------------------------------------------------------
%	CHAPTER 2
%----------------------------------------------------------------------------------------


\chapterimage{chapter_head_2.pdf} % Chapter heading image

\chapter{Geometri Ruang}
\section{Jarak antar Titik}

\section{Jarak Titik Ke Garis}
\subsection{Menemukan Konsep Jarak, Titik dan Garis}
\begin{enumerate}
\item Kedudukan Titik


\includegraphics{jembatan.jpg}


Jika dimisalkan jembatan penyeberangan merupakan suatu garis dan lokomotif kereta adalah suatu titik. Kita dapat melihat bahwa lokomotif tidak terletak atau melalui jembatan penyeberangan. Artinya jika dihubungkan dengan garis dan titik maka dapat disebut bahwa contoh di atas merupakan suatu titik yang tidak terletak pada garis.

\includegraphics{bola.jpg}

Gambar di atas merupakan contoh kedudukan titik terhadap bidang, dengan bola sebagai titik dan lapangan sebagai bidang. Sebuah titik dikatakan terletak pada sebuah bidang jika titik itu dapat dilalui bidang seperti terlihat pada titik A pada gambar dan sebuah titik dikatakan terletak di luar bidang jika titik itu tidak dapat dilalui bidang.

DEFINISI
\begin{enumerate}
\item 1)	Jika suatu titik dilalui garis, maka dikatakan titik terletak pada garis tersebut.
\item 2)	Jika suatu titik tidak dilalui garis, maka dikatakan titik tersebut berada di luar garis.
\item 3)	Jika suatu titik dilewati suatu bidang, maka dikatakan titik itu terletak pada bidang.
Jika titik tidak dilewati suatu bidang, maka titik itu berada di luar bidang.
\end{enumerate}

\item Jarak titik ke garis


Jarak merupakan salah satu permasalahan matematika yang sering dijumpai di sekitar kita. Jarak dapat diukur di antara dua objek, seperti rumah dengan kantor pos, rumah sakit dengan jalan raya, dan jalan raya dengan jalan raya lainnya. Pada pembahasan ini hanya akan dibahas mengenai jarak antara dua objek yang berupa titik dan garis lurus. 

Jarak titik ke garis adalah jarak terdekat sebuah titik ke garis, jarak terdekat diperoleh dengan menarik garis yang tegak lurus dengan garis yang dimaksud. Jarak titik B dengan garis g adalah panjang garis BB'.

\begin{center}
\includegraphics{panjanggaris.jpg}
\end{center}

Perhatikan contoh permasalahan berikut:

Vihara Dharma Agung terletak pada koordinat (71, 76) dan Jalan Sungai Kelara berupa garis lurus dengan persamaan $$5x-8y-280=0$$ (satuan dalam meter). Bagaimana cara mengukur jarak antara vihara dengan jalan tersebut? Salah satunya adalah dengan menggunakan rumus jarak antara titik dengan garis lurus.


\end{enumerate}

Misalkan akan ditentukan jarak antara titik A(a, b) dengan garis lurus yang memiliki persamaan $$px+qy+r = 0.$$ Perhatikan gambar berikut.

\begin{center}
\includegraphics{jarak.jpg}
\end{center}

Perlu diingat bahwa jarak dua objek adalah panjang lintasan terpendek yang menghubungkan kedua objek tersebut. Karena ruas garis yang tegak lurus dengan garis $$px+qy+r = 0$$ dan memiliki ujung di titik A dan ujung satunya di garis tersebut merupakan lintasan terpendek yang menghubungkan titik dan garis tersebut, maka panjang dari ruas garis tersebut, yaitu d, adalah jarak titik A terhadap garis $$px+qy+r = 0$$.

Jarak titik ke garis adalah jarak terdekat sebuah titik ke garis, jarak terdekat diperoleh dengan menarik garis yang tegak lurus dengan garis yang dimaksud.
Jarak titik B dengan garis g adalah panjang garis BB’  

\includegraphics[width = 8cm, height= 5cm]{Pictures/gi1.png}

Contoh :
1. Kubus ABCDEFGH memiliki panjang rusuk 8 cm, titik P merupakan perpotongan
diagonal bidang atas, hitunglah jarak titik P dengan garis AD

Penyelesaian


\includegraphics[width = 8cm, height= 5cm]{Pictures/gi2.png}
\includegraphics[width = 8cm, height= 5cm]{Pictures/gi3.png}

2. Sebuah kubus ABCD.EFGH dengan panjang rusuk 6 cm. tentukan jarak titik A ke
garis CE adalah…

\includegraphics[width = 8cm, height= 5cm]{Pictures/gi4.png}

\includegraphics[width = 8cm, height= 5cm]{Pictures/gi5.png}

Contoh Soal

Diketahui kubus ABCD.EFGH. Tentukan projeksi titik
A pada garis

a. CD!

b. BD!

Jika dari titik A ditarik garis yang tegak lurus terhadap
segmen garis CD maka diperoleh titik D sebagai hasil
proyeksinya (AD CD). 

\includegraphics[width = 8cm, height= 5cm]{Pictures/gi6.png}

b. Proyeksi titik A pada garis BD
Jika dari titik A ditarik garis yang tegak lurus
terhadap segmen garis BD maka diperoleh titik T
sebagai hasil proyeksinya (AT  BD).

\includegraphics[width = 8cm, height= 5cm]{Pictures/gi7.png}

Contoh Soal
Kubus ABCD.EFGH memiliki rusuk 8 cm. Jarak titik D ke garis HB adalah …

\includegraphics[width = 8cm, height= 5cm]{Pictures/gi8.png}

Pandanglah segitiga BDH yang terdapat dalam kubus. Segitiga BDH adalah segitiga siku-siku di D.

DH adalah salah satu rusuk kubus.

DH = 8 cm

BD adalah diagonal bidang atau diagonal sisi.


\includegraphics[width = 4cm, height= 4cm]{Pictures/gi9.png}


Contoh Soal 2
Perhatikan gambar kubus PQRS.TUVW di bawah ini.

\includegraphics[width = 4cm, height= 4cm]{Pictures/gi10.png}

Jika panjang rusuk kubus di atas adalah 8 cm dan titik X merupakan pertengahan antara rusuk PQ. Maka hitung jarak:
a) titik X ke garis ST
b) titik X ke garis RT

Penyelesaian:
Perhatikan gambar di bawah ini

\includegraphics[width = 4cm, height= 4cm]{Pictures/gi11.png}
a) titik X ke garis ST merupakan panjang garis dari titik X ke titik M (garis MX) yang tegak lurus dengan garis ST, seperti gambar berikut.

\includegraphics[width = 4cm, height= 4cm]{Pictures/gi12.png}

\includegraphics[width = 4cm, height= 4cm]{Pictures/gi14.png}

b) titik X ke garis RT merupakan panjang garis dari titik X ke titik N (garis NX) yang tegak lurus dengan garis RT, seperti gambar berikut.

\includegraphics[width = 4cm, height= 4cm]{Pictures/gi15.png}


Contoh soal 3
Diketahui panjang rusuk sebuah kubus ABCD.EFGH adalah 6cm. Maka hitunglah jarak:

a).titik D ke garis BF
b).titik B ke garis EG

Penyelesaiannya:

a).Agar lebih mudah dalam menjawabnya, mari kita perhatikan gambar di bawah ini:

\includegraphics[width = 4cm, height= 4cm]{Pictures/gi16.png}

Dari gambar di atas kita bisa melihat bahwa jarak titik D ke garis BF adalah panjang diagonal BD yang dapat ditentukan dengan menggunakan teorema phytagoras ataupun dengan rumus. Mari kita selesaikan dengan teorema phytagoras terlebih dahulu:

\includegraphics[width = 4cm, height= 4cm]{Pictures/gi17.png}

berikut bila kita mencarinya dengan menggunakan rumus:

\includegraphics[width = 4cm, height= 4cm]{Pictures/gi18.png}

b). Sama halnya dengan soal a) kita juga harus membuat gambarnya terlebih dahulu agar lebih mudah mengerjakannya

\includegraphics[width = 4cm, height= 4cm]{Pictures/gi19.png}

\includegraphics[width = 4cm, height= 4cm]{Pictures/gi20.png}

\includegraphics[width = 4cm, height= 4cm]{Pictures/20.png}


\subsection{Menemukan Rumus Jarak Titik dengan Garis}

Karena px+qy+r=0 y=-(p/q)x-(r/q) maka gradien garis yang tegak lurus dengan garis px+qy+r=0 adalah q/p, karena -(p/q)xq/p=-1. Selain tegak lurus dengan garis px+qy+r=0, garis tersebut juga melalu titik A (a,b), sehingaa:

\begin{center}
\includegraphics{nunu1.png}
\end{center}

Diperoleh, persamaan garis yang tegak lurus dengan garis px + qy + r = 0 dan melalui titik A(a,b) adalah

\begin{center}
\includegraphics{nunu2.png}
\end{center}

Setelah persamaan garisnya diperoleh, titik potong garis px + qy + r = 0 dan garis tersebut dapat ditentukan. Pertama, tentukan nilai absisnya, x2,  terlebih dahulu.

\begin{center}
\includegraphics{nunu3.png}
\end{center}

Selanjutnya, kita tentukan nilai dari ordinatnya (y2).

\begin{center}
\includegraphics{nunu4.png}
\end{center}

Setelah koordinat (x2, y2) sudah ditemukan, maka selanjutnya kita tentukan jarak antara titik tersebut dengan titik A(a, b), dengan menggunakan rumus jarak antara dua titik,

\begin{center}
\includegraphics{nunu5.png}
\end{center}

Agar lebih sederhana, kita tentukan x2 – x1 dan y2 – y1 terlebih dahulu, yaitu

\begin{center}
\includegraphics{nunu6.png}
\end{center}

dan,

\begin{center}
\includegraphics{nunu7.png}
\end{center}

Sehingga jarak antara titik (x2, y2) dan A(a, b) dapat ditentukan sebagai berikut.

\begin{center}
\includegraphics{nunu8.png}
\end{center}


This is an example of a definition. A definition could be mathematical or it could define a concept.

\begin{definition}[Definition name]
Given a vector space $E$, a norm on $E$ is an application, denoted $||\cdot||$, $E$ in $\mathbb{R}^+=[0,+\infty[$ such that:
\begin{align}
& ||\mathbf{x}||=0\ \Rightarrow\ \mathbf{x}=\mathbf{0}\\
& ||\lambda \mathbf{x}||=|\lambda|\cdot ||\mathbf{x}||\\
& ||\mathbf{x}+\mathbf{y}||\leq ||\mathbf{x}||+||\mathbf{y}||
\end{align}
\end{definition}

\chapter{Kaidah Pencacahan}

Pengertian Kaidah Pencacahan (Caunting Slots)


Kaidah pencacahan atau Caunting Slots adalah suatu kaidah yang digunakan untuk menentukan atau menghitung berapa banyak cara yang terjadi dari suatu peristiwa. Kaidah pencacahan terdiri atas :

\section{Aturan Penjumlahan}
1.Aturan penjumlahan
Aturan Penjumlahan digunakan untuk mengetahui banyaknya cara yang dapat digunakan jika terdapat banyak n1 cara pada kejadian pertama dan n2 cara pada kejadian kedua . Banyaknya
cara keseluruhan adalah n1+ n2.
Contoh Penerapan Aturan Perkalian adalah:
Berapa banyak cara keseluruhan jika akan menggunakan baju jika mempunyai 8 buah baju jenis
lengan pendek dan 17 baju lengan panjang.
Banyak cara menggunakan baju adalah 8 + 17 = 25 cara
 Jika ada A dan B yang merupakan himpunan saling lepas dengan banyak anggota himpunannya adalah x dan y, maka banyaknya cara mengambil satu anggota dari gabungan keduanya akan sama dengan x+y, dinotasikan:
 
 \includegraphics[width = 9cm, height= 2cm]{Pictures/notasijumlah.JPG}
 
 
 Gambar. Notasi aturan penjumlahan
 
 Atau secara sederhana digunakan saat ada sejumlah kejadian yang tidak saling berhubungan (saling lepas). Dalam kondisi ini kejadian-kejadian tersebut dijumlahkan untuk mendapatkan total kejadian yang mungkin terjadi.

Contoh 1:

Dari kota A ke kota B ada beberapa jenis angkutan yang dapat digunakan. Ada 4 travel, 2 kapal laut, dan 1 pesawat terbang yang dapat dipilih. Ada berapa total cara berbeda untuk berangkat dari kota A menuju kota B?

Pembahasan:


\includegraphics[width = 6cm, height= 4cm]{Pictures/contoh1.JPG}

	Dalam soal di atas ketika kita memilih travel, kapal laut, maupun pesawat terbang tidak berpengaruh satu sama lain, ketiganya merupakan himpunan yang saling lepas. Sehingga ada 4+2+1 = 7 cara berbeda untuk berangkat dari kota A menuju kota B. []

Definisi 1
Jika suatu kejadian dapat dikerjakan dengan beberapa cara, tetapi cara-cara ini tidak dapat dikerjakan pada waktu yang sama.
Jika kejadian tersebut dapat terjadi dengan n1n1 cara, atau
kejadian tersebut terjadi dengan n2n2 cara, atau
kejadian tersebut dapat terjadi dengan n3n3 cara, atau
...................
kejadian tersebut dapat terjadi dengan npnp cara,
maka kejadian dengan ciri yang demikian dapat terjadi dengan 
(n1+n2+n3+...+np)(n1+n2+n3+...+np) cara.

Contoh 1

Dalam sebuah pantia, wakil dari sebuah jurusan dapat dipilih dari dosen, atau mahasiswa. Jika pada jurusan tersebut memiliki 37 dosen dan 83 mahasiswa, Berapa banyak cara memilih wakil dari jurusan tersebut?

Jawab:

Ada 37 cara untuk memilih wakil dari sebuah jurusan yang berasal dari kalangan dosen dan ada 83 cara memilih wakil dari sebuah jurusan yang berasal dari kalangan mahasiswa. Karena pada jurusan tersebut tidak ada dosen yang berstatus mahasiswa ataupun mahasiswa yang berstatus dosen, maka berdasarkan aturan penjumlahan, ada37+83=120 cara untuk memilih wakil dari sebuah jurusan. 

Contoh 2

Seorang pelajar dapat memilih sebuah proyek komputer dari salah satu diantara tiga aftar yang tersedia. ketiga daftar tersebut terdiri atas 23, 15, dan 19 kemungkinan proyek. Proyek - proyek komputer yang ada pada ketiga daftar tersebut semuanya berbeda. Berapa banyakkemungkinan siswa tersebut memilih proyek komputer?

Jawab: 

karena dari ketiga daftar tersebut semua proyek berbeda, dimana pada daftar pertama ada 23 proyek, daftar kedua ada 15 proyek , dan daftar ketiga ada 19 proyek Maka  berdasarkan aturan penjumlahan ada 23+15+19 = 57 kemungkinan siswa tersebut memilih proyekkomputer. []

Contoh 3

Misalkan Andi akan berangkat sekolah bersama dengan teman sekelasnya, Amir. Rumah Andi terletak pada titik P dan rumah Amir terletak pada titik Q (lihat gambar). Sehingga, dalam perjalanan ke sekolah Andi akan menuju rumah Amir terlebih dahulu, kemudian bersama-sama dengan Amir ia akan berangkat ke sekolah. Ada berapa cara yang dapat ditempuh Andi untuk berangkat ke sekolah apabila ia harus melalui rumah Amir terlebih dahulu?

 \includegraphics[width = 8cm, height= 5cm]{Pictures/contoh3.JPG}

Banyaknya cara perjalanan dari titik P ke titik Q dilanjutkan ke titik R dapat digambarkan dengan diagram pohon seperti pada gambar berikut.

 \includegraphics[width = 8cm, height= 5cm]{Pictures/contoh3a.JPG}
 
 Dari diagram pohon tersebut terlihat rute perjalanan dari titik P ke titik R melalui titi Q ada 6 cara yang dapat ditulis dalam bentuk himpunan pasangan berurutan {(a, x), (a, y), (a, z), (b, x), (b, y), (b, z)}.

 
 \includegraphics[width = 8cm, height= 5cm]{Pictures/contoh3b.JPG}
 
 

%------------------------------------------------


\section{Aturan Perkalian}

1.Aturan perkalian

1.Aturan perkalian

	Kaidah Pencacahan adalah istilah dalam bahasan PELUANG. Kaidah pencacahan merupakan cara atau aturan untuk menghitung semua kemungkinan yang dapat terjadi dalam suatu percobaan tertentu. Metode yang dapat digunakan antara lain metode pengisian tempat (filling slot), Permutasi, dan Kombinasi. 

	Dalam kehidupan sehari-hari sering dihadapkan pada pemecahan masalah yang berkaitan dengan menentukan banyak cara yang mungkin terjadi dari sebuah percobaan, misalnya jika sebuah uang logam dilemparkan, akan tampak permukaan gambar atau angka. 



1. Aturan Perkalian
n1 = banyak cara unsur pertama
n2 = banyak cara unsur kedua

...
...
nk = banyak cara unsur ke-k
Maka banyak cara untuk menyusun k unsur yang tersedia adalah :
n1 x n2 x .... x nk.



\includegraphics[width = 8cm, height= 5cm]{Pictures/materikaidah1.png}

 Misalkan, dari 3 orang siswa, yaitu Algi, Bianda, dan Cahyadi akan dipilih untuk menjadi ketua kelas, sekretaris, dan bendahara dengan aturan bahwa seseorang tidak boleh merangkap jabatan pengurus kelas. Banyak cara 3 orang dipilih menjadi pengurus kelas tersebut akan dipelajari melalui uraian berikut. Amati Gambar
 
 \includegraphics[width = 8cm, height= 5cm]{Pictures/gen1.png}
Gambar  Aturan perkalian pemilihan pengurus kelas.
a. Untuk ketua kelas (K)
 Posisi ketua kelas dapat dipilih dari 3 orang, yaitu Algi  (A), Bianda (B), atau Cahyadi (C).

 Jadi, posisi ketua kelas dapat dipilih dengan 3 cara.
b. Untuk sekertaris (S)
 Jika posisi ketua kelas sudah terisi oleh seseorang maka  posisi sekretaris hanya dapat dipilih dari 2 orang yang belum terpilih menjadi pengurus kelas. 

 Jadi, posisi sekretaris dapat dipilih dengan 2 cara.
c. Untuk bendahara (A)
 
Jika posisi ketua kelas dan sekretaris sudah terisi maka posisi bendahara hanya ada satu pilihan, yaitu dijabat oleh orang yang belum terpilih menjadi pengurus kelas.

Jadi, posisi bendahara dapat dipilih dengan 1 cara.

Dengan demikian, banyak cara yang dilakukan untuk memilih 3 orang pengurus kelas dari 3 orang kandidat adalah :

3 x 2 x 1 = 6 cara.

Uraian tersebut akan lebih jelas apabila mengamati skema berikut.


\includegraphics[width = 6cm, height= 3cm]{Pictures/gen2.png}

Dari uraian tersebut, dapatkah Anda menyatakan aturan perkalian? Cobalah nyatakan aturan perkalian itu dengan kata-kata Anda sendiri.

Aturan Perkalian :

Misalkan,

• operasi 1 dapat dilaksanakan dalam n1 cara;
• operasi 2 dapat dilaksanakan dalam n2 cara;
• operasi k dapat dilaksanakan dalam nk cara.

Banyak cara k operasi dapat dilaksanakan secara berurutan adalah n = n1 x n2 x n3 ... x nk.

Contoh Soal 1 :

Berapa cara yang dapat diperoleh untuk memilih posisi seorang tekong, apit kiri, dan apit kanan dari 15 atlet sepak takraw pelatnas SEA GAMES jika tidak ada posisi yang rangkap? (Tekong adalah pemain sepak takraw yang melakukan sepak permulaan).

Jawaban :

• Untuk posisi tekong.

Posisi tekong dapat dipilih dengan 15 cara dari 15 atlet pelatnas yang tersedia.

• Untuk posisi apit kiri.

Dapat dipilih dengan 14 cara dari 14 atlet yang ada (1 atlet lagi tidak terpilih karena menjadi tekong).

• Untuk posisi apit kanan.

Cara untuk memilih apit kanan hanya dengan 13 cara dari 13 atlet yang ada (2 atlet tidak dapat dipilih karena telah menjadi tekong dan apit kiri).

Dengan demikian, banyak cara yang dilakukan untuk memilih posisi dalam regu sepak takraw adalah  15 x 14 x 13 = 2.730 cara.
Ingatlah :

Apabila terdapat n buah tempat yang akan diduduki oleh n orang, terdapat :

n x ( n - 1 ( x )n -2 ) x ... x 1 cara orang menduduki tempat tersebut.
2. Faktorial

Anda telah mempelajari, banyak cara yang dilakukan untuk memilih 3 orang pengurus kelas dari 3 orang kandidat adalah 3 x 2 x 1 = 6 cara.

Selanjutnya, 3 x 2 x 1 dapat dinyatakan dengan 3! (dibaca 3 faktorial). Jadi,

3! = 3 x 2 x 1 = 6

Dengan penalaran yang sama,

4! = 4 x 3 x 2 x 1 = 4 x 3! = 4 x 6 = 24
5! = 5 x 4 x 3 x 2 x 1 = 5 x 4! = 5 x 24 = 120
6! = 6 x 5! = 6 x 120 = 720

Uraian tersebut memperjelas definisi berikut.

Definisi :

a. n! = n x ( n – 1 ) x ( n – 2 ) ... x 3 x 2 x 1, dengan n bilangan asli, untuk n >=  2.
b. 1! = 1
c. 0! = 1

Contoh Soal 2 :

Hitunglah :

a. 7!
b. 17! / 0!16!
c. 12! / 2!8!
d. 8! / 5!

Penyelesaian :


\includegraphics[width = 7cm, height= 4cm]{Pictures/gen3.png}

Contoh Soal 3 :

Nyatakan bentuk-bentuk berikut ke dalam faktorial:

\includegraphics[width = 4cm, height= 2cm]{Pictures/gen5.png}

Penyelesaian :
\includegraphics[width = 5cm, height=3cm]{Pictures/gen4.png}

Contoh Soal 4 :

3. Tentukan nilai n dari ( n + 3 ) ! = 10 ( n + 2 ) !

Pembahasan :


\includegraphics[width = 5cm, height= 3cm]{Pictures/gen6.png}


Contoh Soal 5 :

Berapa banyak bilangan yang terdiri dari
3 angka dapat dibentuk dari angka-angka 1,2,3,4,5,6,7, dan 8 jika tiap-tiap angka boleh diulang?

Pembahasan :

Unsur pertama ada 8 pilihan, kedua ada 8, ketiga 8 (karena tiap angka boleh diulang.
                                    8          8          8                      
                                    Tinggal dikalikan
                                                                                    8 x 8 x 8 = 512

Contoh Soal 6 :

Berapa banyak susunan huruf yang dapat dibentuk oleh huruf-huruf pada kata “GARDU” tanpa ada pengulangan kata :

a.  Huruf pertama adalah huruf hidup

Jawab : maka, huruf konsonan 3 buah dan huruf vokal ada 2 buah. 

Jadi

\includegraphics[width = 5cm, height=3cm]{Pictures/soalkaidah1.png}


b.      Huruf pertama huruf mati dan huruf ketiga huruf hidup

Jawab :

\includegraphics[width = 5cm, height=3cm]{Pictures/soalkaidah2.png}

Contoh Soal 7 :

Jika diperlukan 5 orang laki-laki dan 4 orang perempuan untuk membentuk suatu barisan sedemikian rupa hingga yang perempuan menempati posisi genap, berapa banyak kemungkinan susunan barisan itu?

Jawab :

\includegraphics[width = 5cm, height=3cm]{Pictures/soalkaidah3.png}

Contoh Soal 8:

Ana mempunyai baju merah,hijau, biru, dan ungu. Ana juga memiliki rok hitam, putih, dan coklat. Berapa banyak pasangan baju dan rok yang dapat dipakai Ana?

Jawaban:

Jumlah baju = 4. 
Jumlah rok = 3.
 Jadi 3 x 4 = 12.
 Maksudnya Ana bisa memakai baju dan rok dengan warna : 
merah hitam, hijau hitam, biru hitam, dan seterusnya sampai 12 pasang.

Contoh Soal 9: 

Terdapat angka 3, 4, 5, 6, 7 yang hendak disusun menjadi suatu bilangan dengan tiga digit. 
Berapa banyak bilangan yang dapat disusun bila angka boleh berulang?

Jawaban:

Angka terdiri dari 3, 4, 5, 6, 7 dengan total ada lima angka. Dan membutuhkan tiga digit angka dari kombinasi lima angka tersebut secara acak. Tiga digit terdiri dari angka ratusan, puluhan dan satuan. Karena angka boleh berulang maka angka ratusan, puluhan dan satuan dapat diisi dengan kelima angka tersebut sehingga :

 5 x 5 x 5 = 125 kombinasi angka.
 
Contoh Soal UN Tahun 2014:

Budi mempunyai koleksi 3 pasang sepatu dengan merk yang berbeda, dan 4 baju yang berlainan coraknya, serta 3 celana yang berbeda warna. Banyak cara berpakaian Budi dengan penampilan yang berbeda adalah ….

A.   10
B.   12
C.   22
D.   41
E.   36

Pembahasan :

3 pasang sepatu masing-masing bisa dipadukan dengan 4 corak baju dan 3 celana yang berbeda. Banyak cara yang mungkin adalah:

3 x 4 x 3 = 36

Jadi, banyak cara Budi berpakaian dengan penampilan berbeda adalah 36 cara (E).

Contoh Soal UN Tahun 2013
Empat siswa dan dua siswi akan duduk berdampingan. Apabila siswi selalu duduk paling pinggir, banyak cara mereka duduk adalah ….

A.   24
B.   48
C.   56
D.   64
E.   72

Pembahasan
Banyak cara 2 siswi duduk di pinggir:

2! = 2 x 1 = 2

Di antara kedua siswi tersebut ada 4 siswa. Banyak cara mereka duduk adalah:

4! = 4 x 3 x 2 x 1 = 24

Dengan demikian, banyak cara siswa dan siswi tersebut duduk adalah:

2 x 24 = 48

Jadi, banyak cara duduk empat siswa dan dua siswa tersebut adalah 48 cara (B).

Contoh Soal UN Tahun 2013
Dua keluarga yang masing-masing terdiri dari 2 orang dan 3 orang ingin foto bersama. Banyak posisi foto yang berbeda dengan anggota keluarga yang sama selalu berdampingan adalah ....

A.   24
B.   36
C.   48
D.   72
E.   96


Pembahasan
Pertama, anggaplah dua keluarga tersebut masing-masing merupakan dua kesatuan. Banyak posisi dua keluarga berfoto adalah:

2! = 2 x 1 = 2

Selanjutnya masing-masing keluarga melakukan tukar posisi antaranggota keluarga. Banyak posisi foto keluarga yang beranggotakan 2 orang adalah:

2! = 2 x 1 = 2

Banyak posisi foto keluarga yang beranggotakan 3 orang adalah:

3! = 3 x 2 x 1 = 6

Dengan demikian, banyak seluruh posisi foto dua keluarga tersebut adalah:

2 x 2 x 6 = 24

Jadi, banyak posisi foto yang berbeda dengan anggota keluarga yang sama selalu berdampingan adalah 24 posisi (A). 

Contoh Soal UN Tahun 2014 
Pada suatu rapat terdapat 10 orang yang saling berjabat tangan. Banyak jabatan tangan tersebut adalah ....

A.   90
B.   50
C.   45
D.   25
E.   20


Pembahasan
Setiap orang akan berjabat tangan sebanyak 9 kali. Karena ada 10 orang, banyak jabatan yang terjadi adalah:

9 x 10 = 90

Jadi, banyak jabatan tangan yang terjadi dari 10 orang  adalah 90 (A).


Contoh Soal :

Berapa cara yang dapat diperoleh untuk memilih posisi seorang tekong, apit kiri, dan apit kanan dari 15 atlet sepak takraw pelatnas SEA GAMES jika tidak ada posisi yang rangkap? (Tekong adalah pemain sepak takraw yang melakukan sepak permulaan).

Jawaban :

Untuk posisi tekong.

Posisi tekong dapat dipilih dengan 15 cara dari 15 atlet pelatnas yang tersedia.

Untuk posisi apit kiri.

Dapat dipilih dengan 14 cara dari 14 atlet yang ada (1 atlet lagi tidak terpilih karena menjadi tekong).

Untuk posisi apit kanan.

Cara untuk memilih apit kanan hanya dengan 13 cara dari 13 atlet yang ada (2 atlet tidak dapat dipilih karena telah menjadi tekong dan apit kiri).

Dengan demikian, banyak cara yang dilakukan untuk memilih posisi dalam regu sepak takraw adalah 15 x 14 x 13 = 2.730 cara.

 
Contoh soal dan jawaban

1. Dari angka-angka 1, 2, 3, 4, 5, 6, akan disusun suatu bilangan yang terdiri dari 3 angka berbeda. Banyaknya bilangan yang dapat disusun adalah ... 

a.	18

b.	36

c.	60

d.	120

e.	216



2. Dari angka-angka 2, 3, 5, 7, dan 8 disusun bilangan yang terdiri atas tiga angka yang berbeda. Banyak bilangan yang dapat disusun adalah ... 

a.	10

b.	15

c.	20

d.	48

e.	60


3. Dari angka-angka 1,2,3,4,5, dan 6 akan disusun suatu bilangan terdiri dari empat angka. Banyak bilangan genap yang dapat tersusun dan tidak ada angka yang berulang adalah ... 

a.	120

b.	180

c.	360

d.	480

e.	648 


4. Dari angka-angka 3,4,5,6, dan 7 akan dibuat bilangan terdiri dari empat angka berlainan. Banyaknya bilangan kurang dari 6.000 yang dapat dibuat adalah …. 

a.	24

b.	36

c.	48

d.	72

e.	96


5. Banyaknya bilangan antara 1.000 dan 4.000 yang dapat disusun dari angka-angka 1,2,3,4,5,6 dengan tidak ada angka yang sama adalah ... 

a.	72

b.	80

c.	96

d.	120

e.	180


6. Perjalanan dari Surabaya ke Sidoarjo bisa melalui dua jalan dan dari Sidoarjo ke Malang bisa melalui tiga jalan. Banyaknya cara untuk bepergian dari Surabaya ke Malang melalui Sidoarjo ada ... 

a.	1 cara

b.	2 cara

c.	3 cara

d.	5 cara

e.	6 cara


7. Suatu keluarga yang tinggal di Surabaya ingin liburan ke Eropa via Arab Saudi. Jika rute dari Surabaya ke Arab Saudi sebanyak 5 rute penerbangan, sedangkan Arab Saudi ke Eropa ada 6 rute, maka banyaknya semua pilihan rute penerbangan dari Surabaya ke Eropa pergi pulang dengan tidak boleh melalui rute yang sama adalah ... 

a.	900

b.	800

c.	700

d.	600

e.	460


8. Jika seorang ibu mempunyai 3 kebaya, 5 selendang, dan 2 buah sepatu, maka banyaknya komposisi pemakaian kebaya, selendang, dan sepatu adalah ... 

a.	6 cara

b.	8 cara

c.	10 cara

d.	15 cara

e.	30 cara

9. Seorang ingin melakukan pembicaraan melalui sebuah wartel. Ada 4 buah kamar bicara dan ada 6 buah nomor yang akan dihubungi. Banyak susunan pasangan kamar bicara dan nomor telepon yang dapat dihubungi adalah ... 

a.	10

b.	24

c.	360

d.	1.296

e.	4.096

10. Bagus memiliki koleksi 5 macam celana panjang dengan warna berbeda dan 15 kemeja dengan corak berbeda. Banyak cara Bagus berpakaian dengan penampilan berbeda adalah ... 

a.	5 cara

b.	15 cara

c.	20 cara

d.	30 cara

e.	75 cara



kunci jawaban :

1. D

2. E

3. B 

4. D

5. E

6. E

7. D

8. E

9. B

10.E


1.	UN 2012 BHS/A13
Dari 6 orang calon pengurus termasuk Doni akan dipilih ketua, wakil, dan bendahara. Jika Doni terpilih sebagai ketua maka banyak pilihan yang mungkin terpilih sebagai wakil dan bendahara adalah … pilihan

a.	12

b.	16

c.	20

d.	25

e.	30

2.	UN 2012 BHS/C37
Suatu regu pramuka terdiri dari 7 orang. Jika dipilih ketua, sekretaris, dan bendahara, maka banyak pasangan yang mungkin akan terpilih adalah …

a.	100

b.	110

c.	200

d.	210

e.	300

3.	UN 2010 BAHASA PAKET A 
Dalam rangka memperingati HUT RI, Pak RT membentuk tim panitia HUT RI yang dibentuk dari 8 pemuda untuk dijadikan ketua panitia, sekretaris, dan bendahara masing-masing 1 orang. Banyaknya cara pemilihan tim panitia yang dapat disusun adalah …

a.	24

b.	56

c.	168

d.	336

e.	6720


4.	UN 2012 IPS/B25
Dari 7 orang pengurus suatu ekstrakurikuler akan dipilih seorang ketua, wakil ketua, sekretaris, bendahara, dan humas. Banyak cara pemilihan pengurus adalah ….

a.	2.100

b.	2.500

c.	2.520


d.	4.200

e.	8.400

5.	UN 2010 IPS PAKET B 
Dari 7 orang pengurus suatu ekstrakurikuler akan dipilih seorang ketua, wakil ketua, sekretaris, bendahara, dan humas. Banyak cara pemilihan pengurus adalah …

a.	2.100

b.	2.500

c.	2.520

d.	4.200


e.	8.400

6.	UN 2012 BHS/B25
Dari 7 orang pelajar berprestasi di suatu sekolah akan dipilih 3 orang pelajar berprestasi I, II, dan III. Banyaknya cara susunan pelajar yang mungkin terpilih sebagai pelajar berprestasi I, II, dan III adalah …

a.	21

b.	35

c.	120

d.	210

e.	720


7.	UN 2010 IPS PAKET A 
Dalam kompetisi bola basket yang terdiri dari 10 regu akan dipilih juara 1, 2, dan 3. Banyak cara memilih adalah …

a.	120

b.	360

c.	540

d.	720

e.	900

8.	UN 2011 IPS PAKET 46
Jika seorang penata bunga ingin mendapatkan informasi penataan bunga dari 5 macam bunga yang berbeda, yaitu B1, B2, …, B5 pada lima tempat yang tersedia, maka banyaknya formasi yang mungkin terjadi adalah …

a.	720

b.	360

c.	180

d.	120

e.	24

9.	UN 2011 IPS PAKET 12 
Banyak cara memasang 5 bendera dari negara yang berbeda disusun dalam satu baris adalah …

a.	20

b.	24

c.	69

d.	120

e.	132

10.	UN 2008 BAHASA PAKET A/B 
Di depan sebuah gedung terpasang secara berjajar sepuluh tiang bendera. Jika terdapat 6 buah bendera yang berbeda, maka banyak cara berbeda menempatkan bendera-bendera itu pada tiang-tiang tersebut adalah …

a.	10!6!

b.	10!4!

c.	6!4!

d.	10!2!

e.	6!2!

11.	UN 2010 BAHASA PAKET A/B 
Susunan berbeda yang dapat dibentuk dari kata “DITATA” adalah …

a.	90

b.	180

c.	360

d.	450

e.	720

12.	UN 2008 BAHASA PAKET A/B 
Nilai kombinasi 8C3 sama dengan …

a.	5

b.	40

c.	56

d.	120

e.	336

13.	UN 2009 BAHASA PAKET A/B 
Diketahui himpunan A = {1, 2, 3, 4, 5} Banyak himpunan bagian A yang banyak anggotanya 3 adalah …

a.	6

b.	10

c.	15

d.	24

e.	30

14.	UN 2012 BHS/A13 
Banyaknya cara memilih 3 orang utusan dari 10 orang calon untuk mengikuti suatu perlombaan adalah …

a.	120

b.	180

c.	240

d.	360

e.	720

15.	UN 2010 IPS PAKET B 
Banyak cara menyusun suatu regu cerdas cermat yang terdiri dari 3 siswa dipilih dari 10 siswa yang tersedia adalah …

a.	80

b.	120

c.	160

d.	240

e.	720

16.	UN 2010 BAHASA PAKET A/B 
Banyak kelompok yang terdiri atas 3 siswa berbeda dapat dipilih dari 12 siswa pandai untuk mewakili sekolahnya dalam kompetisi matematika adalah …

a.	180

b.	220

c.	240

d.	420

e.	1.320

17.	UN 2011 IPS PAKET 12 
Dari 20 kuntum bunga mawar akan diambil 15 kuntum secara acak. Banyak cara pengambilan ada …

a.	15.504

b.	12.434

c.	93.024

d.	4.896

e.	816

18.	UN 2012 BHS/B25 
Lima orang bermain bulutangkis satu lawan satu secara bergantian. Banyaknya pertandingan adalah …

a.	5

b.	10

c.	15

d.	20

e.	25

19.	UN 2012 BHS/C37 
Dari 8 pemain basket akan dibentuk tim inti yang terdiri dari 5 pemain. Banyaknya susunan tim inti yang mungkin terbentuk adalah …

a.	56

b.	36

c.	28

d.	16

e.	5

20.	UN 2011 IPS PAKET 46 
Kelompok tani Suka Maju terdiri dari 6 orang yang berasal dari dusun A dan 8 orang berasal dari dusun B. Jika dipilih 2 orang dari dusun A dan 3 orang dari dusun B untuk mengikuti penelitian tingkat kabupaten, maka banyaknya susunan kelompok yang mungkin terjadi adalah …

a.	840

b.	720

c.	560

d.	350

e.	120

21.	UN 2009 IPS PAKET A/B 
Dari 20 orang siswa yang berkumpul, mereka saling berjabat tangan, maka banyaknya jabatan tangan yang terjadi adalah …

a.	40

b.	80


c.	190

d.	360

e.	400

22.	UN 2011 BHS PAKET 12 
Dari 10 warna berbeda akan dibuat warna-warna baru yang berbeda dari campuran 4 warna dengan banyak takaran yang sama. Banyaknya warna baru yang mungkin dibuat adalah … warna

a.	200

b.	210

c.	220

d.	230

e.	240

23.	UN 2010 BAHASA PAKET A 
Seorang ibu mempunyai 8 sahabat. Banyak komposisi jika ibu ingin mengundang 5 sahabatnya untuk makan malam adalah …

a.	8! 5! 

b.	8! 3! 

c.	8!3!

d.	8!5!

e.	8!5!3!

24.	UN 2008 BAHASA PAKET A/B 
Seorang peserta ujian harus mengerjakan 6 soal dari 10 soal yang ada. Banyak cara peserta memilih soal ujian yang harus dikerjakan adalah …

a.	210

b.	110

c.	230

d.	5.040

e.	5.400 


KUNCI JAWABAN

1.C

2.D

3.D

4.C

5.C

6.D

7.C

8.D

9.D

10.B

11.D

12.C

13.B

14.A

15.B

16.B

17.A

18.B

19.A

20.A

21.C

22.B

23.E

24.A





Questions and Answers




1. 
Sintia hendak bepergian dari kota A ke kota C melalui kota B. Dari kota A ke kota B terdapat dua jalan dan dari kota B ke kota C terdapat 3 jalan. Banyak jalan yang dapat ditempuh untuk bepergian dari kota A menuju kota C adalah 5 cara.

A. 
True

B. 
False

2. 
Dari huruf-huruf E, T, I, K, dan A akan dibentuk susunan huruf yang saling berbeda. Banyak cara untuk menyusun huruf-huruf itu jika huruf pertama dimulai dengan huruf hidup (vokal) adalah ....

3. 
Diketahui 5 buah angka 2, 3, 4, 5, dan 6 akan disusun bilangan-bilangan genap yang terdiri dari tiga angka. Banyak cara untuk menyusun bilangan itu bila tidak boleh mempunyai angka yang sama adalah ....

4. 
Akan dibuat nomor-nomor kode terdiri dari dua angka dan dua huruf yang saling berbeda. Banyaknya nomor kode tersebut adalah ....

A. 
52.650

B. 
58.500

C. 
70.200

D. 
60.000

E. 
65.000

5. 
Banyaknya bilangan yang dapat dibentuk antara 400 sampai 900 dari angka 3, 4, 5, dan 6 yang genap adalah

A. 
8

B. 
12

C. 
16

D. 
24

E. 
48

6. 
Pada suatu pesta dihadiri 75 orang dan mereka saling jabat tangan. Banyaknya jabat tangan yang terjadi adalah 2075

A. 
True

B. 
False

7. 
Dari angka 1, 2, 3, 5. 6. dan 7 disusun bilangan yang terdiri atas tiga angka yang berlainan. Banyak bilangan yang kuran dari 400 dari angka-angka tersebut adalah ....

8. 
Ada 4 jalan raya yang menghubungkan kota A dan kota B. Antara kota B dan kota C ada 3 jalan raya. Jose bepergian dari kota A ke kota C melalui kota B, kemudian kembali dari kota C ke kota A juga melalui kota B. Jika jalur jalan yang ditempuh pergi pulang yang harus berbeda, maka banyak yang dapat dilalui ada ....

A. 
6

B. 
12

C. 
18

D. 
72

E. 
144

9. 
Agatha, Bunga, Cantika dan Diana akan berfoto selfi bersama secara berdampingan. Banyak caya mereka berfoto dengan Agatha dan Bunga selalu berdampingan adalah 12 cara.

A. 
True

B. 
False

10. 
Dari tujuh orang pengurus kegiatan ekstrakurikuler akan dipilih seorang ketua, wakil, sekretaris, dan bendahara dan humas. Banyak cara pemilihan pengurus adalah ....

A. 
2.100

B. 
2.500

C. 
2,520

D. 
4.200

E. 
8.400

11. 
Lima anak akan duduk pada tiga kursi A, B, dan C secara berdampingan. Banyaknya kemungkinan mereka duduk adalah 
....

12. 
Anthony, Bram, Candra, dan Daniel akan bekerja secara bergilir. Banyaknya urutan bekerja yang dapat disusun dengan Anthony selau pada giliran terakhir adalah 6 cara

A. 
True

B. 
False





\section{Permutasi}
	Dalam kehidupan sehari-hari kita sering menghadapi masalah pengaturan suatu obyek yang terdiri dari beberapa unsur, baik yang disusun dengan mempertimbangkan urutan sesuai dengan posisi yang diinginkan maupun yang tidak. Misalnya menyusun kepanitiaan yang terdiri dari Ketua, Sekretaris dan Bendahara dimana urutan untuk posisi tersebut dipertimbangkan atau memilih beberapa orang untuk mewakili sekelompok orang dalam mengikuti suatu kegiatan yang dalam hal ini urutan tidak menjadi pertimbangan. Dalam matematika, penyusunan obyek yang terdiri dari beberapa unsur dengan mempertimbangkan urutan disebut dengan permutasi, sedangkan yang tidak mempertimbangkan urutan disebut dengan kombinasi.


Masalah penyusunan kepanitiaan yang terdiri dari Ketua, Sekretaris dan Bendahara dimana urutan dipertimbangkan merupakan salah satu contoh permutasi. Jika terdapat 3 orang (misalnya Amir, Budi dan Cindy) yang akan dipilih untuk menduduki posisi tersebut, maka dengan menggunakan Prinsip Perkalian kita dapat menentukan banyaknya susunan panitia yang mungkin, yaitu:


• Pertama menentukan Ketua, yang dapat dilakukan dalam 3 cara.


• Begitu Ketua ditentukan, Sekretaris dapat ditentukan dalam 2 cara.


• Setelah Ketua dan Sekretaris ditentukan, Bendahara dapat ditentukan dalam 1 cara.

 
• Sehingga banyaknya susunan panitia yang mungkin


Secara formal Permutasi didefinisikan sebagai berikut:


Permutasi dari n unsur yang berbeda x1,x2,...,xn adalah pengurutan dari n unsur tersebut.


CONTOH :


Tentukan permutasi dari 3 huruf yang berbeda, misalnya ABC !


Permutasi dari huruf ABC adalah ABC, ACB, BAC, BCA, CAB, CBA. Sehingga terdapat 6 permutasi dari huruf ABC.

TEOREMA 2


Terdapat n permutasi dari n unsur yang berbeda.


Bukti


Asumsikan bahwa permutasi dari undur yang berbeda merupakan aktivitas yang terdiri dari langkah ang beurutan. Langkah pertama adalah memilih unsur pertama yang bisa dilakukandebgn n cara. Langkah kedua adlah memilih unsuur pertama sudah terpilih. Lanjutkan langkah tersebut sampai langkah ke n yang bisa dilakukan dengan 1 cara. Berdasarkan prinsip perkalian, terdapat


n(n-1)(n-2)...2.1=n!


permutasi n unsur berbeda

Contoh :


Berapa banyak permutasi dari huruf ABC ?


Terdapat 3.2.1 = 6 permutasi dari huruf ABC

Berapa banyak permutasi dari huruf ABCDEF jika subuntai ABC harus selalu muncul bersama?


Karena subuntai ABC harus selalu muncul bersama, maka subuntai ABC bisa dinyatakan sebagai satu unsur. Dengan demikian terdapat 4 unsur yang dipermutasikan, sehingga banyaknya permutasi adalah 4.3.2.1 = 24.

DEFINISI


Permutasi-r dari n unsur yang berbeda x1,x2,...,xn adalah pengurutan dari sub-himpunan dengan r anggota dari himpunan {x1,x2,...,xn}. Banyaknya permutasi-r dari n unsur yang berbeda dinotasikan dengan P(n,r).

Tentukan permutasi-3 dari 5 huruf yang berbeda, misalnya ABCDE.


Permutasi-3 dari huruf ABCDE adalah

ABC ABD ABE ACB ACD ACE 

ADB ADC ADE AEB AEC AED

BAC BAD BAE BCA BCD BCE 

BDA BDC BDE BEA BEC BED 

CAB CAD CAE CBA CBD CBE 

CDA CDB CDE CEA CEB CED 

DAB DAC DAE DBA DBC DBE 

DCA DCB DCE DEA DEB DEC 

EAB EAC EAD EBA EBC EBD 

ECA ECB ECD EDA EDB EDC

Sehingga banyaknya permutasi-3 dari 5 huruf ABCDE adalah 60.


TEOREMA 3

Banyaknya permutasi-r dari n unsur yang berbeda adalah

\includegraphics[width = 8cm, height= 4cm]{Pictures/herlin1.png}

Asumsikan bahwa permutasi-r n unsur yang berbeda merupakan aktifitas yang terdiri dari r langkah yang merupakan ktifitas yang terdiri dari r langkah yang berurutan. Langkah pertama adalah memilih unsur pertama yang bisa dilakukan dengan n cara. angkah kedua adalah memilih unsur kedu yang bisa dilakukan dengan n-1 cara karena unsur pertama sudah terpilih. Lanjutkan langkah tersebut sampai dengan n-r +1 cara. Berdasarkan prinsip perkalian Diperoleh


\includegraphics[width = 12cm, height= 6cm]{Pictures/herlin2.png}

Gunakan Teorema 3.2 untuk menentukan permutasi-3 dari 5 huruf yang berbeda, misalnya ABCDE.


Karena r = 3 dan n = 5 maka permutasi-3 dari 5 huruf ABCDE adalah

\includegraphics[width = 12cm, height= 4cm]{Pictures/herlin3.png}

Jadi banyaknya permutasi-3 dari 5 huruf ABCDE adalah 60.

\section{kombinasi}


Berbeda dengan permutasi yang urutan menjadi pertimbangan, pada kombinasi urutan tidak dipertimbangkan. Misalnya pemilihan 3 orang untuk mewakili kelompak 5 orang (misalnya Dedi, Eka, Feri, Gani dan Hari) dalam mengikuti suatu kegiatan. Dalam masalah ini, urutan tidak dipertimbangkan karena tidak ada bedanya antara Dedi, Eka dan Feri dengan Eka, Dedi dan Feri. Dengan mendata semua kemungkinan 3 orang yang akan dipilih dari 5 orang yang ada, diperoleh:

\includegraphics[width = 12cm, height= 4cm]{Pictures/herlin4.png}

Sehingga terdapat 10 cara untuk memilih 3 orang dari 5 orang yang ada.

Selanjutnya kita dapat kombinasikan secara formal.

Seperti keterangan diatas.



DEFINISI


Kombinasi-r dari n unsur yang berbeda x1,x2,...,xn adalah seleksi tak terurut r anggota dari himpunan {x1,x2,...,xn} (sub-himpunan dengan r unsur). Banyaknya kombinasi-r dari n unsur yang berbeda dinotasikan dengan C(n,r) atau (n r).

CONTOH :

Tentukan kombinasi-3 dari 5 huruf yang berbeda, misalnya ABCDE.
Kombinasi-3 dari huruf ABCDE adalah

\includegraphics[width = 12cm, height= 4cm]{Pictures/herlin5.png}

Sehingga banyaknya kombinasi-3 dari 5 huruf ABCDE adalah 10.

Teorema 3

Banyaknya kombinasi-r dari n unsur yang berbeda adalah

\includegraphics[width = 12cm, height= 4cm]{Pictures/herlin6.png}

Bukti.

• Langkah pertama adalah menghitung kombinasi-r dari n, yaitu C(n,r). 


• Langkah kedua adalah mengurutkan r unsur tersebut, yaitu r!.

 Dengan demikian
 
 
Pembuktian dilakukan dengan menghitung permutasi dari n unsur yang berbeda dengan cara berikut ini.

\includegraphics[width = 12cm, height= 6cm]{Pictures/herlin7.png}

seperti yang diinginkan

Gunakan Teorema 3.3 untuk menentukan kombinasi-3 dari 5 huruf yang berbeda, misalnya ABCDE.


Karena r = 3 dan n = 5 maka kombinasi-3 dari 5 huruf ABCDE adalah

\includegraphics[width = 12cm, height= 4cm]{Pictures/herlin8.png}


Jadi banyaknya kombinasi-3 dari 5 huruf ABCDE adalah 10.


Contoh

Berapa banyak cara sebuah panitia yang terdiri dari 4 orang bisa dipilih dari 6 orang


Karena panitia yang terdiri dari 4 orang merupakan susunan yang tidak terurut, maka masalah ini merupakan kombinasi-4 dari 6 unsur yang tersedia. Sehingga dengan mengunakan Teorema 3.3 dimana n = 6 dan r = 4 diperoleh: 


\includegraphics[width = 12cm, height= 4cm]{Pictures/herlin9.png}

Jadi terdapat 15 cara untuk membentuk sebuah panitia yang terdiri dari 4 orang bisa dipilih dari 6 orang.

Contoh :

Berapa banyak cara sebuah panitia yang terdiri dari 2 mahasiswa dan 3 mahasiswi yang bisa dipilih dari 5 mahasiswa dan 6 mahasiswi?

Pertamai, memilih 2 mahasiswa dari 5 mahasiswa yang ada, yaitu:

\includegraphics[width = 12cm, height= 4cm]{Pictures/herlin10.png}

Kedua, memilih 3 mahasiswi dari 6 mahasiswi yang ada, yaitu:


\includegraphics[width = 12cm, height= 4cm]{Pictures/herlin11.png}

Sehingga terdapat 10.20 = 200 cara untuk membentuk sebuah panitia yang terdiri dari 2 mahasiswa dan 3 mahasiswi yang bisa dipilih dari 5 mahasiswa dan 6 mahasiswi?

Kalau pada pembahasan permutasi sebelumnya unsur-unsur yang diurutkan berbeda, pada bagian ini akan dibahas permutasi yang digeneralisasikan dengan membolehkan pengulangan unsur-unsur yang akan diurutkan, dengan kata lain unsur-unsurnya boleh sama.



Misalkan kita akan mengurutkan huruf-huruf dari kata KAKIKUKAKU. Karena huruf-huruf pada kata tersebut ada yang sama, maka banyaknya permutasi bukan 10!, tetapi kurang dari 10!.


Untuk mengurutkan 10 huruf pada kata KAKIKUKAKU dapat dilakukan dengan cara:

Kalau pada pembahasan permutasi sebelumnya unsur-unsur yang diurutkan berbeda, pada bagian ini akan dibahas permutasi yang digeneralisasikan dengan membolehkan pengulangan unsur-unsur yang akan diurutkan, dengan kata lain unsur-unsurnya boleh sama.
Misalkan kita akan mengurutkan huruf-huruf dari kata KAKIKUKAKU. Karena huruf-huruf pada kata tersebut ada yang sama, maka banyaknya permutasi bukan 10!, tetapi kurang dari 10!.
Untuk mengurutkan 10 huruf pada kata KAKIKUKAKU dapat dilakukan dengan cara:

-Asumsikan masalah ini dengan 10 posisi kosong yang akan diisi dengana huruf-huruf pada kata KAKIKUKAKU.

-Pertama menempatkan 5 huruf K pada 10 posisi kosong, yang dapat dilkukan dalam c(10,5) cara

-Setelah 5 huruf k ditempatkan, maka terdapat 10-5 =5 posisi kosong

-Berikutnya adlah menempatkan 2 huruf A pada 5 posisi kosong, yang dapat dilakukan dalam c(5,2) cara. b begitu 2huruf A ditempatkan, terdapat C(3,2) cara untuk menempatkan 2 huruf A ditempatkan, terdapat C(3,2) cara untuk menempatkan 2 huruf U pada 3 posisi kosong yang ada


- Akhirnya terdapat C(1,1) cara untuk menempatkan 1 huruf I pada 1 posisi kosong yang tersisa

 
\includegraphics[width = 12cm, height= 4cm]{Pictures/herlin11.png}

Jadi banyaknya cara untuk mengurutkan huruf-huruf dari kata KAKIKUKAKU adalah 7560.

Secara umum banyaknya permutasi dari obyek yang mempunyai beberapa unsur sama dapat dijabarkan seperti pada teorema berikut ini.

TEOREMA

Misalkan X merupakan sebuah barisan yang mempunyai n unsur, dimana terdapat n1 unsur yang sama untuk jenis 1, n2 unsur yang sama untuk jenis 2 dan seterusnya sampai nt unsur yang sama untuk jenis t. Banyaknya permutasi dari barisan X adalah

\includegraphics[width = 12cm, height= 6cm]{Pictures/herlin12.png}

%------------------------------------------------

\section{Notations}\index{Notations}

\begin{notation}
Given an open subset $G$ of $\mathbb{R}^n$, the set of functions $\varphi$ are:
\begin{enumerate}
\item Bounded support $G$;
\item Infinitely differentiable;
\end{enumerate}
a vector space is denoted by $\mathcal{D}(G)$. 
\end{notation}

%------------------------------------------------

\section{Remarks}\index{Remarks}

This is an example of a remark.

\begin{remark}
The concepts presented here are now in conventional employment in mathematics. Vector spaces are taken over the field $\mathbb{K}=\mathbb{R}$, however, established properties are easily extended to $\mathbb{K}=\mathbb{C}$.
\end{remark}

%------------------------------------------------

\section{Corollaries}\index{Corollaries}

This is an example of a corollary.

\begin{corollary}[Corollary name]
The concepts presented here are now in conventional employment in mathematics. Vector spaces are taken over the field $\mathbb{K}=\mathbb{R}$, however, established properties are easily extended to $\mathbb{K}=\mathbb{C}$.
\end{corollary}

%------------------------------------------------

\section{Propositions}\index{Propositions}

This is an example of propositions.

\subsection{Several equations}\index{Propositions!Several Equations}

\begin{proposition}[Proposition name]
It has the properties:
\begin{align}
& \big| ||\mathbf{x}|| - ||\mathbf{y}|| \big|\leq || \mathbf{x}- \mathbf{y}||\\
&  ||\sum_{i=1}^n\mathbf{x}_i||\leq \sum_{i=1}^n||\mathbf{x}_i||\quad\text{where $n$ is a finite integer}
\end{align}
\end{proposition}

\subsection{Single Line}\index{Propositions!Single Line}

\begin{proposition} 
Let $f,g\in L^2(G)$; if $\forall \varphi\in\mathcal{D}(G)$, $(f,\varphi)_0=(g,\varphi)_0$ then $f = g$. 
\end{proposition}

%------------------------------------------------

\section{Examples}\index{Examples}

This is an example of examples.

\subsection{Equation and Text}\index{Examples!Equation and Text}

\begin{example}
Let $G=\{x\in\mathbb{R}^2:|x|<3\}$ and denoted by: $x^0=(1,1)$; consider the function:
\begin{equation}
f(x)=\left\{\begin{aligned} & \mathrm{e}^{|x|} & & \text{si $|x-x^0|\leq 1/2$}\\
& 0 & & \text{si $|x-x^0|> 1/2$}\end{aligned}\right.
\end{equation}
The function $f$ has bounded support, we can take $A=\{x\in\mathbb{R}^2:|x-x^0|\leq 1/2+\epsilon\}$ for all $\epsilon\in\intoo{0}{5/2-\sqrt{2}}$.
\end{example}

\subsection{Paragraph of Text}\index{Examples!Paragraph of Text}

\begin{example}[Example name]
\lipsum[2]
\end{example}

%------------------------------------------------

\section{Exercises}\index{Exercises}

This is an example of an exercise.

\begin{exercise}
This is a good place to ask a question to test learning progress or further cement ideas into students' minds.
\end{exercise}

%------------------------------------------------

\section{Problems}\index{Problems}

\begin{problem}
What is the average airspeed velocity of an unladen swallow?
\end{problem}

%------------------------------------------------

\section{Vocabulary}\index{Vocabulary}

Define a word to improve a students' vocabulary.

\begin{vocabulary}[Word]
Definition of word.
\end{vocabulary}

%----------------------------------------------------------------------------------------
%	PART
%----------------------------------------------------------------------------------------

\part{Part Two}

%----------------------------------------------------------------------------------------
%	CHAPTER 3
%----------------------------------------------------------------------------------------
\chapterimage{chapter_head_2.pdf} % Chapter heading image

\chapter{Statistika}

\section{A. Penyajian Data Dalam Bentuk Diagram}\index{A. Penyajian Data Dalam Bentuk Diagram}

Statistika adalah cabang dari matematika terapan yang mempunyai cara-cara, maksudnya mengkaji/membahas, mengumpulkan, dan menyusun data, mengolah dan menganalisis data, serta menyajikan data dalam bentuk kurva atau diagram, menarik kesimpulan, menasirkan parameter, dan menguji hipotesa yang didasarkan pada hasil pengolahan data. Contoh : statistik jumlah lulusan siswa SMA dari tahun ke tahun, statistic jumlah kendaraan yang melewatu suau jalan, statistic perdagangan antara Negara-negara di Asia, dan sebagainya.

1.	Diagram Garis

Penyajian data statistik dengan menggunakan diagram berbentuk garis lurus disebut diagram garis lurus atau diagram garis. Diagram garis biasanya digunakan untuk menyajikan data statistic yang diperoleh berdasarkan pengamatan dari waktu ke waktu sevara berurutan.
	Sumbu X menunjukan waktu-waktu pengamatan, sedangkan sumbu Y menunjukkan nilai data pengamatan untuk suatu waktu tertentu. Kumpulan waktu dan pengamatan membentuk titik-titik pada bidang XY, selanjutnya kolom dari tiap dua titik yang berdekatan tadi dihubungkan dengan garis lurus sehingga akan diperoleh diagram garis atau grafik garis. Diagram ini biasanya digunakan untuk menggambarkan suatu kondisi yang berlangsung secara kontinu, misalnya perkembangan nilai tukar mata uang suatu Negara terhadap nilai tukar Negara lain, jumlah penjualan setiap waktu tertentu, dan jumlah penduduk suatu daerah setiap periode tertentu. Untuk lebih jelasnya, perhatikan contoh soal berikut.
	
Contoh soal 1

Fluktuasi nilai tukar rupiah terhadap doal AS dari tanggal 18 Februari 2008 sampai dengan tanggal 22 Februari 2008 ditunjukkan oleh tabel berikut.

\includegraphics[width = 12cm, height= 1.5cm]{Pictures/Gb1_diana.png}

Nyatakan data di atas dalam bentuk diagram garis.
Penyelesaian
Jika digambar dengan menggunakan diagram garis adalah sebagai berikut.

\includegraphics[width = 9cm, height= 6cm]{Pictures/Gb2_diana.png}


Contoh Soal 2

Sebuah dealer mobil sejak tahun 1995 hingga akhir tahun 2004 selalu mencatat jumlah mobil yang terjual setiap tahun sebagai berikut.

\includegraphics[width = 12cm, height= 1.5cm]{Pictures/Gb3_diana.png}

Buatlah diagram garis untuk data tersebut.

Penyelesaian

\includegraphics[width = 10.6cm, height= 6.1cm]{Pictures/Gb4_diana.png}


Dari diagram tersebut, tampak penjualan mobil terbanyak pada tahun 2001. Dari tahun 1995-1997, penjualan mobil cenderung mengalami kenaikan dan tahun 1998-1999 cenderung mengalami penurusan.

Contoh Soal 3

Sebuah perusahaan yang memproduksi barang elektronik mencatat akumulasi biaya produksi tahunan dan akumulasi nilai penjualan selama sepuluh tahun dari tahun 1995 sampai dengan 2004 sebagai berikut (dalam jutaan rupiah)

\includegraphics[width = 12.8cm, height= 4cm]{Pictures/Gb5_diana.png}

Buatlah diagram garis untuk data tersebut

Penyelesaian

Diagram garis untuk akumulasi biaya produksi dan akumulasi nilai penjualan adalah sebagai berikut.

\includegraphics[width = 10.6cm, height= 6.4cm]{Pictures/Gb6_diana.png}

Dari gambar di atas Anda dapat mengetahui bahwa perusahaan mulai memperoleh laba (keuntungan) di antara tahun 1999 dan 2000, yaitu pada saat kedua garis berpotongan. Titik potong kedua garis tersebut disebut titik pulang pokok (break event point).

	Diagram garis biasanya digunakan untuk menaksir atau memperkirakan data berdasarkan pola-pola yang telah diperoleh. Diagram pada Gambar 1.2 merupakan diagram garis tunggal. Adapun diagram pada Gambar 1.3 disebut diagram garis majemuk, dikatakan majemuk karena dalam satu gambar terdapat lebih dari satu garis. Diagram garis majemuk biasanya digunakan untuk membandingkan dua keadaan atau lebih yang mempunyai hubungan, misalnya diagram dua garis yang melukiskan akumulasi biaya produksi dan akumulasi nilai penjualan setiap tahun selama sepuluh tahun .

2.	Diagram Lingkaran


Diagram lingkaran adalah penyajian data statistik dengan menggunakan gambar yang berbentuk lingkaran. Bagian-bagian dari daerah lingkaran menunjukkan bagina-bagian atau persen dari keseluruhan. Untuk membuat diagram lingkaran, terlebih dahulu ditentukan besarnya presentase tiap objek terhadap keseluruhan data dan besarnya sudut pusat sektor lingkaran. Perhatikan contoh berikut ini.

Contoh soal 1


Ranah privat (pengaduan) darikoran Solo Pos pada tanggal 22 Februari 2008 ditunjukkan seperti pada tabel berikut .


\includegraphics[width = 9cm, height= 7cm]{Pictures/Gb7_diana.png}

Nyatakan data di atas dalam bentuk diagram lingkaran.
Penyelesaian
Sebelum data pada tabel di atas disajikan dengan diagram lingkaran, terlebih dahulu ditentukan besarnya sudut dalam lingkaran dari data tersebut.


\includegraphics[width = 10.16cm, height= 12.36cm]{Pictures/Gb8_diana.png}


Diagram lingkarannya adalah sebagai berikut.

\includegraphics[width = 10.78cm, height= 7.41cm]{Pictures/Gb9_diana.png}

Contoh soal 2


Berikut ini adalah data penjualan 6 jenis mobil dari suatu perusahaan pada kurun waktu 2002-2007.


\includegraphics[width = 12.75cm, height= 1.54cm]{Pictures/Gb10_diana.png}


Buatlah diagram lingkaran dari data di atas.

Penyelesaian

Untuk dapat menggambarkan diagram lingkaran, terlebih dahulu tentukan besar sudut masing-masing juring yang mewakili masing-masing jenis mobil (Jumlah penjualan 18 + 26 + 15 + 36 + 50 + 8 = 153)

\includegraphics[width = 7.33cm, height= 7.46cm]{Pictures/Gb11_diana.png}


Setelah menemukan besar sudut dari masing-masing jenis mobil yang terjual, kalian dapat menggambarkannya dalam lingkaran .
Kalian juga dapat menyatakan diagram lingkaran tersebut dalam bentuk persentase. Untuk menentukan persentase mobil jenis I, caranya adalah 


\includegraphics[width = 9.1cm, height= 7.28cm]{Pictures/Gb12_diana.png}

\includegraphics[width = 11.48cm, height= 5.61cm]{Pictures/Gb13_diana.png}



\section{B. Penyajian Data Dalam Bentuk Tabel distribusi Frekuensi}\index{B. Penyajian Data Dalam Bentuk Tabel distribusi Frekuensi}

Selain dalam bentuk diagram, penyajian data juga dengan menggunakan tabel distribusi
frekuensi. Berikut ini akan dipelajari lebih jelas mengenai tabel distribusi frekuensi tersebut.


1. Ditribusi Frekuensi Tunggal
Data tunggal seringkali dinyatakan dalam bentuk daftar bilangan, namun kadangkala
dinyatakan dalam bentuk tabel distribusi frekuensi. Tabel distribusi frekuensi tunggal
merupakan cara untuk menyusun data yang relatif sedikit. Perhatikan contoh data berikut.
5, 4, 6, 7, 8, 8, 6, 4, 8, 6, 4, 6, 6, 7, 5, 5, 3, 4, 6, 6 , 8, 7, 8, 7, 5, 4, 9, 10, 5, 6, 7, 6, 4, 5, 7, 7, 4, 8, 7, 6

Dari data di atas tidak tampak adanya pola yang tertentu maka agar mudah dianalisis
data tersebut disajikan dalam tabel seperti di bawah ini

\includegraphics[width = 6cm, height= 4cm]{Pictures/1reska.png}

Daftar di atas sering disebut sebagai distribusi frekuensi dan karena datanya
tunggal maka disebut distribusi frekuensi tunggal.


2. Distribusi Frekuensi Bergolong

Tabel distribusi frekuensi bergolong biasa digunakan untuk menyusun data yang
memiliki kuantitas yang besar dengan mengelompokkan ke dalam interval-interval kelas
yang sama panjang. Perhatikan contoh data hasil nilai pengerjaan tugas Matematika
dari 40 siswa kelas XI berikut ini.


66 75 74 72 79 78 75 75 79 71

75 76 74 73 71 72 74 74 71 70

74 77 73 73 70 74 72 72 80 70

73 67 72 72 75 74 74 68 69 80



Apabila data di atas dibuat dengan menggunakan tabel distribusi frekuensi tunggal,
maka penyelesaiannya akan panjang sekali. Oleh karena itu dibuat tabel distribusi
frekuensi bergolong dengan langkah-langkah sebagai berikut.
a. Mengelompokkan ke dalam interval-interval kelas yang sama panjang, misalnya
65 – 67, 68 – 70, … , 80 – 82. Data 66 masuk dalam kelompok 65 – 67.
b. Membuat turus (tally), untuk menentukan sebuah nilai termasuk ke dalam kelas
yang mana.
c. Menghitung banyaknya turus pada setiap kelas, kemudian menuliskan banyaknya
turus pada setiap kelas sebagai frekuensi data kelas tersebut. Tulis dalam kolom
frekuensi.
d. Ketiga langkah di atas direpresentasikan pada tabel berikut ini.


\includegraphics[width = 6cm, height= 4cm]{Pictures/2reska.png}

Istilah-istilah yang banyak digunakan dalam pembahasan distribusi frekuensi
bergolong atau distribusi frekuensi berkelompok antara lain sebagai berikut.



a. Interval Kelas

Tiap-tiap kelompok disebut interval kelas atau sering disebut interval atau kelas
saja. Dalam contoh sebelumnya memuat enam interval ini.

65 – 67 Interval kelas pertama

68 – 70 Interval kelas kedua

71 – 73 Interval kelas ketiga

74 – 76 Interval kelas keempat

77 – 79 Interval kelas kelima

80 – 82 Interval kelas keenam


b. Batas Kelas

Berdasarkan tabel distribusi frekuensi di atas, angka 65, 68, 71, 74, 77, dan 80

merupakan batas bawah dari tiap-tiap kelas, sedangkan angka 67, 70, 73, 76, 79,

dan 82 merupakan batas atas dari tiap-tiap kelas.


c. Tepi Kelas (Batas Nyata Kelas)

Untuk mencari tepi kelas dapat dipakai rumus berikut ini.

\includegraphics[width = 6cm, height= 1cm]{Pictures/3reska.png}

Dari tabel di atas maka tepi bawah kelas pertama 64,5 dan tepi atasnya 67,5, tepi
bawah kelas kedua 67,5 dan tepi atasnya 70,5 dan seterusnya.

d. Lebar kelas
Untuk mencari lebar kelas dapat dipakai rumus:

Untuk mencari lebar kelas dapat dipakai rumus:

Lebar kelas = tepi atas – tepi bawah

Jadi, lebar kelas dari tabel diatas adalah 67,5 – 64,5 = 3.

e. Titik Tengah
Untuk mencari titik tengah dapat dipakai rumus:

\includegraphics[width = 8cm, height= 1cm]{Pictures/4reska.png}

Dari tabel di atas: titik tengah kelas pertama = 0.5 (67 + 65) = 66

titik tengah kedua =0.5 (70 + 68) = 69
 dan seterusnya.
 
 
3. Distribusi Frekuensi Kumulatif


Daftar distribusi kumulatif ada dua macam, yaitu sebagai berikut.

a. Daftar distribusi kumulatif kurang dari (menggunakan tepi atas).

b. Daftar distribusi kumulatif lebih dari (menggunakan tepi bawah).

Untuk lebih jelasnya, perhatikan contoh data berikut ini.

\includegraphics[width = 8cm, height= 2cm]{Pictures/5reska.png}

Dari tabel di atas dapat dibuat daftar frekuensi kumulatif kurang dari dan lebih
dari seperti berikut: 




\includegraphics[width = 12cm, height= 2cm]{Pictures/6reska.png}



4. Histogram


Dari suatu data yang diperoleh dapat disusun dalam tabel distribusi frekuensi dan
disajikan dalam bentuk diagram yang disebut histogram. Jika pada diagram batang,
gambar batang-batangnya terpisah maka pada histogram gambar batang-batangnya berimpit. Histogram dapat disajikan dari distribusi frekuensi tunggal maupun distribusi
frekuensi bergolong. Untuk lebih jelasnya, perhatikan contoh berikut ini.
Data banyaknya siswa kelas XI IPA yang tidak masuk sekolah dalam 8 hari berurutan
sebagai berikut: 

\includegraphics[width = 12cm, height= 1cm]{Pictures/7reska.png}

Berdasarkan data diatas dapat dibentuk histogramnya seperti berikut dengan membuat
tabel distribusi frekuensi tunggal terlebih dahulu

\includegraphics[width = 4cm, height= 4cm]{Pictures/8reska.png}

5.Poligon Frekuensi


Apabila pada titik-titik tengah dari histogram dihubungkan dengan garis dan batangbatangnya
dihapus, maka akan diperoleh poligon frekuensi. Berdasarkan contoh di atas
dapat dibuat poligon frekuensinya seperti gambar berikut ini.

\includegraphics[width = 4cm, height= 4cm]{Pictures/9reska.png}

Untuk lebih jelasnya, perhatikan contoh soal berikut ini.

Contoh soal

Hasil pengukuran berat badan terhadap 100 siswa SMP X digambarkan dalam distribusi
bergolong seperti di bawah ini. Sajikan data tersebut dalam histogram dan poligon frekuensi.


\includegraphics[width = 5cm, height= 2cm]{Pictures/10reska.png}

Penyelesaian

Histogram dan poligon frekuensi dari tabel di atas dapat ditunjukkan sebagai berikut

\includegraphics[width = 6cm, height= 4cm]{Pictures/11reska.png}

6. Poligon Frekuensi Kumulatif

Dari distribusi frekuensi kumulatif dapat dibuat grafik garis yang disebut poligon
frekuensi kumulatif. Jika poligon frekuensi kumulatif dihaluskan, diperoleh kurva yang
disebut kurva ogive. Untuk lebih jelasnya, perhatikan 

contoh soal berikut ini.

\includegraphics[width = 4cm, height= 4cm]{Pictures/12reska.png}

Hasil tes ulangan Matematika terhadap
40 siswa kelas XI IPA digambarkan dalam
tabel di samping.

a. Buatlah daftar frekuensi kumulatif kurang
dari dan lebih dari.

b. Gambarlah ogive naik dan ogive turun

Penyelesaian

a. Daftar frekuensi kumulatif kurang dari dan lebih dari adalah sebagai berikut.

\includegraphics[width = 8cm, height= 3cm]{Pictures/13reska.png}

b. Ogive naik dan ogive turun

Daftar frekuensi kumulatif kurang dari dan lebih dari dapat disajikan dalam bidang
Cartesius. 
Tepi atas (67,5; 70,5; …; 82,5) atau tepi bawah (64,5; 67,5; …; 79,5)
diletakkan pada sumbu X sedangkan frekuensi kumulatif kurang dari atau frekuensi
kumulatif lebih dari diletakkan pada sumbu Y.
Apabila titik-titik yang diperlukan
dihubungkan, maka terbentuk kurva yang disebut ogive. Ada dua macam ogive,
yaitu ogive naik dan ogive turun. Ogive naik apabila grafik disusun berdasarkan
distribusi frekuensi kumulatif kurang dari. Sedangkan ogive turun apabila berdasarkan
distribusi frekuensi kumulatif lebih dari.
Ogive naik dan ogive turun data di atas adalah
sebagai berikut:


\includegraphics[width = 8cm, height= 2cm]{Pictures/14reska.png}



\section{Ukuran Pemusatan Data}\index{Ukuran Pemusatan Data}

Ukuran pemusatan data disebut juga sembarang ukuran yang menunjukkan pusat sekumpulan data, yang telah diurutkan dari angka yang terkecil sampai terbesar atau sebaliknya dari angka yang terbesar sampai terkecil. Beberapa fungsi dari ukuran pemusatan data adalah untuk membandingkan dua data atau contoh, karena sangat sulit untuk membandingkan banyaknya anggota dari masing-masing anggota populasi atau banyaknya anggota data contoh. Nilai ukuran pemusatan ini dibuat sehingga dapat mewakili seluruh nilai pada data yang bersangkutan.

Ukuran pemusatan yang sering digunakan adalah mean, modus, dan median. Nilai tengah (mean) akan sangat dipengaruh nilai banyaknya data. Median yang sangat beragam sulit dalam penggunaan parameter populasi. Dan modus hanya digunakan untuk data ukuran yang besar.

Salah satu ukuran yang paling penting untuk menggambarkan suatu distribusi data adalah nilai pusat data pengamat. Setiap pengukuran aritmatika yang ditujukan untuk menggambarkan suatu nilai yang mewakili nilai pusat atau nilai sentral dari suatu gugus data (himpunan pengamatan) dikenal sebagai ukuran tendensi sentral. Biasanya Ukuran pemusatan data sering kali digunakan agar data yang diperoleh mudah untuk dipahami oleh siswa. Ukuran pemusatan data debagi menjadi mean yang digunakan untuk mengetahui nilai rata rata pada setiap himpunan angka, median digunakan untuk mengetahui suatu nilai tengah suatu himpunan angka, dan modus adalah data yang sering muncul.


%------------------------------------------------

\section{Mean}\index{Mean}

Mean yaitu suatu nilai rata rata dan di dapatkan dari sekumpulan data adalah jumlah seluruh data dibagi banyaknya data. Dengan mengetahui mean suatu data, maka variasi data yang lain akan mudah diperkirakan.atau juga dapat disebut suatu metode yang sering digunakan untuk menggambarkan ukuran suatu data. Mean dapat dihitung dengan menjumlahkan seluruh nilai data pengukuran dan dibagi dengan banyaknya data yang digunakan. Definisi tersebut di nyatakan dengan persamaan sebagai berikut:

\begin{theorem}[Mean]
rumus mencari nilai rata-rata:
\begin{align}
Sampel\\
& \bar{X}=\frac{x_{1}+x_{2}+x_{3}+......+x_{n}}{n}=\sum_{i=0}^{n}\frac{x_{i}}{n}\\
Populasi\\
& \bar{\mu}=\frac{x_{1}+x_{2}+x_{3}+......+x_{n}}{n}=\sum_{i=0}^{n}\frac{x_{i}}{n}
\end{align}
\end{theorem}

Keterangan

$\sum$ = lambang penjumlahan semua gugus data pengamatan 

n = banyaknya sampel data 

N = banyaknya data populasi 

$\bar x$ = nilai rata-rata sampel

$\mu$ = nilai rata-rata populasi

\subsection{Distribusi frekuensi}\index{Theorems!Several Equations}
Rata-rata yang dihitung berdasarkan data yang sudah ditata dalam bentuk tabel distribusi frekuensi dan dapat ditentukan dengan menggunakan formula / rumus rumus yang sama dengan formula untuk menghitung nilai rata-rata dari data yang sudah dikelompokkan atau data yang terdistribusi, dengan rumus sebagai berikut:

\begin{theorem}[Mean]
rumus mencari nilai distribusi frekuensi:
\begin{align}
& \bar{X} = \frac{\sum f_{i}x_{i}}{\sum f_{i}} 
\end{align}
\end{theorem}

Keterangan
 
$ \sum $ = lambang penjumlahan semua gugus data

$ f_{i} $ = frekuensi data ke-i

$ \bar x $= nilai rata-rata sampel

%------------------------------------------------

\section{Median}\index{Median}

Median merupakan nilai tengah dari sekumpulan data yang telah diurutkan dari angka terkecil sampai ke angka terbesar. Median ditentukan berdasarkan jumlah data, dengan jumlah data yang ganjil maka mediannya memiliki nilai tengah dari data yang telah diurutkan, dan dengan jumlah data genap maka mediannya adalah mean / rataan dari dua bilangan yang ditengah data yang sudah diurutkan

\begin{theorem}[Mean]
rumus mencari nilai rata-rata:
\begin{align}
untuk n ganjil\\
& Me =x_{\frac{1}{2}(n+1)} \\
Untuk n genap\\
& Me =\frac{x_{\frac{n}{2}}+x_{\frac{n}{2}+1}}{2}
\end{align}
\end{theorem}

Keterangan 

$x_{\frac{n}{2}}$ = data pada urutan ke-$\frac{n}{2}$ setelah diurutkan


\section{Modus}\index{Modus}

Modus adalah data yang sering muncul atau data yang memiliki jumlah frekuensi paling banyak. Sebuah data dapat dikatakan tidak memiliki modus ketika seluruh data yang muncul memiliki frekuensi yang sama atau dapat disebut sebuah data memiliki modus lebih dari satu.
Untuk data yang ditampilkan dalam bentuk tabel distribusi frekuensi berkelompok, dapat digunakan menentukan letak modus dengan cara melihat kelas interval yang mempunyai frekuensi paling besar.Bila data mempunyai satu modus dapat disebut unimodal dan data yang memiliki dua modus disebut bimodal, sedangkan jika data mempunyai modus yang lebih dari dua disebut multimodal. Modus dapat dilambangkan dengan Mo

\begin{theorem}[Mean]
rumus mencari nilai rata - rata:
\begin{align}
untuk n ganjil\\
& Mo = T_{b}+(\frac{s_{1}}{s_{1}+s_{2}})i
\end{align}
\end{theorem}

Keterangan 

$Mo$ = Modus

$T_{b}$ = Tepi bawah dari kelas modus

$s_{1}$ = Selisih frekuensi kelas modus dengan frekuensi kelas sebelum kelas modus

$s_{2}$ = Selisih frekuensi kelas modus dengan frekuensi kelas sesudah kelas modus

$i$ = panjang kelas interval

\subsection{Rata-Rata Hitung (Mean/Arhitmetic Mean)}\index{Theorems!Several Equations}

Rata-rata merupakan nilai yang mewakili kumpul data yaitu nilai yang kurang dari nilai itu, nilai yang lebih dari nilai itu dan nilai itu sendiri. 
Contoh:
-	Ani cantik
-	Rina tidak cantik		=      Kesimpulannya rata-rata perempuan itu cantik
-	Dini  sangat cantik          
Mean dari sekumpulan data adalah jumlah dari kumpulan bilangan dibagi banyak bilangan tersebut. 

Untuk data tunggal seperti: $x_{1}, x_{2}, x_{3}.....x_{n} $. Maka:

\begin{theorem}[Mean]
rumus mencari nilai rata-rata:
\begin{align}
& \overline{x} = \frac{\sum x_{i}}{n}
\end{align}
\end{theorem}

Keterangan:
	 
$\overline{x}$= Rataan Hitung
		 
$n$ = banyak data 

$x_{i}$ = data ke-i

Contoh = Tentukan rata-rata dari nilai siswa sebagai berikut: 70, 69, 45, 80 dan 56!

\begin{table}[h]
\centering
\begin{tabular}{l l}
\toprule
\textbf{$x_{i}$} & \textbf{$f_{i}$}\\
\midrule
70 & 5 \\
69 & 6 \\
45 & 3 \\
80 & 1 \\
56 & 1 \\
\bottomrule
\end{tabular}
\caption{Data Frekuensi Tunggal}
\end{table}

$x_{i}$ menyatakan nilai ujian dan $f_{i}$ menyatakan frekuensi untuk nilai $x_{i}$ yang bersesuaian. Untuk mencari rata-rata tabel diatas, akan lebih mudah bila dibuat tabel penolong seperti berikut:


$\overline{x} = \frac{\sum f_{i}\sum x_{i}}{\sum x_{i}}$


\begin{table}[h]
\centering
\begin{tabular}{l l l}
\toprule
\textbf{$x_{i}$} & \textbf{$f_{i}$} & \textbf{$f_{i}x_{i}$}\\
\midrule
70 & 5 & 350 \\
69 & 6 & 414\\
45 & 3 & 135\\
80 & 1 & 80\\
56 & 1 & 56\\
\bottomrule
\end{tabular}
\caption{Table caption}
\end{table}

Dari tabel, dapat kita lihat $\sum f_{i}x_{i}$  = 1035 dan $\sum f_{i}$  = 16. Sehingga:

$ \overline{x} = \frac{\sum f_{i}\sum x_{i}}{\sum x_{i}} = \frac{1035}{16} = 64,6 $

Rataan hitung nilai tersebut adalah 64,6.


Untuk data daftar distribusi frekuensi kelompok rumus yang digunakan sama dengan data daftar distribusi frekuensi tunggal yaitu $\overline{x} = \frac{\sum f_{i}\sum x_{i}}{\sum x_{i}}$ . Hanya saja, karena ada pengelompokan kelas maka $\sum x_{i}$ yang dirumus merupakan titik tengah dari kelas tersebut.$\frac{batas\;atas + batas\;bawah}{2}$

Contoh: tabel nilai ujian 80 Mahasiswa (I)

\begin{table}[h]
\centering
\begin{tabular}{l l l l}
\toprule
\textbf{Kelas} & \textbf{$x_{i}$} & \textbf{$f_{i}$} & \textbf{$f_{i}x_{i}$}\\
\midrule
31-40 & 1 & 35,5 & 35,5 \\
41-50 & 2 & 45,5 & 91 \\
51-60 & 5 & 55,5 & 277,5\\
61-70 & 15 & 65,6 & 982,5\\
71-80 & 25 & 75,5 & 1887,5\\
81-90 & 20 & 85,5 & 1710\\
91-100 & 18 & 95,5 & 1146\\
\bottomrule
\end{tabular}
\caption{Nilai Ujian}
\end{table}

Dari tabel, dapat kita lihat $\sum f_{i}x_{i}$  = 6130 dan $\sum f_{i}$  = 80. Sehingga:


$ \overline{x} = \frac{\sum f_{i}\sum x_{i}}{\sum x_{i}} = \frac{6130}{80} = 76,62 $

Rataan hitung nilai ujiannya adalah 76,62.

Untuk mencari rataan hitung data distribusi frekuensi kelompok dapat digunakan cara lainnya yaitu cara sandi atau cara singkat. Untuk memakai cara ini maka gunakan langkah-langkah berikut
	Ambil salah satu titik tengah kelas, namakan x0.
	Untuk titik tengah x0 diberi nilai sandi c = 0
	Titik tengah yang nilainya kurang dari x0 berturut-turut diberi harga-harga sandi c = -1, c = -2, c = -3, dan seterusnya.
	Titik tengah yang nilainya lebih dari x0 berturut-turut diberi harga-harga sandi c = +1, c = +2, c = +3, dan seterusnya.
	p merupakan panjang kelas dimana setiap kelas memiliki panjang kelas yang sama.
Gunakan rumus

\chapter{Peluang kejadian majemuk}
\section{Kejadian Saliing bebas}

Peluang Kejadian A dinotarsikan dengan P(A) adalah perbandingan banyaknya hasil kejadian A dinotasikan n(A) terhadap banyak semmua hasil yang mungkin dinotasikan dengan n(S) dalam satu percobaan. Kisaran nilai peluang kejadian A adalah $0 \leq$ P(A) $0 \leq$ 1. jika P(A) = 0 disebut kemustahilan dan P(A) = 1 disebut kepastian.

1. kejadian saling bebas(Stokastik)

Dua kejadian dikatakan saling bebas (independen) jika terjadinya kejadian yang satu tidak mempengaruhi kemungkinan terjadinya kejadian yang lain. Bila kejadian A tidak mempengaruhi terjadinya B dan sebaliknya, maka kejadian semacam ini disebut dua kejadian saling bebas
Contoh
Ketika melempar koin dua kali, hasil dari lemparan pertama tidak mempengaruhi hasil dari lemparan kedua.
Ketika mengambil dua kartu dari satu set kartu permainan (52 kartu), kejadian 'mendapatkan raja (K)' pada kartu pertama dan kejadian 'mendapatkan kartu hitam' pada kartu kedua adalah tidak saling bebas. Peluang pada kartu kedua berubah setelah kartu yang pertama diambil. Kedua kejadian di atas akan menjadi saling bebas jika setelah mengambil
kartu yang pertama, kartu tersebut dikembalikan ke set semula (sehingga set kartu itu lengkap kembali, 52 kartu).
Jika dua keeping mata uang yang homogen dilantunkan bersama-sama, maka kejadian yang mungkin adalah : S = {(G1,G2), (G1,A2), (A1,G2), (A1,A2)} ® n(s) = 4

\section{Penjelasan Kejadian Saling bebas}\index{Penjelasan Kejadian Saling Bebas}
\vspace{0.5cm} 
Jekadian pada percobaan ada dua yaitu;
kejadian sedehana dan kejadian dan jekadian majemuk
peluang kejadian saling lepas dan saling bebas.

Dengan menggunakan sifat-sifat gabungan dua berdasarkan teori gabungan banyak nya anggota himpunan $A\bigcup B$ yang disimbol kan.



$A\bigcap B$ = n(A) + n(B) ($A\bigcup B$) dengan  ($A\bigcup B$) yang menyatakan irisan himpunan A dan B

Menentukan Peluang gabungan dua kejadian : P($A\bigcap B$).

n($A\bigcup B$)= n(A)+(B) -n($A\bigcap B$) [bagi dg n(S)]. 

$\frac{n}{n}$ $A\bigcup B$ = $\frac{n}{n}$ $\frac{A}{S}$ + $\frac{n}{n}$ $\frac{B}{S}$ n($A\bigcap B)$ $\frac{n}S$

P($A\bigcup B$) = P(A)+P(B)-P($A\bigcap B$)

Jadi rumus peluang gabunganya adalah;
P($A\bigcup B$) = P(A)+P(B)-P($A\bigcap B$)
Keterangan :

P($A\bigcup B$) = Peluang Gabungan kejadian A dan B

P(A) = Peluang kejadian A

P(B) = Peluang kejadian B

P$A\bigcap B$) Peluang irisan kejadian A dan B


%------------------------------------------------

\section{Contoh Soal}\index{Peluang}

1. Sebuah dadu isi enam di lempar sekali, berapakah peluang kejadian muncul nya angka genap atau angka prima ? 

Penyelesaian :

a.) Ruang sample nya adalah S={1,2,3,4,5,6} dengan n(S)=6
misalnya A kejadian muncul mata dadu genap dan B kejadian mata dadu prima,

A={2,4,5} B={2,3,5} dan $A\bigcap B$ ={2}
$A\bigcap B$
sehingga n(A)=3,n(B)=3, n($A\bigcap B$) = 1

Ketika melempar koin dua kali, hasil dari lemparan pertama tidak mempengaruhi hasil dari lemparan kedua.
\vspace{1cm} 
Ketika mengambil dua kartu dari satu set kartu permainan (52 kartu), kejadian 'mendapatkan raja (K)' pada kartu pertama dan kejadian 'mendapatkan kartu hitam' pada kartu kedua adalah tidak saling bebas. Peluang pada kartu kedua berubah setelah kartu yang pertama diambil. Kedua kejadian di atas akan menjadi saling bebas jika setelah mengambil kartu yang pertama, kartu tersebut dikembalikan ke set semula (sehingga set kartu itu lengkap kembali, 52 kartu).
\vspace{1cm} 

Untuk dua kejadian saling bebas, A dan B, peluang untuk keduanya terjadi, P(A dan B), adalah hasil perkalian antara peluang dari masing-masing kejadian.
\vspace{1cm} 

P( A dan B ) =  P($A\bigcap B$) = P(A) X P(B)
Misalnya, ketika melempar koin dua kali, peluang mendapat 'kepala' (K) pada lemparan pertama lalu mendapat 'ekor' (E) pada lemparan kedua adalah
\vspace{1cm} 

P(K dan E) = P(X) X P(B)
\vspace{0.5in}

P(K dan E) = 0.5 X 0.5
\vspace{0.5in}
P(K dan E) = 0.25

Dalam sebuah kantong terdapat 15 alat tulis yang terdiri dari 7 Pensil dan 8
Bolpen. Jika kita disuruh mengambil 2 alat tulis dengan mata tertutup.
Tentukan terambil kedua-duanya Pensil ?
Jawab :

Jika A = Pensil Pengambilan Pertama :

Maka P(A) = n(A)/n(S) = 7/15

Jika B = Pensil Pengambilan Kedua :

Maka P(B) = n(B)/n(S) = 6/15

Jadi P(A n B) = P(A) x P(B)

= 4/10 x 3/9
= 12/90
= 2/15




%----------------------------------------------------------------------------------------
%	BIBLIOGRAPHY
%----------------------------------------------------------------------------------------

\chapter*{Bibliography}
\addcontentsline{toc}{chapter}{\textcolor{ocre}{Bibliography}}
\section*{Books}
\addcontentsline{toc}{section}{Books}
\printbibliography[heading=bibempty,type=book]
\section*{Articles}
\addcontentsline{toc}{section}{Articles}
\printbibliography[heading=bibempty,type=article]

%----------------------------------------------------------------------------------------
%	INDEX
%----------------------------------------------------------------------------------------

\cleardoublepage
\phantomsection
\setlength{\columnsep}{0.75cm}
\addcontentsline{toc}{chapter}{\textcolor{ocre}{Index}}
\printindex

%----------------------------------------------------------------------------------------

\end{document}
%%%%%%%%%%%%%%%%%%%%%%%%%%%%%%%%%%%%%%%%%
% The Legrand Orange Book
% LaTeX Template
% Version 2.2 (30/3/17)
%
% This template has been downloaded from:
% http://www.LaTeXTemplates.com
%
% Original author:
% Mathias Legrand (legrand.mathias@gmail.com) with modifications by:
% Vel (vel@latextemplates.com)
%
% License:
% CC BY-NC-SA 3.0 (http://creativecommons.org/licenses/by-nc-sa/3.0/)
%
% Compiling this template:
% This template uses biber for its bibliography and makeindex for its index.
% When you first open the template, compile it from the command line with the 
% commands below to make sure your LaTeX distribution is configured correctly:
%
% 1) pdflatex main
% 2) makeindex main.idx -s StyleInd.ist
% 3) biber main
% 4) pdflatex main x 2
%
% After this, when you wish to update the bibliography/index use the appropriate
% command above and make sure to compile with pdflatex several times 
% afterwards to propagate your changes to the document.
%
% This template also uses a number of packages which may need to be
% updated to the newest versions for the template to compile. It is strongly
% recommended you update your LaTeX distribution if you have any
% compilation errors.
%
% Important note:
% Chapter heading images should have a 2:1 width:height ratio,
% e.g. 920px width and 460px height.
%
%%%%%%%%%%%%%%%%%%%%%%%%%%%%%%%%%%%%%%%%%

%----------------------------------------------------------------------------------------
%	PACKAGES AND OTHER DOCUMENT CONFIGURATIONS
%----------------------------------------------------------------------------------------

\documentclass[11pt,fleqn]{book} % Default font size and left-justified equations

%----------------------------------------------------------------------------------------

\input{structure} % Insert the commands.tex file which contains the majority of the structure behind the template

\begin{document}

%----------------------------------------------------------------------------------------
%	TITLE PAGE
%----------------------------------------------------------------------------------------

\begingroup
\thispagestyle{empty}
\begin{tikzpicture}[remember picture,overlay]
\node[inner sep=0pt] (background) at (current page.center) {\includegraphics[width=\paperwidth]{background}};
\draw (current page.center) node [fill=ocre!30!white,fill opacity=0.6,text opacity=1,inner sep=1cm]{\Huge\centering\bfseries\sffamily\parbox[c][][t]{\paperwidth}{\centering The Search for a Title\\[15pt] % Book title
{\Large A Profound Subtitle}\\[20pt] % Subtitle
{\huge Dr. John Smith}}}; % Author name
\end{tikzpicture}
\vfill
\endgroup

%----------------------------------------------------------------------------------------
%	COPYRIGHT PAGE
%----------------------------------------------------------------------------------------

\newpage
~\vfill
\thispagestyle{empty}

\noindent Copyright \copyright\ 2013 John Smith\\ % Copyright notice

\noindent \textsc{Published by Publisher}\\ % Publisher

\noindent \textsc{book-website.com}\\ % URL

\noindent Licensed under the Creative Commons Attribution-NonCommercial 3.0 Unported License (the ``License''). You may not use this file except in compliance with the License. You may obtain a copy of the License at \url{http://creativecommons.org/licenses/by-nc/3.0}. Unless required by applicable law or agreed to in writing, software distributed under the License is distributed on an \textsc{``as is'' basis, without warranties or conditions of any kind}, either express or implied. See the License for the specific language governing permissions and limitations under the License.\\ % License information

\noindent \textit{First printing, March 2013} % Printing/edition date

%----------------------------------------------------------------------------------------
%	TABLE OF CONTENTS
%----------------------------------------------------------------------------------------

%\usechapterimagefalse % If you don't want to include a chapter image, use this to toggle images off - it can be enabled later with \usechapterimagetrue

\chapterimage{chapter_head_1.pdf} % Table of contents heading image

\pagestyle{empty} % No headers

\tableofcontents % Print the table of contents itself

\cleardoublepage % Forces the first chapter to start on an odd page so it's on the right

\pagestyle{fancy} % Print headers again

%----------------------------------------------------------------------------------------
%	PART
%----------------------------------------------------------------------------------------

\part{Part One}

%----------------------------------------------------------------------------------------
%	CHAPTER 1
%----------------------------------------------------------------------------------------

\chapterimage{chapter_head_2.pdf} % Chapter heading image

\chapter{Geometri Bidang Datar}
\section{Kesebangunan antar bangun datar}\index{Paragraphs of Text}

Kesebangunan dan kekongruenan biasanya digunakan untuk membandingkan dua buah bangun datar (atau lebih) dengan bentuk yang sama. dua buah bangun datar dapat dikatakan sebangun apabila panjang setiap sisi pada kedua bangun datar tersebut memiliki nilai perbandingan yang sama. sedangkan kongruen memiliki konsep yang lebih mendetail, apabila dua buah (atau lebih) bangun datar memiliki bentuk, ukuran, serta besar sudut yang sama barulah mereka dapat disebut sebagai bangun datar yang kongruen.Perhatikan gambar berikut:

\includegraphics[width=3cm,height=3cm]{Kesebangunan.jpg}


Kesebangunan Pada Persegi Panjang

Perhatikan gambar dua buah persegi panjang di bawah ini.keduanya merupakan bangun datar yang sebangun karena memiliki kesamaan sifat yang dapat dijelaskan sebagai berikut:

\includegraphics[width=3cm,height=3cm]{persegi.jpg}


\textbf{1.Perbandingan antara sisi terpanjang dengan sisi terpendek memiliki nilai yang sama.}

Perbandingan sisi terpanjang PQ dengan sisi terpendek QR  = 39 : 13  = 1 : 3
Perbandingan sisi terpanjang KL dengan sisi terpendek LM   = 24 : 8    = 1 : 3
Perbandingan sisi terpanjang RS dengan sisi terpendek QP   = 39 : 13  = 1 : 3
Perbandingan sisi terpanjang MN dengan sisi terpendek NK = 24 : 8    = 1 : 3

Dari perhitungan diatas dapat dilihat bahwa sisi terpanjang dan terpendek pada kedua persegi panjang diatas  memiliki perbandingan yang sama yaitu 1 : 3.


\textbf{2.Besar sudut pada kedua persegi panjang tersebut memiliki nilai yang sama besar.}

Sudut P = Sudut K; Sudut Q = Sudut L; Sudut R = Sudut M; Sudut S = Sudut N

Karena kedua persegi panjang tersebut hanya memiliki bentuk dan sudut yang sama besar namun tidak memiliki ukuran yang sama, maka dua bangun datar tersebut tidak bisa disebut kongruen.

\textbf{Contoh Soal Kesebangunan pada Persegi Panjang}

Ada dua buah persegi panjang dengan ukuran yang berbeda ABCD dan KLMN. Persegi panjang ABCD memiliki panjang 16cm dan lebar 4cm. Bila persegi panjang ABCD sebangun dengan persegi panjang KLMN yang memiliki panjang 32cm, maka berapakah lebar dari persegi panjang KLMN?

Karena kedua persegi panjang tersebut sebangun, maka berlaku rumus:

AB/KL = BC/LM
16/32 = 4/LM
   LM = 32x4/16
   LM = 124/16
   LM = 8 cm

Maka lebar dari persegi panjang KLMN adalah 8 cm.


Kesebangunan pada Segitiga
Kesebangunan pada segitiga agak lebih sulit dicapai karena ada tiga buah sisi yang harus sama perbandingannya. 

Contoh segitiga yang sebangun:


\includegraphics[width=3cm,height=3cm]{segitiga.jpg}


Segitiga tersebut dapat dikatakan sebangun karena perbandingan sisi-sisinya sama besar:

Sisi AC sesuai dengan sisi PR = AC/PR = 4/2 = 2/1
Sisi AB sesuai dengan sisi PQ = AB/PQ = 8/4 = 2/1
Sisi BC sesuai dengan sisi QR = BC/QR = 6/3 = 2/1

Maka AC/PR = AB/PQ = BC/QR = 2/1


Besar sudut yang bersesuaian memiliki besar yang sama:

Sudut A = sudut P; sudut B = sudut Q; sudut C = sudut R

\textbf{Contoh Soal Kesebangunan pada Persegi Panjang}


\includegraphics[width=3cm,height=3cm]{soal.jpg}


Diketahui segitiga ABC sebangun dengan segitiga KLM, maka berapakah panjang LM dan MK?

Jawab:

AB/KL = BC/LM
18/6  = 15/LM
   3  = 15/LM
   LM = 15/3
   LM = 5 cm

Dari hasil tersebut kita dapat mengetahui bahwa perbandingan sisi pada kedua segitiga tersebut adalah:

18 : 6 = 3 : 1
15 : 5 = 3 : 1
12 : MK = 3 : 1
MK = 12/3
MK = 4 cm
\\

\textbf{Contoh Kesebangunan pada Trapesium}

Perhatikan gambar di bawah ini!

\includegraphics[width=3cm,height=3cm]{soal1.jpg}

Buktikan bahwa,

Soal 6 Rumus

Jika DC = 20 cm, AB = 34 cm, DE = 9 cm dan AE = 15 cm, tentukan EF!

Pembahasan Untuk membuktikan rumus yang ditentukan, kita harus menggambar garis DH yang sejajar dengan garis BC, seperti berikut.
Karena garis EG sejajar dengan garis AH, maka segitiga DEG sebangun dengan segitiga DAH. Akibatnya,

\includegraphics[width=3cm,height=3cm]{rumus1.jpg}

Untuk DC = 20 cm, AB = 34 cm, DE = 9 cm dan AE = 15 cm, maka

\includegraphics[width=3cm,height=3cm]{rumus2.jpg}

Jadi, diperoleh panjang EF adalah 25,25 cm.

\section{Kekongruenan Antar Bangun Datar}\index{Paragraphs of Text}

Definisi kekongruenan tidak lepas dari kesebangunan karena kekongruenan
merupakan kasus khusus kesebangunan. Jadi definisinya sebagai berikut.
Dua segibanyak (polygon) dikatakan kongruen jika ada korespondensi satu-satu
antara titik-titik sudut kedua segi banyak tersebut sedemikian hingga berlaku: 

1. sudut-sudut yang bersesuaian sama besar, dan

2. semua perbandingan panjang sisi-sisi yang bersesuaian adalah satu.

Syarat kedua ini dapat diringkas menjadi 2`. sisi-sisi yang bersesuaian sama panjang. 

%------------------------------------------------

\section{Contoh}\index{Contoh}
\includegraphics[width = 8cm, height= 5cm]{Pictures/1.png}
 
Pada gambar di atas telah dibuat korespondensi satu-satu antar titik-titik sudut pada kedua bangun sehingga sudut-sudut yang bersesuaian sama besar dan sisi-sisi yang bersesuaian sama panjang Berarti (sesuai definisi) dapat disimpulkan segiempat
ABCD kongruen dengan segiempat EFGH atau ditulis segiempat ABCD $latex\cong $ EFGH.

Sekali lagi, perhatikan bahwa korespondensi yang menjadikan dua bangun datar kongruen tidak terpengaruh oleh posisi kedua bangun. Jadi sekali telah ditemukan korespondensi satu-satu antar kedua bangun maka posisi apapun tetap kongruen. 

\includegraphics[width = 8cm, height= 5cm]{Pictures/2.png}

Perhatikan gambar di atas. Kedua bangun pada posisi I, II, III, mupun IV tetap
kongruen walaupun posisi kedua bangun tersebut berubah-ubah. Jika dicermati lebih
lanjut, keempat posisi itu mewakili proses translasi, refleksi, rotasi, dan kombinasi
dari ketiganya. Secara bahasa sederhana, dua bangun dikatakan kongruen jika kedua
bangun tersebut sama dalam hal bentuk dan ukurannya. 

\paragraph{}


Selanjutnya perhatikan segiempat dan segilima berikut. 

\includegraphics[width = 8cm, height= 5cm]{Pictures/3.png}

Berdasar gambar di atas, segiempat dapat disusun dari dua segitiga dan segilima
dapat disusun dari tiga segitiga. Secara umum segi-n dapat disusun dari n – 2 segitiga.
Hal tersebut merupakan gambaran bahwa setiap segibanyak dapat disusun dari segitiga-segitiga. Oleh karena itu sifat-sifat kesebangunan dan kekongruenan pada
segitiga perlu untuk dibicarakan secara khusus. 

\section{Teorema}\index{Teorema}

Secara sederhana sesuai dengan pengertian kekongruenan, dua segitiga dikatakan
kongruen jika sudut-sudut yang bersesuaian sama besar dan sisi-sisi yang bersesuaian
sama panjang. Ada satu postulat dan tiga teorema yang terkait dengan kekongruenan
segitiga. Kita ingat bahwa postulat tidak dibuktikan sedangkan teorema perlu
dibuktikan. Tetapi pada modul ini kita tidak membahas bukti teorema karena telah
dibahas pada modul BERMUTU tahun sebelumnya. 

\subsection{Postulat kekongruenan s.sd.s (sisi-sudut-sisi}\index{Teorema!Postulat kekongruenan s.sd.s (sisi-sudut-sisi)}


\begin{theorem}[Postulat kekongruenan s.sd.s (sisi-sudut-sisi)]

Diberikan dua segitiga $\vartriangle $ABC dan $vartriangle $DEF dimana m$\angle$A = m$\angle$D, AB = DF maka $\vartriangle $ABC $\cong$ $\vartriangle $DEF
\end{theorem}
\includegraphics[width = 8cm, height= 4cm]{Pictures/4.png}
\subsection{Teorema kekongruenan sd.s.sd (sudut-sisi-sudut)}\index{Theorems!Teorema kekongruenan sd.s.sd (sudut-sisi-sudut)}
\begin{theorem}
Diberikan dua segitiga $\vartriangle $ABC dan $vartriangle $DEF dimana m$\angle$A = m$\angle$D, AC = DF, m$\angle$A = m$\angle$D maka $\vartriangle $ABC $\cong$ $\vartriangle $DEF
\end{theorem}
\includegraphics[width = 8cm, height= 4cm]{Pictures/5.png}

%------------------------------------------------

\subsection{Teorema Teorema kekongruenan s.s.s (sisi-sisi-sisi)}\index{Theorems!Teorema kekongruenan s.s.s (sisi-sisi-sisi)}
\begin{theorem}
Diberikan dua segitiga $\vartriangle $ABC dan $vartriangle $DEF dimana, AB = DE,  m$\angle$A = m$\angle$D,dan  m$\angle$C = m$\angle$F , BC = EF  maka $\vartriangle $ABC $\cong$ $\vartriangle $DEF
\end{theorem}
\includegraphics[width = 8cm, height= 4cm]{Pictures/6.png}

\subsection{Teorema kekongruenan s.sd.sd (sisi-sudut-sudut)}\index{Theorems!Teorema kekongruenan s.sd.sd (sisi-sudut-sudut)}
\begin{theorem}
Diberikan dua segitiga $\vartriangle $ABC dan $vartriangle $DEF dimana, AB = DE, AC = DF,dan , BC = EF  maka $\vartriangle $ABC $\cong$ $\vartriangle $DEF
\end{theorem}
\includegraphics[width = 8cm, height= 4cm]{Pictures/7.png}

\section{Kekongruenan Segitiga}\index{Kekongruenan Segitiga}

Pada bagian ini, pembahasan bangun-bangun yang kongruen difokuskan pada bangun segitiga. Untuk menunjukkan apakah dua segitiga kongruen atau tidak, cukup ukur setiap sisi dan sudut pada segitiga. Kemudian,bandingkan sisi-sisi dan sudut-sudut yang bersesuaian. Perhatikan tabel syarat kekongruenan dua segitiga berikut.


\includegraphics[width = 13cm, height= 12cm]{Pictures/a21.png}

\subsection{Sifat-Sifat Dua Segitiga yang Sebangun dan Kongruen}
\includegraphics[width = 13cm, height= 8cm]{Pictures/a25.png}

Setelah kita memahami pengertian kesebangunan dan kekongruenan secara umum,sekarang kita akan mendalami sifat-sifat kesebangunan dan kekongruenan, khusus mengenai segitiga. Namun sebelumnya perlu diingat bahwa dua bangun yang kongruen pasti sebangun sementara dua bangun yang sebangun belum tentu kongruen. Oleh karena itu dalam pembahasan ini akan dimulai dari sifat kekongruenan.

Secara sederhana sesuai dengan pengertian kekongruenan, dua segitiga dikatakan kongruen jika sudut-sudut yang bersesuaian sama besar dan sisi-sisi yang bersesuaian sama panjang. Ada satu postulat dan tiga teorema yang  terkait dengan kekongruenan segitiga. Kita ingat bahwa postulat tidak dibuktikan sedangkan teorema perlu dibuktikan. Tetapi pada modul ini kita tidak membahas bukti teorema karena telah dibahas pada modul BERMUTU tahun sebelumnya. 

Contoh : 

\includegraphics[width = 13cm, height= 10cm]{Pictures/a26.png}

\subsection{Contoh Soal 1}
\includegraphics[width = 13cm, height= 8cm]{Pictures/a22.png}

\includegraphics[width = 13cm, height= 8cm]{Pictures/a23.png}

\subsection{Contoh Soal 2}
\includegraphics[width = 13cm, height= 8cm]{Pictures/a24.png}
%------------------------------------------------



%----------------------------------------------------------------------------------------
%	CHAPTER 2
%----------------------------------------------------------------------------------------


\chapterimage{chapter_head_2.pdf} % Chapter heading image

\chapter{Geometri Ruang}
\section{Jarak antar Titik}

\section{Jarak Titik Ke Garis}
\subsection{Menemukan Konsep Jarak, Titik dan Garis}
\begin{enumerate}
\item Kedudukan Titik


\includegraphics{jembatan.jpg}


Jika dimisalkan jembatan penyeberangan merupakan suatu garis dan lokomotif kereta adalah suatu titik. Kita dapat melihat bahwa lokomotif tidak terletak atau melalui jembatan penyeberangan. Artinya jika dihubungkan dengan garis dan titik maka dapat disebut bahwa contoh di atas merupakan suatu titik yang tidak terletak pada garis.

\includegraphics{bola.jpg}

Gambar di atas merupakan contoh kedudukan titik terhadap bidang, dengan bola sebagai titik dan lapangan sebagai bidang. Sebuah titik dikatakan terletak pada sebuah bidang jika titik itu dapat dilalui bidang seperti terlihat pada titik A pada gambar dan sebuah titik dikatakan terletak di luar bidang jika titik itu tidak dapat dilalui bidang.

DEFINISI
\begin{enumerate}
\item 1)	Jika suatu titik dilalui garis, maka dikatakan titik terletak pada garis tersebut.
\item 2)	Jika suatu titik tidak dilalui garis, maka dikatakan titik tersebut berada di luar garis.
\item 3)	Jika suatu titik dilewati suatu bidang, maka dikatakan titik itu terletak pada bidang.
Jika titik tidak dilewati suatu bidang, maka titik itu berada di luar bidang.
\end{enumerate}

\item Jarak titik ke garis


Jarak merupakan salah satu permasalahan matematika yang sering dijumpai di sekitar kita. Jarak dapat diukur di antara dua objek, seperti rumah dengan kantor pos, rumah sakit dengan jalan raya, dan jalan raya dengan jalan raya lainnya. Pada pembahasan ini hanya akan dibahas mengenai jarak antara dua objek yang berupa titik dan garis lurus. 

Jarak titik ke garis adalah jarak terdekat sebuah titik ke garis, jarak terdekat diperoleh dengan menarik garis yang tegak lurus dengan garis yang dimaksud. Jarak titik B dengan garis g adalah panjang garis BB'.

\begin{center}
\includegraphics{panjanggaris.jpg}
\end{center}

Perhatikan contoh permasalahan berikut:

Vihara Dharma Agung terletak pada koordinat (71, 76) dan Jalan Sungai Kelara berupa garis lurus dengan persamaan $$5x-8y-280=0$$ (satuan dalam meter). Bagaimana cara mengukur jarak antara vihara dengan jalan tersebut? Salah satunya adalah dengan menggunakan rumus jarak antara titik dengan garis lurus.

\begin{enumerate}
\item Menemukan Rumus Jarak Titik dengan Garis
\end{enumerate}
\end{enumerate}

Misalkan akan ditentukan jarak antara titik A(a, b) dengan garis lurus yang memiliki persamaan $$px+qy+r = 0.$$ Perhatikan gambar berikut.

\begin{center}
\includegraphics{jarak.jpg}
\end{center}

Perlu diingat bahwa jarak dua objek adalah panjang lintasan terpendek yang menghubungkan kedua objek tersebut. Karena ruas garis yang tegak lurus dengan garis $$px+qy+r = 0$$ dan memiliki ujung di titik A dan ujung satunya di garis tersebut merupakan lintasan terpendek yang menghubungkan titik dan garis tersebut, maka panjang dari ruas garis tersebut, yaitu d, adalah jarak titik A terhadap garis $$px+qy+r = 0$$.

Jarak titik ke garis adalah jarak terdekat sebuah titik ke garis, jarak terdekat diperoleh dengan menarik garis yang tegak lurus dengan garis yang dimaksud.
Jarak titik B dengan garis g adalah panjang garis BB’  

\includegraphics[width = 8cm, height= 5cm]{Pictures/gi1.png}

Contoh :
1. Kubus ABCDEFGH memiliki panjang rusuk 8 cm, titik P merupakan perpotongan
diagonal bidang atas, hitunglah jarak titik P dengan garis AD

Penyelesaian


\includegraphics[width = 8cm, height= 5cm]{Pictures/gi2.png}
\includegraphics[width = 8cm, height= 5cm]{Pictures/gi3.png}

2. Sebuah kubus ABCD.EFGH dengan panjang rusuk 6 cm. tentukan jarak titik A ke
garis CE adalah…

\includegraphics[width = 8cm, height= 5cm]{Pictures/gi4.png}

\includegraphics[width = 8cm, height= 5cm]{Pictures/gi5.png}

Contoh Soal

Diketahui kubus ABCD.EFGH. Tentukan projeksi titik
A pada garis

a. CD!

b. BD!

Jika dari titik A ditarik garis yang tegak lurus terhadap
segmen garis CD maka diperoleh titik D sebagai hasil
proyeksinya (AD CD). 

\includegraphics[width = 8cm, height= 5cm]{Pictures/gi6.png}

b. Proyeksi titik A pada garis BD
Jika dari titik A ditarik garis yang tegak lurus
terhadap segmen garis BD maka diperoleh titik T
sebagai hasil proyeksinya (AT  BD).

\includegraphics[width = 8cm, height= 5cm]{Pictures/gi7.png}

Contoh Soal
Kubus ABCD.EFGH memiliki rusuk 8 cm. Jarak titik D ke garis HB adalah …

\includegraphics[width = 8cm, height= 5cm]{Pictures/gi8.png}

Pandanglah segitiga BDH yang terdapat dalam kubus. Segitiga BDH adalah segitiga siku-siku di D.

DH adalah salah satu rusuk kubus.

DH = 8 cm

BD adalah diagonal bidang atau diagonal sisi.


\includegraphics[width = 4cm, height= 4cm]{Pictures/gi9.png}


Contoh Soal 2
Perhatikan gambar kubus PQRS.TUVW di bawah ini.

\includegraphics[width = 4cm, height= 4cm]{Pictures/gi10.png}

Jika panjang rusuk kubus di atas adalah 8 cm dan titik X merupakan pertengahan antara rusuk PQ. Maka hitung jarak:
a) titik X ke garis ST
b) titik X ke garis RT

Penyelesaian:
Perhatikan gambar di bawah ini

\includegraphics[width = 4cm, height= 4cm]{Pictures/gi11.png}
a) titik X ke garis ST merupakan panjang garis dari titik X ke titik M (garis MX) yang tegak lurus dengan garis ST, seperti gambar berikut.

\includegraphics[width = 4cm, height= 4cm]{Pictures/gi12.png}

\includegraphics[width = 4cm, height= 4cm]{Pictures/gi14.png}

b) titik X ke garis RT merupakan panjang garis dari titik X ke titik N (garis NX) yang tegak lurus dengan garis RT, seperti gambar berikut.

\includegraphics[width = 4cm, height= 4cm]{Pictures/gi15.png}


Contoh soal 3
Diketahui panjang rusuk sebuah kubus ABCD.EFGH adalah 6cm. Maka hitunglah jarak:

a).titik D ke garis BF
b).titik B ke garis EG

Penyelesaiannya:

a).Agar lebih mudah dalam menjawabnya, mari kita perhatikan gambar di bawah ini:

\includegraphics[width = 4cm, height= 4cm]{Pictures/gi16.png}

Dari gambar di atas kita bisa melihat bahwa jarak titik D ke garis BF adalah panjang diagonal BD yang dapat ditentukan dengan menggunakan teorema phytagoras ataupun dengan rumus. Mari kita selesaikan dengan teorema phytagoras terlebih dahulu:

\includegraphics[width = 4cm, height= 4cm]{Pictures/gi17.png}

berikut bila kita mencarinya dengan menggunakan rumus:

\includegraphics[width = 4cm, height= 4cm]{Pictures/gi18.png}

b). Sama halnya dengan soal a) kita juga harus membuat gambarnya terlebih dahulu agar lebih mudah mengerjakannya

\includegraphics[width = 4cm, height= 4cm]{Pictures/gi19.png}

\includegraphics[width = 4cm, height= 4cm]{Pictures/gi20.png}

\includegraphics[width = 4cm, height= 4cm]{Pictures/20.png}


This is an example of a definition. A definition could be mathematical or it could define a concept.

\begin{definition}[Definition name]
Given a vector space $E$, a norm on $E$ is an application, denoted $||\cdot||$, $E$ in $\mathbb{R}^+=[0,+\infty[$ such that:
\begin{align}
& ||\mathbf{x}||=0\ \Rightarrow\ \mathbf{x}=\mathbf{0}\\
& ||\lambda \mathbf{x}||=|\lambda|\cdot ||\mathbf{x}||\\
& ||\mathbf{x}+\mathbf{y}||\leq ||\mathbf{x}||+||\mathbf{y}||
\end{align}
\end{definition}

\chapter{Kaidah Pencacahan}

Pengertian Kaidah Pencacahan (Caunting Slots)


Kaidah pencacahan atau Caunting Slots adalah suatu kaidah yang digunakan untuk menentukan atau menghitung berapa banyak cara yang terjadi dari suatu peristiwa. Kaidah pencacahan terdiri atas :

\section{Aturan Penjumlahan}
1.Aturan penjumlahan

 Jika ada A dan B yang merupakan himpunan saling lepas dengan banyak anggota himpunannya adalah x dan y, maka banyaknya cara mengambil satu anggota dari gabungan keduanya akan sama dengan x+y, dinotasikan:
 
 \includegraphics[width = 9cm, height= 2cm]{Pictures/notasijumlah.JPG}
 
 
 Gambar. Notasi aturan penjumlahan
 
 Atau secara sederhana digunakan saat ada sejumlah kejadian yang tidak saling berhubungan (saling lepas). Dalam kondisi ini kejadian-kejadian tersebut dijumlahkan untuk mendapatkan total kejadian yang mungkin terjadi.

Contoh 1:

Dari kota A ke kota B ada beberapa jenis angkutan yang dapat digunakan. Ada 4 travel, 2 kapal laut, dan 1 pesawat terbang yang dapat dipilih. Ada berapa total cara berbeda untuk berangkat dari kota A menuju kota B?

Pembahasan:


\includegraphics[width = 6cm, height= 4cm]{Pictures/contoh1.JPG}

	Dalam soal di atas ketika kita memilih travel, kapal laut, maupun pesawat terbang tidak berpengaruh satu sama lain, ketiganya merupakan himpunan yang saling lepas. Sehingga ada 4+2+1 = 7 cara berbeda untuk berangkat dari kota A menuju kota B. []

Definisi 1
Jika suatu kejadian dapat dikerjakan dengan beberapa cara, tetapi cara-cara ini tidak dapat dikerjakan pada waktu yang sama.
Jika kejadian tersebut dapat terjadi dengan n1n1 cara, atau
kejadian tersebut terjadi dengan n2n2 cara, atau
kejadian tersebut dapat terjadi dengan n3n3 cara, atau
...................
kejadian tersebut dapat terjadi dengan npnp cara,
maka kejadian dengan ciri yang demikian dapat terjadi dengan 
(n1+n2+n3+...+np)(n1+n2+n3+...+np) cara.

Contoh 1

Dalam sebuah pantia, wakil dari sebuah jurusan dapat dipilih dari dosen, atau mahasiswa. Jika pada jurusan tersebut memiliki 37 dosen dan 83 mahasiswa, Berapa banyak cara memilih wakil dari jurusan tersebut?

Jawab:

Ada 37 cara untuk memilih wakil dari sebuah jurusan yang berasal dari kalangan dosen dan ada 83 cara memilih wakil dari sebuah jurusan yang berasal dari kalangan mahasiswa. Karena pada jurusan tersebut tidak ada dosen yang berstatus mahasiswa ataupun mahasiswa yang berstatus dosen, maka berdasarkan aturan penjumlahan, ada37+83=120 cara untuk memilih wakil dari sebuah jurusan. 

Contoh 2

Seorang pelajar dapat memilih sebuah proyek komputer dari salah satu diantara tiga aftar yang tersedia. ketiga daftar tersebut terdiri atas 23, 15, dan 19 kemungkinan proyek. Proyek - proyek komputer yang ada pada ketiga daftar tersebut semuanya berbeda. Berapa banyakkemungkinan siswa tersebut memilih proyek komputer?

Jawab: 

karena dari ketiga daftar tersebut semua proyek berbeda, dimana pada daftar pertama ada 23 proyek, daftar kedua ada 15 proyek , dan daftar ketiga ada 19 proyek Maka  berdasarkan aturan penjumlahan ada 23+15+19 = 57 kemungkinan siswa tersebut memilih proyekkomputer. []

Contoh 3

Misalkan Andi akan berangkat sekolah bersama dengan teman sekelasnya, Amir. Rumah Andi terletak pada titik P dan rumah Amir terletak pada titik Q (lihat gambar). Sehingga, dalam perjalanan ke sekolah Andi akan menuju rumah Amir terlebih dahulu, kemudian bersama-sama dengan Amir ia akan berangkat ke sekolah. Ada berapa cara yang dapat ditempuh Andi untuk berangkat ke sekolah apabila ia harus melalui rumah Amir terlebih dahulu?

 \includegraphics[width = 8cm, height= 5cm]{Pictures/contoh3.JPG}

Banyaknya cara perjalanan dari titik P ke titik Q dilanjutkan ke titik R dapat digambarkan dengan diagram pohon seperti pada gambar berikut.

 \includegraphics[width = 8cm, height= 5cm]{Pictures/contoh3a.JPG}
 
 Dari diagram pohon tersebut terlihat rute perjalanan dari titik P ke titik R melalui titi Q ada 6 cara yang dapat ditulis dalam bentuk himpunan pasangan berurutan {(a, x), (a, y), (a, z), (b, x), (b, y), (b, z)}.

 
 \includegraphics[width = 8cm, height= 5cm]{Pictures/contoh3b.JPG}
 
 

%------------------------------------------------


\section{Aturan Perkalian}

1.Aturan perkalian

1.Aturan perkalian

	Kaidah Pencacahan adalah istilah dalam bahasan PELUANG. Kaidah pencacahan merupakan cara atau aturan untuk menghitung semua kemungkinan yang dapat terjadi dalam suatu percobaan tertentu. Metode yang dapat digunakan antara lain metode pengisian tempat (filling slot), Permutasi, dan Kombinasi. 

	Dalam kehidupan sehari-hari sering dihadapkan pada pemecahan masalah yang berkaitan dengan menentukan banyak cara yang mungkin terjadi dari sebuah percobaan, misalnya jika sebuah uang logam dilemparkan, akan tampak permukaan gambar atau angka. 



1. Aturan Perkalian
n1 = banyak cara unsur pertama
n2 = banyak cara unsur kedua

...
...
nk = banyak cara unsur ke-k
Maka banyak cara untuk menyusun k unsur yang tersedia adalah :
n1 x n2 x .... x nk.



\includegraphics[width = 8cm, height= 5cm]{Pictures/materikaidah1.png}

 Misalkan, dari 3 orang siswa, yaitu Algi, Bianda, dan Cahyadi akan dipilih untuk menjadi ketua kelas, sekretaris, dan bendahara dengan aturan bahwa seseorang tidak boleh merangkap jabatan pengurus kelas. Banyak cara 3 orang dipilih menjadi pengurus kelas tersebut akan dipelajari melalui uraian berikut. Amati Gambar
 
 \includegraphics[width = 8cm, height= 5cm]{Pictures/gen1.png}
Gambar  Aturan perkalian pemilihan pengurus kelas.
a. Untuk ketua kelas (K)
 Posisi ketua kelas dapat dipilih dari 3 orang, yaitu Algi  (A), Bianda (B), atau Cahyadi (C).

 Jadi, posisi ketua kelas dapat dipilih dengan 3 cara.
b. Untuk sekertaris (S)
 Jika posisi ketua kelas sudah terisi oleh seseorang maka  posisi sekretaris hanya dapat dipilih dari 2 orang yang belum terpilih menjadi pengurus kelas. 

 Jadi, posisi sekretaris dapat dipilih dengan 2 cara.
c. Untuk bendahara (A)
 
Jika posisi ketua kelas dan sekretaris sudah terisi maka posisi bendahara hanya ada satu pilihan, yaitu dijabat oleh orang yang belum terpilih menjadi pengurus kelas.

Jadi, posisi bendahara dapat dipilih dengan 1 cara.

Dengan demikian, banyak cara yang dilakukan untuk memilih 3 orang pengurus kelas dari 3 orang kandidat adalah :

3 x 2 x 1 = 6 cara.

Uraian tersebut akan lebih jelas apabila mengamati skema berikut.


\includegraphics[width = 6cm, height= 3cm]{Pictures/gen2.png}

Dari uraian tersebut, dapatkah Anda menyatakan aturan perkalian? Cobalah nyatakan aturan perkalian itu dengan kata-kata Anda sendiri.

Aturan Perkalian :

Misalkan,

• operasi 1 dapat dilaksanakan dalam n1 cara;
• operasi 2 dapat dilaksanakan dalam n2 cara;
• operasi k dapat dilaksanakan dalam nk cara.

Banyak cara k operasi dapat dilaksanakan secara berurutan adalah n = n1 x n2 x n3 ... x nk.

Contoh Soal 1 :

Berapa cara yang dapat diperoleh untuk memilih posisi seorang tekong, apit kiri, dan apit kanan dari 15 atlet sepak takraw pelatnas SEA GAMES jika tidak ada posisi yang rangkap? (Tekong adalah pemain sepak takraw yang melakukan sepak permulaan).

Jawaban :

• Untuk posisi tekong.

Posisi tekong dapat dipilih dengan 15 cara dari 15 atlet pelatnas yang tersedia.

• Untuk posisi apit kiri.

Dapat dipilih dengan 14 cara dari 14 atlet yang ada (1 atlet lagi tidak terpilih karena menjadi tekong).

• Untuk posisi apit kanan.

Cara untuk memilih apit kanan hanya dengan 13 cara dari 13 atlet yang ada (2 atlet tidak dapat dipilih karena telah menjadi tekong dan apit kiri).

Dengan demikian, banyak cara yang dilakukan untuk memilih posisi dalam regu sepak takraw adalah  15 x 14 x 13 = 2.730 cara.
Ingatlah :

Apabila terdapat n buah tempat yang akan diduduki oleh n orang, terdapat :

n x ( n - 1 ( x )n -2 ) x ... x 1 cara orang menduduki tempat tersebut.
2. Faktorial

Anda telah mempelajari, banyak cara yang dilakukan untuk memilih 3 orang pengurus kelas dari 3 orang kandidat adalah 3 x 2 x 1 = 6 cara.

Selanjutnya, 3 x 2 x 1 dapat dinyatakan dengan 3! (dibaca 3 faktorial). Jadi,

3! = 3 x 2 x 1 = 6

Dengan penalaran yang sama,

4! = 4 x 3 x 2 x 1 = 4 x 3! = 4 x 6 = 24
5! = 5 x 4 x 3 x 2 x 1 = 5 x 4! = 5 x 24 = 120
6! = 6 x 5! = 6 x 120 = 720

Uraian tersebut memperjelas definisi berikut.

Definisi :

a. n! = n x ( n – 1 ) x ( n – 2 ) ... x 3 x 2 x 1, dengan n bilangan asli, untuk n >=  2.
b. 1! = 1
c. 0! = 1

Contoh Soal 2 :

Hitunglah :

a. 7!
b. 17! / 0!16!
c. 12! / 2!8!
d. 8! / 5!

Penyelesaian :


\includegraphics[width = 7cm, height= 4cm]{Pictures/gen3.png}

Contoh Soal 3 :

Nyatakan bentuk-bentuk berikut ke dalam faktorial:

\includegraphics[width = 4cm, height= 2cm]{Pictures/gen5.png}

Penyelesaian :
\includegraphics[width = 5cm, height=3cm]{Pictures/gen4.png}

Contoh Soal 4 :

3. Tentukan nilai n dari ( n + 3 ) ! = 10 ( n + 2 ) !

Pembahasan :


\includegraphics[width = 5cm, height= 3cm]{Pictures/gen6.png}


Contoh Soal 5 :

Berapa banyak bilangan yang terdiri dari
3 angka dapat dibentuk dari angka-angka 1,2,3,4,5,6,7, dan 8 jika tiap-tiap angka boleh diulang?

Pembahasan :

Unsur pertama ada 8 pilihan, kedua ada 8, ketiga 8 (karena tiap angka boleh diulang.
                                    8          8          8                      
                                    Tinggal dikalikan
                                                                                    8 x 8 x 8 = 512

Contoh Soal 6 :

Berapa banyak susunan huruf yang dapat dibentuk oleh huruf-huruf pada kata “GARDU” tanpa ada pengulangan kata :

a.  Huruf pertama adalah huruf hidup

Jawab : maka, huruf konsonan 3 buah dan huruf vokal ada 2 buah. 

Jadi

\includegraphics[width = 5cm, height=3cm]{Pictures/soalkaidah1.png}


b.      Huruf pertama huruf mati dan huruf ketiga huruf hidup

Jawab :

\includegraphics[width = 5cm, height=3cm]{Pictures/soalkaidah2.png}

Contoh Soal 7 :

Jika diperlukan 5 orang laki-laki dan 4 orang perempuan untuk membentuk suatu barisan sedemikian rupa hingga yang perempuan menempati posisi genap, berapa banyak kemungkinan susunan barisan itu?

Jawab :

\includegraphics[width = 5cm, height=3cm]{Pictures/soalkaidah3.png}

Contoh Soal 8:

Ana mempunyai baju merah,hijau, biru, dan ungu. Ana juga memiliki rok hitam, putih, dan coklat. Berapa banyak pasangan baju dan rok yang dapat dipakai Ana?

Jawaban:

Jumlah baju = 4. 
Jumlah rok = 3.
 Jadi 3 x 4 = 12.
 Maksudnya Ana bisa memakai baju dan rok dengan warna : 
merah hitam, hijau hitam, biru hitam, dan seterusnya sampai 12 pasang.

Contoh Soal 9: 

Terdapat angka 3, 4, 5, 6, 7 yang hendak disusun menjadi suatu bilangan dengan tiga digit. 
Berapa banyak bilangan yang dapat disusun bila angka boleh berulang?

Jawaban:

Angka terdiri dari 3, 4, 5, 6, 7 dengan total ada lima angka. Dan membutuhkan tiga digit angka dari kombinasi lima angka tersebut secara acak. Tiga digit terdiri dari angka ratusan, puluhan dan satuan. Karena angka boleh berulang maka angka ratusan, puluhan dan satuan dapat diisi dengan kelima angka tersebut sehingga :

 5 x 5 x 5 = 125 kombinasi angka.
 
Contoh soal dan jawaban

1. Dari angka-angka 1, 2, 3, 4, 5, 6, akan disusun suatu bilangan yang terdiri dari 3 angka berbeda. Banyaknya bilangan yang dapat disusun adalah ... 

a.	18

b.	36

c.	60

d.	120

e.	216



2. Dari angka-angka 2, 3, 5, 7, dan 8 disusun bilangan yang terdiri atas tiga angka yang berbeda. Banyak bilangan yang dapat disusun adalah ... 

a.	10

b.	15

c.	20

d.	48

e.	60


3. Dari angka-angka 1,2,3,4,5, dan 6 akan disusun suatu bilangan terdiri dari empat angka. Banyak bilangan genap yang dapat tersusun dan tidak ada angka yang berulang adalah ... 

a.	120

b.	180

c.	360

d.	480

e.	648 


4. Dari angka-angka 3,4,5,6, dan 7 akan dibuat bilangan terdiri dari empat angka berlainan. Banyaknya bilangan kurang dari 6.000 yang dapat dibuat adalah …. 

a.	24

b.	36

c.	48

d.	72

e.	96


5. Banyaknya bilangan antara 1.000 dan 4.000 yang dapat disusun dari angka-angka 1,2,3,4,5,6 dengan tidak ada angka yang sama adalah ... 

a.	72

b.	80

c.	96

d.	120

e.	180


6. Perjalanan dari Surabaya ke Sidoarjo bisa melalui dua jalan dan dari Sidoarjo ke Malang bisa melalui tiga jalan. Banyaknya cara untuk bepergian dari Surabaya ke Malang melalui Sidoarjo ada ... 

a.	1 cara

b.	2 cara

c.	3 cara

d.	5 cara

e.	6 cara


7. Suatu keluarga yang tinggal di Surabaya ingin liburan ke Eropa via Arab Saudi. Jika rute dari Surabaya ke Arab Saudi sebanyak 5 rute penerbangan, sedangkan Arab Saudi ke Eropa ada 6 rute, maka banyaknya semua pilihan rute penerbangan dari Surabaya ke Eropa pergi pulang dengan tidak boleh melalui rute yang sama adalah ... 

a.	900

b.	800

c.	700

d.	600

e.	460


8. Jika seorang ibu mempunyai 3 kebaya, 5 selendang, dan 2 buah sepatu, maka banyaknya komposisi pemakaian kebaya, selendang, dan sepatu adalah ... 

a.	6 cara

b.	8 cara

c.	10 cara

d.	15 cara

e.	30 cara

9. Seorang ingin melakukan pembicaraan melalui sebuah wartel. Ada 4 buah kamar bicara dan ada 6 buah nomor yang akan dihubungi. Banyak susunan pasangan kamar bicara dan nomor telepon yang dapat dihubungi adalah ... 

a.	10

b.	24

c.	360

d.	1.296

e.	4.096

10. Bagus memiliki koleksi 5 macam celana panjang dengan warna berbeda dan 15 kemeja dengan corak berbeda. Banyak cara Bagus berpakaian dengan penampilan berbeda adalah ... 

a.	5 cara

b.	15 cara

c.	20 cara

d.	30 cara

e.	75 cara



kunci jawaban :

1. D

2. E

3. B 

4. D

5. E

6. E

7. D

8. E

9. B

10.E


1.	UN 2012 BHS/A13
Dari 6 orang calon pengurus termasuk Doni akan dipilih ketua, wakil, dan bendahara. Jika Doni terpilih sebagai ketua maka banyak pilihan yang mungkin terpilih sebagai wakil dan bendahara adalah … pilihan

a.	12

b.	16

c.	20

d.	25

e.	30

2.	UN 2012 BHS/C37
Suatu regu pramuka terdiri dari 7 orang. Jika dipilih ketua, sekretaris, dan bendahara, maka banyak pasangan yang mungkin akan terpilih adalah …

a.	100

b.	110

c.	200

d.	210

e.	300

3.	UN 2010 BAHASA PAKET A 
Dalam rangka memperingati HUT RI, Pak RT membentuk tim panitia HUT RI yang dibentuk dari 8 pemuda untuk dijadikan ketua panitia, sekretaris, dan bendahara masing-masing 1 orang. Banyaknya cara pemilihan tim panitia yang dapat disusun adalah …

a.	24

b.	56

c.	168

d.	336

e.	6720


4.	UN 2012 IPS/B25
Dari 7 orang pengurus suatu ekstrakurikuler akan dipilih seorang ketua, wakil ketua, sekretaris, bendahara, dan humas. Banyak cara pemilihan pengurus adalah ….

a.	2.100

b.	2.500

c.	2.520


d.	4.200

e.	8.400

5.	UN 2010 IPS PAKET B 
Dari 7 orang pengurus suatu ekstrakurikuler akan dipilih seorang ketua, wakil ketua, sekretaris, bendahara, dan humas. Banyak cara pemilihan pengurus adalah …

a.	2.100

b.	2.500

c.	2.520

d.	4.200


e.	8.400

6.	UN 2012 BHS/B25
Dari 7 orang pelajar berprestasi di suatu sekolah akan dipilih 3 orang pelajar berprestasi I, II, dan III. Banyaknya cara susunan pelajar yang mungkin terpilih sebagai pelajar berprestasi I, II, dan III adalah …

a.	21

b.	35

c.	120

d.	210

e.	720


7.	UN 2010 IPS PAKET A 
Dalam kompetisi bola basket yang terdiri dari 10 regu akan dipilih juara 1, 2, dan 3. Banyak cara memilih adalah …

a.	120

b.	360

c.	540

d.	720

e.	900

8.	UN 2011 IPS PAKET 46
Jika seorang penata bunga ingin mendapatkan informasi penataan bunga dari 5 macam bunga yang berbeda, yaitu B1, B2, …, B5 pada lima tempat yang tersedia, maka banyaknya formasi yang mungkin terjadi adalah …

a.	720

b.	360

c.	180

d.	120

e.	24

9.	UN 2011 IPS PAKET 12 
Banyak cara memasang 5 bendera dari negara yang berbeda disusun dalam satu baris adalah …

a.	20

b.	24

c.	69

d.	120

e.	132

10.	UN 2008 BAHASA PAKET A/B 
Di depan sebuah gedung terpasang secara berjajar sepuluh tiang bendera. Jika terdapat 6 buah bendera yang berbeda, maka banyak cara berbeda menempatkan bendera-bendera itu pada tiang-tiang tersebut adalah …

a.	10!6!

b.	10!4!

c.	6!4!

d.	10!2!

e.	6!2!

11.	UN 2010 BAHASA PAKET A/B 
Susunan berbeda yang dapat dibentuk dari kata “DITATA” adalah …

a.	90

b.	180

c.	360

d.	450

e.	720

12.	UN 2008 BAHASA PAKET A/B 
Nilai kombinasi 8C3 sama dengan …

a.	5

b.	40

c.	56

d.	120

e.	336

13.	UN 2009 BAHASA PAKET A/B 
Diketahui himpunan A = {1, 2, 3, 4, 5} Banyak himpunan bagian A yang banyak anggotanya 3 adalah …

a.	6

b.	10

c.	15

d.	24

e.	30

14.	UN 2012 BHS/A13 
Banyaknya cara memilih 3 orang utusan dari 10 orang calon untuk mengikuti suatu perlombaan adalah …

a.	120

b.	180

c.	240

d.	360

e.	720

15.	UN 2010 IPS PAKET B 
Banyak cara menyusun suatu regu cerdas cermat yang terdiri dari 3 siswa dipilih dari 10 siswa yang tersedia adalah …

a.	80

b.	120

c.	160

d.	240

e.	720

16.	UN 2010 BAHASA PAKET A/B 
Banyak kelompok yang terdiri atas 3 siswa berbeda dapat dipilih dari 12 siswa pandai untuk mewakili sekolahnya dalam kompetisi matematika adalah …

a.	180

b.	220

c.	240

d.	420

e.	1.320

17.	UN 2011 IPS PAKET 12 
Dari 20 kuntum bunga mawar akan diambil 15 kuntum secara acak. Banyak cara pengambilan ada …

a.	15.504

b.	12.434

c.	93.024

d.	4.896

e.	816

18.	UN 2012 BHS/B25 
Lima orang bermain bulutangkis satu lawan satu secara bergantian. Banyaknya pertandingan adalah …

a.	5

b.	10

c.	15

d.	20

e.	25

19.	UN 2012 BHS/C37 
Dari 8 pemain basket akan dibentuk tim inti yang terdiri dari 5 pemain. Banyaknya susunan tim inti yang mungkin terbentuk adalah …

a.	56

b.	36

c.	28

d.	16

e.	5

20.	UN 2011 IPS PAKET 46 
Kelompok tani Suka Maju terdiri dari 6 orang yang berasal dari dusun A dan 8 orang berasal dari dusun B. Jika dipilih 2 orang dari dusun A dan 3 orang dari dusun B untuk mengikuti penelitian tingkat kabupaten, maka banyaknya susunan kelompok yang mungkin terjadi adalah …

a.	840

b.	720

c.	560

d.	350

e.	120

21.	UN 2009 IPS PAKET A/B 
Dari 20 orang siswa yang berkumpul, mereka saling berjabat tangan, maka banyaknya jabatan tangan yang terjadi adalah …

a.	40

b.	80


c.	190

d.	360

e.	400

22.	UN 2011 BHS PAKET 12 
Dari 10 warna berbeda akan dibuat warna-warna baru yang berbeda dari campuran 4 warna dengan banyak takaran yang sama. Banyaknya warna baru yang mungkin dibuat adalah … warna

a.	200

b.	210

c.	220

d.	230

e.	240

23.	UN 2010 BAHASA PAKET A 
Seorang ibu mempunyai 8 sahabat. Banyak komposisi jika ibu ingin mengundang 5 sahabatnya untuk makan malam adalah …

a.	8! 5! 

b.	8! 3! 

c.	8!3!

d.	8!5!

e.	8!5!3!

24.	UN 2008 BAHASA PAKET A/B 
Seorang peserta ujian harus mengerjakan 6 soal dari 10 soal yang ada. Banyak cara peserta memilih soal ujian yang harus dikerjakan adalah …

a.	210

b.	110

c.	230

d.	5.040

e.	5.400 


KUNCI JAWABAN

1.C

2.D

3.D

4.C

5.C

6.D

7.C

8.D

9.D

10.B

11.D

12.C

13.B

14.A

15.B

16.B

17.A

18.B

19.A

20.A

21.C

22.B

23.E

24.A




\section{Permutasi}
	Dalam kehidupan sehari-hari kita sering menghadapi masalah pengaturan suatu obyek yang terdiri dari beberapa unsur, baik yang disusun dengan mempertimbangkan urutan sesuai dengan posisi yang diinginkan maupun yang tidak. Misalnya menyusun kepanitiaan yang terdiri dari Ketua, Sekretaris dan Bendahara dimana urutan untuk posisi tersebut dipertimbangkan atau memilih beberapa orang untuk mewakili sekelompok orang dalam mengikuti suatu kegiatan yang dalam hal ini urutan tidak menjadi pertimbangan. Dalam matematika, penyusunan obyek yang terdiri dari beberapa unsur dengan mempertimbangkan urutan disebut dengan permutasi, sedangkan yang tidak mempertimbangkan urutan disebut dengan kombinasi.


Masalah penyusunan kepanitiaan yang terdiri dari Ketua, Sekretaris dan Bendahara dimana urutan dipertimbangkan merupakan salah satu contoh permutasi. Jika terdapat 3 orang (misalnya Amir, Budi dan Cindy) yang akan dipilih untuk menduduki posisi tersebut, maka dengan menggunakan Prinsip Perkalian kita dapat menentukan banyaknya susunan panitia yang mungkin, yaitu:


• Pertama menentukan Ketua, yang dapat dilakukan dalam 3 cara.


• Begitu Ketua ditentukan, Sekretaris dapat ditentukan dalam 2 cara.


• Setelah Ketua dan Sekretaris ditentukan, Bendahara dapat ditentukan dalam 1 cara.

 
• Sehingga banyaknya susunan panitia yang mungkin


Secara formal Permutasi didefinisikan sebagai berikut:


Permutasi dari n unsur yang berbeda x1,x2,...,xn adalah pengurutan dari n unsur tersebut.


CONTOH :


Tentukan permutasi dari 3 huruf yang berbeda, misalnya ABC !


Permutasi dari huruf ABC adalah ABC, ACB, BAC, BCA, CAB, CBA. Sehingga terdapat 6 permutasi dari huruf ABC.

TEOREMA 2


Terdapat n permutasi dari n unsur yang berbeda.


Bukti


Asumsikan bahwa permutasi dari undur yang berbeda merupakan aktivitas yang terdiri dari langkah ang beurutan. Langkah pertama adalah memilih unsur pertama yang bisa dilakukandebgn n cara. Langkah kedua adlah memilih unsuur pertama sudah terpilih. Lanjutkan langkah tersebut sampai langkah ke n yang bisa dilakukan dengan 1 cara. Berdasarkan prinsip perkalian, terdapat


n(n-1)(n-2)...2.1=n!


permutasi n unsur berbeda

Contoh :


Berapa banyak permutasi dari huruf ABC ?


Terdapat 3.2.1 = 6 permutasi dari huruf ABC

Berapa banyak permutasi dari huruf ABCDEF jika subuntai ABC harus selalu muncul bersama?


Karena subuntai ABC harus selalu muncul bersama, maka subuntai ABC bisa dinyatakan sebagai satu unsur. Dengan demikian terdapat 4 unsur yang dipermutasikan, sehingga banyaknya permutasi adalah 4.3.2.1 = 24.

DEFINISI


Permutasi-r dari n unsur yang berbeda x1,x2,...,xn adalah pengurutan dari sub-himpunan dengan r anggota dari himpunan {x1,x2,...,xn}. Banyaknya permutasi-r dari n unsur yang berbeda dinotasikan dengan P(n,r).

Tentukan permutasi-3 dari 5 huruf yang berbeda, misalnya ABCDE.


Permutasi-3 dari huruf ABCDE adalah

ABC ABD ABE ACB ACD ACE 

ADB ADC ADE AEB AEC AED

BAC BAD BAE BCA BCD BCE 

BDA BDC BDE BEA BEC BED 

CAB CAD CAE CBA CBD CBE 

CDA CDB CDE CEA CEB CED 

DAB DAC DAE DBA DBC DBE 

DCA DCB DCE DEA DEB DEC 

EAB EAC EAD EBA EBC EBD 

ECA ECB ECD EDA EDB EDC

Sehingga banyaknya permutasi-3 dari 5 huruf ABCDE adalah 60.


TEOREMA 3

Banyaknya permutasi-r dari n unsur yang berbeda adalah

\includegraphics[width = 8cm, height= 4cm]{Pictures/herlin1.png}

Asumsikan bahwa permutasi-r n unsur yang berbeda merupakan aktifitas yang terdiri dari r langkah yang merupakan ktifitas yang terdiri dari r langkah yang berurutan. Langkah pertama adalah memilih unsur pertama yang bisa dilakukan dengan n cara. angkah kedua adalah memilih unsur kedu yang bisa dilakukan dengan n-1 cara karena unsur pertama sudah terpilih. Lanjutkan langkah tersebut sampai dengan n-r +1 cara. Berdasarkan prinsip perkalian Diperoleh


\includegraphics[width = 12cm, height= 6cm]{Pictures/herlin2.png}

Gunakan Teorema 3.2 untuk menentukan permutasi-3 dari 5 huruf yang berbeda, misalnya ABCDE.


Karena r = 3 dan n = 5 maka permutasi-3 dari 5 huruf ABCDE adalah

\includegraphics[width = 12cm, height= 4cm]{Pictures/herlin3.png}

Jadi banyaknya permutasi-3 dari 5 huruf ABCDE adalah 60.

\section{kombinasi}


Berbeda dengan permutasi yang urutan menjadi pertimbangan, pada kombinasi urutan tidak dipertimbangkan. Misalnya pemilihan 3 orang untuk mewakili kelompak 5 orang (misalnya Dedi, Eka, Feri, Gani dan Hari) dalam mengikuti suatu kegiatan. Dalam masalah ini, urutan tidak dipertimbangkan karena tidak ada bedanya antara Dedi, Eka dan Feri dengan Eka, Dedi dan Feri. Dengan mendata semua kemungkinan 3 orang yang akan dipilih dari 5 orang yang ada, diperoleh:

\includegraphics[width = 12cm, height= 4cm]{Pictures/herlin4.png}

Sehingga terdapat 10 cara untuk memilih 3 orang dari 5 orang yang ada.

Selanjutnya kita dapat kombinasikan secara formal.

Seperti keterangan diatas.



DEFINISI


Kombinasi-r dari n unsur yang berbeda x1,x2,...,xn adalah seleksi tak terurut r anggota dari himpunan {x1,x2,...,xn} (sub-himpunan dengan r unsur). Banyaknya kombinasi-r dari n unsur yang berbeda dinotasikan dengan C(n,r) atau (n r).

CONTOH :

Tentukan kombinasi-3 dari 5 huruf yang berbeda, misalnya ABCDE.
Kombinasi-3 dari huruf ABCDE adalah

\includegraphics[width = 12cm, height= 4cm]{Pictures/herlin5.png}

Sehingga banyaknya kombinasi-3 dari 5 huruf ABCDE adalah 10.

Teorema 3

Banyaknya kombinasi-r dari n unsur yang berbeda adalah

\includegraphics[width = 12cm, height= 4cm]{Pictures/herlin6.png}

Bukti.

• Langkah pertama adalah menghitung kombinasi-r dari n, yaitu C(n,r). 


• Langkah kedua adalah mengurutkan r unsur tersebut, yaitu r!.

 Dengan demikian
 
 
Pembuktian dilakukan dengan menghitung permutasi dari n unsur yang berbeda dengan cara berikut ini.

\includegraphics[width = 12cm, height= 6cm]{Pictures/herlin7.png}

seperti yang diinginkan

Gunakan Teorema 3.3 untuk menentukan kombinasi-3 dari 5 huruf yang berbeda, misalnya ABCDE.


Karena r = 3 dan n = 5 maka kombinasi-3 dari 5 huruf ABCDE adalah

\includegraphics[width = 12cm, height= 4cm]{Pictures/herlin8.png}


Jadi banyaknya kombinasi-3 dari 5 huruf ABCDE adalah 10.


Contoh

Berapa banyak cara sebuah panitia yang terdiri dari 4 orang bisa dipilih dari 6 orang


Karena panitia yang terdiri dari 4 orang merupakan susunan yang tidak terurut, maka masalah ini merupakan kombinasi-4 dari 6 unsur yang tersedia. Sehingga dengan mengunakan Teorema 3.3 dimana n = 6 dan r = 4 diperoleh: 


\includegraphics[width = 12cm, height= 4cm]{Pictures/herlin9.png}

Jadi terdapat 15 cara untuk membentuk sebuah panitia yang terdiri dari 4 orang bisa dipilih dari 6 orang.

Contoh :

Berapa banyak cara sebuah panitia yang terdiri dari 2 mahasiswa dan 3 mahasiswi yang bisa dipilih dari 5 mahasiswa dan 6 mahasiswi?

Pertamai, memilih 2 mahasiswa dari 5 mahasiswa yang ada, yaitu:

\includegraphics[width = 12cm, height= 4cm]{Pictures/herlin10.png}

Kedua, memilih 3 mahasiswi dari 6 mahasiswi yang ada, yaitu:


\includegraphics[width = 12cm, height= 4cm]{Pictures/herlin11.png}

Sehingga terdapat 10.20 = 200 cara untuk membentuk sebuah panitia yang terdiri dari 2 mahasiswa dan 3 mahasiswi yang bisa dipilih dari 5 mahasiswa dan 6 mahasiswi?

Kalau pada pembahasan permutasi sebelumnya unsur-unsur yang diurutkan berbeda, pada bagian ini akan dibahas permutasi yang digeneralisasikan dengan membolehkan pengulangan unsur-unsur yang akan diurutkan, dengan kata lain unsur-unsurnya boleh sama.



Misalkan kita akan mengurutkan huruf-huruf dari kata KAKIKUKAKU. Karena huruf-huruf pada kata tersebut ada yang sama, maka banyaknya permutasi bukan 10!, tetapi kurang dari 10!.


Untuk mengurutkan 10 huruf pada kata KAKIKUKAKU dapat dilakukan dengan cara:

Kalau pada pembahasan permutasi sebelumnya unsur-unsur yang diurutkan berbeda, pada bagian ini akan dibahas permutasi yang digeneralisasikan dengan membolehkan pengulangan unsur-unsur yang akan diurutkan, dengan kata lain unsur-unsurnya boleh sama.
Misalkan kita akan mengurutkan huruf-huruf dari kata KAKIKUKAKU. Karena huruf-huruf pada kata tersebut ada yang sama, maka banyaknya permutasi bukan 10!, tetapi kurang dari 10!.
Untuk mengurutkan 10 huruf pada kata KAKIKUKAKU dapat dilakukan dengan cara:

-Asumsikan masalah ini dengan 10 posisi kosong yang akan diisi dengana huruf-huruf pada kata KAKIKUKAKU.

-Pertama menempatkan 5 huruf K pada 10 posisi kosong, yang dapat dilkukan dalam c(10,5) cara

-Setelah 5 huruf k ditempatkan, maka terdapat 10-5 =5 posisi kosong

-Berikutnya adlah menempatkan 2 huruf A pada 5 posisi kosong, yang dapat dilakukan dalam c(5,2) cara. b begitu 2huruf A ditempatkan, terdapat C(3,2) cara untuk menempatkan 2 huruf A ditempatkan, terdapat C(3,2) cara untuk menempatkan 2 huruf U pada 3 posisi kosong yang ada


- Akhirnya terdapat C(1,1) cara untuk menempatkan 1 huruf I pada 1 posisi kosong yang tersisa

 
\includegraphics[width = 12cm, height= 4cm]{Pictures/herlin11.png}

Jadi banyaknya cara untuk mengurutkan huruf-huruf dari kata KAKIKUKAKU adalah 7560.

Secara umum banyaknya permutasi dari obyek yang mempunyai beberapa unsur sama dapat dijabarkan seperti pada teorema berikut ini.

TEOREMA

Misalkan X merupakan sebuah barisan yang mempunyai n unsur, dimana terdapat n1 unsur yang sama untuk jenis 1, n2 unsur yang sama untuk jenis 2 dan seterusnya sampai nt unsur yang sama untuk jenis t. Banyaknya permutasi dari barisan X adalah

\includegraphics[width = 12cm, height= 6cm]{Pictures/herlin12.png}

%------------------------------------------------

\section{Notations}\index{Notations}

\begin{notation}
Given an open subset $G$ of $\mathbb{R}^n$, the set of functions $\varphi$ are:
\begin{enumerate}
\item Bounded support $G$;
\item Infinitely differentiable;
\end{enumerate}
a vector space is denoted by $\mathcal{D}(G)$. 
\end{notation}

%------------------------------------------------

\section{Remarks}\index{Remarks}

This is an example of a remark.

\begin{remark}
The concepts presented here are now in conventional employment in mathematics. Vector spaces are taken over the field $\mathbb{K}=\mathbb{R}$, however, established properties are easily extended to $\mathbb{K}=\mathbb{C}$.
\end{remark}

%------------------------------------------------

\section{Corollaries}\index{Corollaries}

This is an example of a corollary.

\begin{corollary}[Corollary name]
The concepts presented here are now in conventional employment in mathematics. Vector spaces are taken over the field $\mathbb{K}=\mathbb{R}$, however, established properties are easily extended to $\mathbb{K}=\mathbb{C}$.
\end{corollary}

%------------------------------------------------

\section{Propositions}\index{Propositions}

This is an example of propositions.

\subsection{Several equations}\index{Propositions!Several Equations}

\begin{proposition}[Proposition name]
It has the properties:
\begin{align}
& \big| ||\mathbf{x}|| - ||\mathbf{y}|| \big|\leq || \mathbf{x}- \mathbf{y}||\\
&  ||\sum_{i=1}^n\mathbf{x}_i||\leq \sum_{i=1}^n||\mathbf{x}_i||\quad\text{where $n$ is a finite integer}
\end{align}
\end{proposition}

\subsection{Single Line}\index{Propositions!Single Line}

\begin{proposition} 
Let $f,g\in L^2(G)$; if $\forall \varphi\in\mathcal{D}(G)$, $(f,\varphi)_0=(g,\varphi)_0$ then $f = g$. 
\end{proposition}

%------------------------------------------------

\section{Examples}\index{Examples}

This is an example of examples.

\subsection{Equation and Text}\index{Examples!Equation and Text}

\begin{example}
Let $G=\{x\in\mathbb{R}^2:|x|<3\}$ and denoted by: $x^0=(1,1)$; consider the function:
\begin{equation}
f(x)=\left\{\begin{aligned} & \mathrm{e}^{|x|} & & \text{si $|x-x^0|\leq 1/2$}\\
& 0 & & \text{si $|x-x^0|> 1/2$}\end{aligned}\right.
\end{equation}
The function $f$ has bounded support, we can take $A=\{x\in\mathbb{R}^2:|x-x^0|\leq 1/2+\epsilon\}$ for all $\epsilon\in\intoo{0}{5/2-\sqrt{2}}$.
\end{example}

\subsection{Paragraph of Text}\index{Examples!Paragraph of Text}

\begin{example}[Example name]
\lipsum[2]
\end{example}

%------------------------------------------------

\section{Exercises}\index{Exercises}

This is an example of an exercise.

\begin{exercise}
This is a good place to ask a question to test learning progress or further cement ideas into students' minds.
\end{exercise}

%------------------------------------------------

\section{Problems}\index{Problems}

\begin{problem}
What is the average airspeed velocity of an unladen swallow?
\end{problem}

%------------------------------------------------

\section{Vocabulary}\index{Vocabulary}

Define a word to improve a students' vocabulary.

\begin{vocabulary}[Word]
Definition of word.
\end{vocabulary}

%----------------------------------------------------------------------------------------
%	PART
%----------------------------------------------------------------------------------------

\part{Part Two}

%----------------------------------------------------------------------------------------
%	CHAPTER 3
%----------------------------------------------------------------------------------------
\chapterimage{chapter_head_2.pdf} % Chapter heading image

\chapter{Statistika}
\section{A. Penyajian Data Dalam Bentuk Diagram}\index{B. Penyajian Data Dalam Bentuk Diagram}

Statistika adalah cabang dari matematika terapan yang mempunyai cara-cara, maksudnya mengkaji/membahas, mengumpulkan, dan menyusun data, mengolah dan menganalisis data, serta menyajikan data dalam bentuk kurva atau diagram, menarik kesimpulan, menasirkan parameter, dan menguji hipotesa yang didasarkan pada hasil pengolahan data. Contoh : statistik jumlah lulusan siswa SMA dari tahun ke tahun, statistic jumlah kendaraan yang melewatu suau jalan, statistic perdagangan antara Negara-negara di Asia, dan sebagainya.

1.	Diagram Garis
Penyajian data statistik dengan menggunakan diagram berbentuk garis lurus disebut diagram garis lurus atau diagram garis. Diagram garis biasanya digunakan untuk menyajikan data statistic yang diperoleh berdasarkan pengamatan dari waktu ke waktu sevara berurutan.
	Sumbu X menunjukan waktu-waktu pengamatan, sedangkan sumbu Y menunjukkan nilai data pengamatan untuk suatu waktu tertentu. Kumpulan waktu dan pengamatan membentuk titik-titik pada bidang XY, selanjutnya kolom dari tiap dua titik yang berdekatan tadi dihubungkan dengan garis lurus sehingga akan diperoleh diagram garis atau grafik garis. Diagram ini biasanya digunakan untuk menggambarkan suatu kondisi yang berlangsung secara kontinu, misalnya perkembangan nilai tukar mata uang suatu Negara terhadap nilai tukar Negara lain, jumlah penjualan setiap waktu tertentu, dan jumlah penduduk suatu daerah setiap periode tertentu. Untuk lebih jelasnya, perhatikan contoh soal berikut.

Contoh soal 1
Fluktuasi nilai tukar rupiah terhadap doal AS dari tanggal 18 Februari 2008 sampai dengan tanggal 22 Februari 2008 ditunjukkan oleh tabel berikut.

\includegraphics[width = 6cm, height= 4cm]{Pictures/Gb1_diana.png}

Nyatakan data di atas dalam bentuk diagram garis.
Penyelesaian
Jika digambar dengan menggunakan diagram garis adalah sebagai berikut.

\includegraphics[width = 6cm, height= 4cm]{Pictures/Gb2_diana.png}

Contoh Soal 2
Sebuah dealer mobil sejak tahun 1995 hingga akhir tahun 2004 selalu mencatat jumlah mobil yang terjual setiap tahun sebagai berikut.

\includegraphics[width = 6cm, height= 4cm]{Pictures/Gb3_diana.png}
Buatlah diagram garis untuk data tersebut.

Penyelesaian

\includegraphics[width = 6cm, height= 4cm]{Pictures/Gb4_diana.png}

Dari diagram tersebut, tampak penjualan mobil terbanyak pada tahun 2001. Dari tahun 1995-1997, penjualan mobil cenderung mengalami kenaikan dan tahun 1998-1999 cenderung mengalami penurusan. 

Contoh Soal 3
Sebuah perusahaan yang memproduksi barang elektronik mencatat akumulasi biaya produksi tahunan dan akumulasi nilai penjualan selama sepuluh tahun dari tahun 1995 sampai dengan 2004 sebagai berikut (dalam jutaan rupiah)

\includegraphics[width = 6cm, height= 4cm]{Pictures/Gb5_diana.png}

Buatlah diagram garis untuk data tersebut

Penyelesaian

Diagram garis untuk akumulasi biaya produksi dan akumulasi nilai penjualan adalah sebagai berikut.

\includegraphics[width = 6cm, height= 4cm]{Pictures/Gb6_diana.png}

Dari gambar di atas Anda dapat mengetahui bahwa perusahaan mulai memperoleh laba (keuntungan) di antara tahun 1999 dan 2000, yaitu pada saat kedua garis berpotongan. Titik potong kedua garis tersebut disebut titik pulang pokok (break event point).

	Diagram garis biasnaya digunakan untuk menaksir atau memperkirakan data berdasarkan pola-pola yang telah diperoleh. Diagram pada Gambar 1.2 merupakan diagram garis tunggal. Adapun diagram pada Gambar 1.3 disebut diagram garis majemuk, dikatakan majemuk karena dalam satu gambar terdapat lebih dari satu garis. Diagram garis majemuk biasanya digunakan untuk membandingkan dua keadaan atau lebih yang mempunyai hubungan, misalnya diagram dua garis yang melukiskan akumulasi biaya produksi dan akumulasi nilai penjualan setiap tahun selama sepuluh tahun.


\section{B. Penyajian Data Dalam Bentuk Tabel distribusi Frekuensi}\index{B. Penyajian Data Dalam Bentuk Tabel distribusi Frekuensi}

Selain dalam bentuk diagram, penyajian data juga dengan menggunakan tabel distribusi
frekuensi. Berikut ini akan dipelajari lebih jelas mengenai tabel distribusi frekuensi tersebut.


1. Ditribusi Frekuensi Tunggal
Data tunggal seringkali dinyatakan dalam bentuk daftar bilangan, namun kadangkala
dinyatakan dalam bentuk tabel distribusi frekuensi. Tabel distribusi frekuensi tunggal
merupakan cara untuk menyusun data yang relatif sedikit. Perhatikan contoh data berikut.
5, 4, 6, 7, 8, 8, 6, 4, 8, 6, 4, 6, 6, 7, 5, 5, 3, 4, 6, 6 , 8, 7, 8, 7, 5, 4, 9, 10, 5, 6, 7, 6, 4, 5, 7, 7, 4, 8, 7, 6

Dari data di atas tidak tampak adanya pola yang tertentu maka agar mudah dianalisis
data tersebut disajikan dalam tabel seperti di bawah ini

\includegraphics[width = 6cm, height= 4cm]{Pictures/1reska.png}

Daftar di atas sering disebut sebagai distribusi frekuensi dan karena datanya
tunggal maka disebut distribusi frekuensi tunggal.


2. Distribusi Frekuensi Bergolong

Tabel distribusi frekuensi bergolong biasa digunakan untuk menyusun data yang
memiliki kuantitas yang besar dengan mengelompokkan ke dalam interval-interval kelas
yang sama panjang. Perhatikan contoh data hasil nilai pengerjaan tugas Matematika
dari 40 siswa kelas XI berikut ini.


66 75 74 72 79 78 75 75 79 71

75 76 74 73 71 72 74 74 71 70

74 77 73 73 70 74 72 72 80 70

73 67 72 72 75 74 74 68 69 80



Apabila data di atas dibuat dengan menggunakan tabel distribusi frekuensi tunggal,
maka penyelesaiannya akan panjang sekali. Oleh karena itu dibuat tabel distribusi
frekuensi bergolong dengan langkah-langkah sebagai berikut.
a. Mengelompokkan ke dalam interval-interval kelas yang sama panjang, misalnya
65 – 67, 68 – 70, … , 80 – 82. Data 66 masuk dalam kelompok 65 – 67.
b. Membuat turus (tally), untuk menentukan sebuah nilai termasuk ke dalam kelas
yang mana.
c. Menghitung banyaknya turus pada setiap kelas, kemudian menuliskan banyaknya
turus pada setiap kelas sebagai frekuensi data kelas tersebut. Tulis dalam kolom
frekuensi.
d. Ketiga langkah di atas direpresentasikan pada tabel berikut ini.


\includegraphics[width = 6cm, height= 4cm]{Pictures/2reska.png}

Istilah-istilah yang banyak digunakan dalam pembahasan distribusi frekuensi
bergolong atau distribusi frekuensi berkelompok antara lain sebagai berikut.



a. Interval Kelas

Tiap-tiap kelompok disebut interval kelas atau sering disebut interval atau kelas
saja. Dalam contoh sebelumnya memuat enam interval ini.

65 – 67 Interval kelas pertama

68 – 70 Interval kelas kedua

71 – 73 Interval kelas ketiga

74 – 76 Interval kelas keempat

77 – 79 Interval kelas kelima

80 – 82 Interval kelas keenam


b. Batas Kelas

Berdasarkan tabel distribusi frekuensi di atas, angka 65, 68, 71, 74, 77, dan 80

merupakan batas bawah dari tiap-tiap kelas, sedangkan angka 67, 70, 73, 76, 79,

dan 82 merupakan batas atas dari tiap-tiap kelas.


c. Tepi Kelas (Batas Nyata Kelas)

Untuk mencari tepi kelas dapat dipakai rumus berikut ini.

\includegraphics[width = 6cm, height= 1cm]{Pictures/3reska.png}

Dari tabel di atas maka tepi bawah kelas pertama 64,5 dan tepi atasnya 67,5, tepi
bawah kelas kedua 67,5 dan tepi atasnya 70,5 dan seterusnya.

d. Lebar kelas
Untuk mencari lebar kelas dapat dipakai rumus:

Untuk mencari lebar kelas dapat dipakai rumus:

Lebar kelas = tepi atas – tepi bawah

Jadi, lebar kelas dari tabel diatas adalah 67,5 – 64,5 = 3.

e. Titik Tengah
Untuk mencari titik tengah dapat dipakai rumus:

\includegraphics[width = 8cm, height= 1cm]{Pictures/4reska.png}

Dari tabel di atas: titik tengah kelas pertama = 0.5 (67 + 65) = 66

titik tengah kedua =0.5 (70 + 68) = 69
 dan seterusnya.
 
 
3. Distribusi Frekuensi Kumulatif


Daftar distribusi kumulatif ada dua macam, yaitu sebagai berikut.

a. Daftar distribusi kumulatif kurang dari (menggunakan tepi atas).

b. Daftar distribusi kumulatif lebih dari (menggunakan tepi bawah).

Untuk lebih jelasnya, perhatikan contoh data berikut ini.

\includegraphics[width = 8cm, height= 2cm]{Pictures/5reska.png}

Dari tabel di atas dapat dibuat daftar frekuensi kumulatif kurang dari dan lebih
dari seperti berikut: 




\includegraphics[width = 12cm, height= 2cm]{Pictures/6reska.png}



4. Histogram


Dari suatu data yang diperoleh dapat disusun dalam tabel distribusi frekuensi dan
disajikan dalam bentuk diagram yang disebut histogram. Jika pada diagram batang,
gambar batang-batangnya terpisah maka pada histogram gambar batang-batangnya berimpit. Histogram dapat disajikan dari distribusi frekuensi tunggal maupun distribusi
frekuensi bergolong. Untuk lebih jelasnya, perhatikan contoh berikut ini.
Data banyaknya siswa kelas XI IPA yang tidak masuk sekolah dalam 8 hari berurutan
sebagai berikut: 

\includegraphics[width = 12cm, height= 1cm]{Pictures/7reska.png}

Berdasarkan data diatas dapat dibentuk histogramnya seperti berikut dengan membuat
tabel distribusi frekuensi tunggal terlebih dahulu

\includegraphics[width = 4cm, height= 4cm]{Pictures/8reska.png}

5.Poligon Frekuensi


Apabila pada titik-titik tengah dari histogram dihubungkan dengan garis dan batangbatangnya
dihapus, maka akan diperoleh poligon frekuensi. Berdasarkan contoh di atas
dapat dibuat poligon frekuensinya seperti gambar berikut ini.

\includegraphics[width = 4cm, height= 4cm]{Pictures/9reska.png}

Untuk lebih jelasnya, perhatikan contoh soal berikut ini.

Contoh soal

Hasil pengukuran berat badan terhadap 100 siswa SMP X digambarkan dalam distribusi
bergolong seperti di bawah ini. Sajikan data tersebut dalam histogram dan poligon frekuensi.


\includegraphics[width = 5cm, height= 2cm]{Pictures/10reska.png}

Penyelesaian

Histogram dan poligon frekuensi dari tabel di atas dapat ditunjukkan sebagai berikut

\includegraphics[width = 6cm, height= 4cm]{Pictures/11reska.png}

6. Poligon Frekuensi Kumulatif

Dari distribusi frekuensi kumulatif dapat dibuat grafik garis yang disebut poligon
frekuensi kumulatif. Jika poligon frekuensi kumulatif dihaluskan, diperoleh kurva yang
disebut kurva ogive. Untuk lebih jelasnya, perhatikan 

contoh soal berikut ini.

\includegraphics[width = 4cm, height= 4cm]{Pictures/12reska.png}

Hasil tes ulangan Matematika terhadap
40 siswa kelas XI IPA digambarkan dalam
tabel di samping.

a. Buatlah daftar frekuensi kumulatif kurang
dari dan lebih dari.

b. Gambarlah ogive naik dan ogive turun

Penyelesaian

a. Daftar frekuensi kumulatif kurang dari dan lebih dari adalah sebagai berikut.

\includegraphics[width = 8cm, height= 3cm]{Pictures/13reska.png}

b. Ogive naik dan ogive turun

Daftar frekuensi kumulatif kurang dari dan lebih dari dapat disajikan dalam bidang
Cartesius. 
Tepi atas (67,5; 70,5; …; 82,5) atau tepi bawah (64,5; 67,5; …; 79,5)
diletakkan pada sumbu X sedangkan frekuensi kumulatif kurang dari atau frekuensi
kumulatif lebih dari diletakkan pada sumbu Y.
Apabila titik-titik yang diperlukan
dihubungkan, maka terbentuk kurva yang disebut ogive. Ada dua macam ogive,
yaitu ogive naik dan ogive turun. Ogive naik apabila grafik disusun berdasarkan
distribusi frekuensi kumulatif kurang dari. Sedangkan ogive turun apabila berdasarkan
distribusi frekuensi kumulatif lebih dari.
Ogive naik dan ogive turun data di atas adalah
sebagai berikut:


\includegraphics[width = 8cm, height= 2cm]{Pictures/14reska.png}



\section{Ukuran Pemusatan Data}\index{Ukuran Pemusatan Data}

Ukuran pemusatan data disebut juga sembarang ukuran yang menunjukkan pusat sekumpulan data, yang telah diurutkan dari angka yang terkecil sampai terbesar atau sebaliknya dari angka yang terbesar sampai terkecil. Beberapa fungsi dari ukuran pemusatan data adalah untuk membandingkan dua data atau contoh, karena sangat sulit untuk membandingkan banyaknya anggota dari masing-masing anggota populasi atau banyaknya anggota data contoh. Nilai ukuran pemusatan ini dibuat sehingga dapat mewakili seluruh nilai pada data yang bersangkutan.

Ukuran pemusatan yang sering digunakan adalah mean, modus, dan median. Nilai tengah (mean) akan sangat dipengaruh nilai banyaknya data. Median yang sangat beragam sulit dalam penggunaan parameter populasi. Dan modus hanya digunakan untuk data ukuran yang besar.

Salah satu ukuran yang paling penting untuk menggambarkan suatu distribusi data adalah nilai pusat data pengamat. Setiap pengukuran aritmatika yang ditujukan untuk menggambarkan suatu nilai yang mewakili nilai pusat atau nilai sentral dari suatu gugus data (himpunan pengamatan) dikenal sebagai ukuran tendensi sentral. Biasanya Ukuran pemusatan data sering kali digunakan agar data yang diperoleh mudah untuk dipahami oleh siswa. Ukuran pemusatan data debagi menjadi mean yang digunakan untuk mengetahui nilai rata rata pada setiap himpunan angka, median digunakan untuk mengetahui suatu nilai tengah suatu himpunan angka, dan modus adalah data yang sering muncul.


%------------------------------------------------

\section{Mean}\index{Mean}

Mean yaitu suatu nilai rata rata dan di dapatkan dari sekumpulan data adalah jumlah seluruh data dibagi banyaknya data. Dengan mengetahui mean suatu data, maka variasi data yang lain akan mudah diperkirakan.atau juga dapat disebut suatu metode yang sering digunakan untuk menggambarkan ukuran suatu data. Mean dapat dihitung dengan menjumlahkan seluruh nilai data pengukuran dan dibagi dengan banyaknya data yang digunakan. Definisi tersebut di nyatakan dengan persamaan sebagai berikut:

\begin{theorem}[Mean]
rumus mencari nilai rata-rata:
\begin{align}
Sampel\\
& \bar{X}=\frac{x_{1}+x_{2}+x_{3}+......+x_{n}}{n}=\sum_{i=0}^{n}\frac{x_{i}}{n}\\
Populasi\\
& \bar{\mu}=\frac{x_{1}+x_{2}+x_{3}+......+x_{n}}{n}=\sum_{i=0}^{n}\frac{x_{i}}{n}
\end{align}
\end{theorem}

Keterangan

$\sum$ = lambang penjumlahan semua gugus data pengamatan 

n = banyaknya sampel data 

N = banyaknya data populasi 

$\bar x$ = nilai rata-rata sampel

$\mu$ = nilai rata-rata populasi

\subsection{Distribusi frekuensi}\index{Theorems!Several Equations}
Rata-rata yang dihitung berdasarkan data yang sudah ditata dalam bentuk tabel distribusi frekuensi dan dapat ditentukan dengan menggunakan formula / rumus rumus yang sama dengan formula untuk menghitung nilai rata-rata dari data yang sudah dikelompokkan atau data yang terdistribusi, dengan rumus sebagai berikut:

\begin{theorem}[Mean]
rumus mencari nilai distribusi frekuensi:
\begin{align}
& \bar{X} = \frac{\sum f_{i}x_{i}}{\sum f_{i}} 
\end{align}
\end{theorem}

Keterangan
 
$ \sum $ = lambang penjumlahan semua gugus data

$ f_{i} $ = frekuensi data ke-i

$ \bar x $= nilai rata-rata sampel

%------------------------------------------------

\section{Median}\index{Median}

Median merupakan nilai tengah dari sekumpulan data yang telah diurutkan dari angka terkecil sampai ke angka terbesar. Median ditentukan berdasarkan jumlah data, dengan jumlah data yang ganjil maka mediannya memiliki nilai tengah dari data yang telah diurutkan, dan dengan jumlah data genap maka mediannya adalah mean / rataan dari dua bilangan yang ditengah data yang sudah diurutkan

\begin{theorem}[Mean]
rumus mencari nilai rata-rata:
\begin{align}
untuk n ganjil\\
& Me =x_{\frac{1}{2}(n+1)} \\
Untuk n genap\\
& Me =\frac{x_{\frac{n}{2}}+x_{\frac{n}{2}+1}}{2}
\end{align}
\end{theorem}

Keterangan 

$x_{\frac{n}{2}}$ = data pada urutan ke-$\frac{n}{2}$ setelah diurutkan


\section{Modus}\index{Modus}

Modus adalah data yang sering muncul atau data yang memiliki jumlah frekuensi paling banyak. Sebuah data dapat dikatakan tidak memiliki modus ketika seluruh data yang muncul memiliki frekuensi yang sama atau dapat disebut sebuah data memiliki modus lebih dari satu.
Untuk data yang ditampilkan dalam bentuk tabel distribusi frekuensi berkelompok, dapat digunakan menentukan letak modus dengan cara melihat kelas interval yang mempunyai frekuensi paling besar.Bila data mempunyai satu modus dapat disebut unimodal dan data yang memiliki dua modus disebut bimodal, sedangkan jika data mempunyai modus yang lebih dari dua disebut multimodal. Modus dapat dilambangkan dengan Mo

\begin{theorem}[Mean]
rumus mencari nilai rata - rata:
\begin{align}
untuk n ganjil\\
& Mo = T_{b}+(\frac{s_{1}}{s_{1}+s_{2}})i
\end{align}
\end{theorem}

Keterangan 

$Mo$ = Modus

$T_{b}$ = Tepi bawah dari kelas modus

$s_{1}$ = Selisih frekuensi kelas modus dengan frekuensi kelas sebelum kelas modus

$s_{2}$ = Selisih frekuensi kelas modus dengan frekuensi kelas sesudah kelas modus

$i$ = panjang kelas interval

\subsection{Rata-Rata Hitung (Mean/Arhitmetic Mean)}\index{Theorems!Several Equations}

Rata-rata merupakan nilai yang mewakili kumpul data yaitu nilai yang kurang dari nilai itu, nilai yang lebih dari nilai itu dan nilai itu sendiri. 
Contoh:
-	Ani cantik
-	Rina tidak cantik		=      Kesimpulannya rata-rata perempuan itu cantik
-	Dini  sangat cantik          
Mean dari sekumpulan data adalah jumlah dari kumpulan bilangan dibagi banyak bilangan tersebut. 

Untuk data tunggal seperti: $x_{1}, x_{2}, x_{3}.....x_{n} $. Maka:

\begin{theorem}[Mean]
rumus mencari nilai rata-rata:
\begin{align}
& \overline{x} = \frac{\sum x_{i}}{n}
\end{align}
\end{theorem}

Keterangan:
	 
$\overline{x}$= Rataan Hitung
		 
$n$ = banyak data 

$x_{i}$ = data ke-i

Contoh = Tentukan rata-rata dari nilai siswa sebagai berikut: 70, 69, 45, 80 dan 56!

\begin{table}[h]
\centering
\begin{tabular}{l l}
\toprule
\textbf{$x_{i}$} & \textbf{$f_{i}$}\\
\midrule
70 & 5 \\
69 & 6 \\
45 & 3 \\
80 & 1 \\
56 & 1 \\
\bottomrule
\end{tabular}
\caption{Data Frekuensi Tunggal}
\end{table}

$x_{i}$ menyatakan nilai ujian dan $f_{i}$ menyatakan frekuensi untuk nilai $x_{i}$ yang bersesuaian. Untuk mencari rata-rata tabel diatas, akan lebih mudah bila dibuat tabel penolong seperti berikut:


$\overline{x} = \frac{\sum f_{i}\sum x_{i}}{\sum x_{i}}$


\begin{table}[h]
\centering
\begin{tabular}{l l l}
\toprule
\textbf{$x_{i}$} & \textbf{$f_{i}$} & \textbf{$f_{i}x_{i}$}\\
\midrule
70 & 5 & 350 \\
69 & 6 & 414\\
45 & 3 & 135\\
80 & 1 & 80\\
56 & 1 & 56\\
\bottomrule
\end{tabular}
\caption{Table caption}
\end{table}

Dari tabel, dapat kita lihat $\sum f_{i}x_{i}$  = 1035 dan $\sum f_{i}$  = 16. Sehingga:

$ \overline{x} = \frac{\sum f_{i}\sum x_{i}}{\sum x_{i}} = \frac{1035}{16} = 64,6 $

Rataan hitung nilai tersebut adalah 64,6.


Untuk data daftar distribusi frekuensi kelompok rumus yang digunakan sama dengan data daftar distribusi frekuensi tunggal yaitu $\overline{x} = \frac{\sum f_{i}\sum x_{i}}{\sum x_{i}}$ . Hanya saja, karena ada pengelompokan kelas maka $\sum x_{i}$ yang dirumus merupakan titik tengah dari kelas tersebut.$\frac{batas\;atas + batas\;bawah}{2}$

Contoh: tabel nilai ujian 80 Mahasiswa (I)

\begin{table}[h]
\centering
\begin{tabular}{l l l l}
\toprule
\textbf{Kelas} & \textbf{$x_{i}$} & \textbf{$f_{i}$} & \textbf{$f_{i}x_{i}$}\\
\midrule
31-40 & 1 & 35,5 & 35,5 \\
41-50 & 2 & 45,5 & 91 \\
51-60 & 5 & 55,5 & 277,5\\
61-70 & 15 & 65,6 & 982,5\\
71-80 & 25 & 75,5 & 1887,5\\
81-90 & 20 & 85,5 & 1710\\
91-100 & 18 & 95,5 & 1146\\
\bottomrule
\end{tabular}
\caption{Nilai Ujian}
\end{table}

Dari tabel, dapat kita lihat $\sum f_{i}x_{i}$  = 6130 dan $\sum f_{i}$  = 80. Sehingga:


$ \overline{x} = \frac{\sum f_{i}\sum x_{i}}{\sum x_{i}} = \frac{6130}{80} = 76,62 $

Rataan hitung nilai ujiannya adalah 76,62.

Untuk mencari rataan hitung data distribusi frekuensi kelompok dapat digunakan cara lainnya yaitu cara sandi atau cara singkat. Untuk memakai cara ini maka gunakan langkah-langkah berikut
	Ambil salah satu titik tengah kelas, namakan x0.
	Untuk titik tengah x0 diberi nilai sandi c = 0
	Titik tengah yang nilainya kurang dari x0 berturut-turut diberi harga-harga sandi c = -1, c = -2, c = -3, dan seterusnya.
	Titik tengah yang nilainya lebih dari x0 berturut-turut diberi harga-harga sandi c = +1, c = +2, c = +3, dan seterusnya.
	p merupakan panjang kelas dimana setiap kelas memiliki panjang kelas yang sama.
Gunakan rumus

\chapter{Peluang kejadian majemuk}
\section{Kejadian Saliing bebas}

Peluang Kejadian A dinotarsikan dengan P(A) adalah perbandingan banyaknya hasil kejadian A dinotasikan n(A) terhadap banyak semmua hasil yang mungkin dinotasikan dengan n(S) dalam satu percobaan. Kisaran nilai peluang kejadian A adalah $0 \leq$ P(A) $0 \leq$ 1. jika P(A) = 0 disebut kemustahilan dan P(A) = 1 disebut kepastian.

1. kejadian saling bebas(Stokastik)

Dua kejadian dikatakan saling bebas (independen) jika terjadinya kejadian yang satu tidak mempengaruhi kemungkinan terjadinya kejadian yang lain. Bila kejadian A tidak mempengaruhi terjadinya B dan sebaliknya, maka kejadian semacam ini disebut dua kejadian saling bebas
Contoh
Ketika melempar koin dua kali, hasil dari lemparan pertama tidak mempengaruhi hasil dari lemparan kedua.
Ketika mengambil dua kartu dari satu set kartu permainan (52 kartu), kejadian 'mendapatkan raja (K)' pada kartu pertama dan kejadian 'mendapatkan kartu hitam' pada kartu kedua adalah tidak saling bebas. Peluang pada kartu kedua berubah setelah kartu yang pertama diambil. Kedua kejadian di atas akan menjadi saling bebas jika setelah mengambil
kartu yang pertama, kartu tersebut dikembalikan ke set semula (sehingga set kartu itu lengkap kembali, 52 kartu).
Jika dua keeping mata uang yang homogen dilantunkan bersama-sama, maka kejadian yang mungkin adalah : S = {(G1,G2), (G1,A2), (A1,G2), (A1,A2)} ® n(s) = 4

\section{Penjelasan Kejadian Saling bebas}\index{Penjelasan Kejadian Saling Bebas}
\vspace{0.5cm} 
Jekadian pada percobaan ada dua yaitu;
kejadian sedehana dan kejadian dan jekadian majemuk
peluang kejadian saling lepas dan saling bebas.

Dengan menggunakan sifat-sifat gabungan dua berdasarkan teori gabungan banyak nya anggota himpunan $A\bigcup B$ yang disimbol kan.



$A\bigcap B$ = n(A) + n(B) ($A\bigcup B$) dengan  ($A\bigcup B$) yang menyatakan irisan himpunan A dan B

Menentukan Peluang gabungan dua kejadian : P($A\bigcap B$).

n($A\bigcup B$)= n(A)+(B) -n($A\bigcap B$) [bagi dg n(S)]. 

$\frac{n}{n}$ $A\bigcup B$ = $\frac{n}{n}$ $\frac{A}{S}$ + $\frac{n}{n}$ $\frac{B}{S}$ n($A\bigcap B)$ $\frac{n}S$

P($A\bigcup B$) = P(A)+P(B)-P($A\bigcap B$)

Jadi rumus peluang gabunganya adalah;
P($A\bigcup B$) = P(A)+P(B)-P($A\bigcap B$)
Keterangan :

P($A\bigcup B$) = Peluang Gabungan kejadian A dan B

P(A) = Peluang kejadian A

P(B) = Peluang kejadian B

P$A\bigcap B$) Peluang irisan kejadian A dan B


%------------------------------------------------

\section{Contoh Soal}\index{Peluang}

1. Sebuah dadu isi enam di lempar sekali, berapakah peluang kejadian muncul nya angka genap atau angka prima ? 

Penyelesaian :

a.) Ruang sample nya adalah S={1,2,3,4,5,6} dengan n(S)=6
misalnya A kejadian muncul mata dadu genap dan B kejadian mata dadu prima,

A={2,4,5} B={2,3,5} dan $A\bigcap B$ ={2}
$A\bigcap B$
sehingga n(A)=3,n(B)=3, n($A\bigcap B$) = 1

Ketika melempar koin dua kali, hasil dari lemparan pertama tidak mempengaruhi hasil dari lemparan kedua.
\vspace{1cm} 
Ketika mengambil dua kartu dari satu set kartu permainan (52 kartu), kejadian 'mendapatkan raja (K)' pada kartu pertama dan kejadian 'mendapatkan kartu hitam' pada kartu kedua adalah tidak saling bebas. Peluang pada kartu kedua berubah setelah kartu yang pertama diambil. Kedua kejadian di atas akan menjadi saling bebas jika setelah mengambil kartu yang pertama, kartu tersebut dikembalikan ke set semula (sehingga set kartu itu lengkap kembali, 52 kartu).
\vspace{1cm} 

Untuk dua kejadian saling bebas, A dan B, peluang untuk keduanya terjadi, P(A dan B), adalah hasil perkalian antara peluang dari masing-masing kejadian.
\vspace{1cm} 

P( A dan B ) =  P($A\bigcap B$) = P(A) X P(B)
Misalnya, ketika melempar koin dua kali, peluang mendapat 'kepala' (K) pada lemparan pertama lalu mendapat 'ekor' (E) pada lemparan kedua adalah
\vspace{1cm} 

P(K dan E) = P(X) X P(B)
\vspace{0.5in}

P(K dan E) = 0.5 X 0.5
\vspace{0.5in}
P(K dan E) = 0.25

Dalam sebuah kantong terdapat 15 alat tulis yang terdiri dari 7 Pensil dan 8
Bolpen. Jika kita disuruh mengambil 2 alat tulis dengan mata tertutup.
Tentukan terambil kedua-duanya Pensil ?
Jawab :

Jika A = Pensil Pengambilan Pertama :

Maka P(A) = n(A)/n(S) = 7/15

Jika B = Pensil Pengambilan Kedua :

Maka P(B) = n(B)/n(S) = 6/15

Jadi P(A n B) = P(A) x P(B)

= 4/10 x 3/9
= 12/90
= 2/15




%----------------------------------------------------------------------------------------
%	BIBLIOGRAPHY
%----------------------------------------------------------------------------------------

\chapter*{Bibliography}
\addcontentsline{toc}{chapter}{\textcolor{ocre}{Bibliography}}
\section*{Books}
\addcontentsline{toc}{section}{Books}
\printbibliography[heading=bibempty,type=book]
\section*{Articles}
\addcontentsline{toc}{section}{Articles}
\printbibliography[heading=bibempty,type=article]

%----------------------------------------------------------------------------------------
%	INDEX
%----------------------------------------------------------------------------------------

\cleardoublepage
\phantomsection
\setlength{\columnsep}{0.75cm}
\addcontentsline{toc}{chapter}{\textcolor{ocre}{Index}}
\printindex

%----------------------------------------------------------------------------------------

\end{document}
