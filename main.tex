%%%%%%%%%%%%%%%%%%%%%%%%%%%%%%%%%%%%%%%%%
% The Legrand Orange Book
% LaTeX Template
% Version 2.2 (30/3/17)
%
% This template has been downloaded from:
% http://www.LaTeXTemplates.com
%
% Original author:
% Mathias Legrand (legrand.mathias@gmail.com) with modifications by:
% Vel (vel@latextemplates.com)
%
% License:
% CC BY-NC-SA 3.0 (http://creativecommons.org/licenses/by-nc-sa/3.0/)
%
% Compiling this template:
% This template uses biber for its bibliography and makeindex for its index.
% When you first open the template, compile it from the command line with the 
% commands below to make sure your LaTeX distribution is configured correctly:
%
% 1) pdflatex main
% 2) makeindex main.idx -s StyleInd.ist
% 3) biber main
% 4) pdflatex main x 2
%
% After this, when you wish to update the bibliography/index use the appropriate
% command above and make sure to compile with pdflatex several times 
% afterwards to propagate your changes to the document.
%
% This template also uses a number of packages which may need to be
% updated to the newest versions for the template to compile. It is strongly
% recommended you update your LaTeX distribution if you have any
% compilation errors.
%
% Important note:
% Chapter heading images should have a 2:1 width:height ratio,
% e.g. 920px width and 460px height.
%
%%%%%%%%%%%%%%%%%%%%%%%%%%%%%%%%%%%%%%%%%

%----------------------------------------------------------------------------------------
%	PACKAGES AND OTHER DOCUMENT CONFIGURATIONS
%----------------------------------------------------------------------------------------

\documentclass[11pt,fleqn]{book} % Default font size and left-justified equations

%----------------------------------------------------------------------------------------

\input{structure} % Insert the commands.tex file which contains the majority of the structure behind the template

\begin{document}

%----------------------------------------------------------------------------------------
%	TITLE PAGE
%----------------------------------------------------------------------------------------

\begingroup
\thispagestyle{empty}
\begin{tikzpicture}[remember picture,overlay]
\node[inner sep=0pt] (background) at (current page.center) {\includegraphics[width=\paperwidth]{background}};
\draw (current page.center) node [fill=ocre!30!white,fill opacity=0.6,text opacity=1,inner sep=1cm]{\Huge\centering\bfseries\sffamily\parbox[c][][t]{\paperwidth}{\centering The Search for a Title\\[15pt] % Book title
{\Large A Profound Subtitle}\\[20pt] % Subtitle
{\huge Dr. John Smith}}}; % Author name
\end{tikzpicture}
\vfill
\endgroup

%----------------------------------------------------------------------------------------
%	COPYRIGHT PAGE
%----------------------------------------------------------------------------------------

\newpage
~\vfill
\thispagestyle{empty}

\noindent Copyright \copyright\ 2013 John Smith\\ % Copyright notice

\noindent \textsc{Published by Publisher}\\ % Publisher

\noindent \textsc{book-website.com}\\ % URL

\noindent Licensed under the Creative Commons Attribution-NonCommercial 3.0 Unported License (the ``License''). You may not use this file except in compliance with the License. You may obtain a copy of the License at \url{http://creativecommons.org/licenses/by-nc/3.0}. Unless required by applicable law or agreed to in writing, software distributed under the License is distributed on an \textsc{``as is'' basis, without warranties or conditions of any kind}, either express or implied. See the License for the specific language governing permissions and limitations under the License.\\ % License information

\noindent \textit{First printing, March 2013} % Printing/edition date

%----------------------------------------------------------------------------------------
%	TABLE OF CONTENTS
%----------------------------------------------------------------------------------------

%\usechapterimagefalse % If you don't want to include a chapter image, use this to toggle images off - it can be enabled later with \usechapterimagetrue

\chapterimage{chapter_head_1.pdf} % Table of contents heading image

\pagestyle{empty} % No headers

\tableofcontents % Print the table of contents itself

\cleardoublepage % Forces the first chapter to start on an odd page so it's on the right

\pagestyle{fancy} % Print headers again

%----------------------------------------------------------------------------------------
%	PART
%----------------------------------------------------------------------------------------

\part{Part One}

%----------------------------------------------------------------------------------------
%	CHAPTER 1
%----------------------------------------------------------------------------------------

\chapterimage{chapter_head_2.pdf} % Chapter heading image

\chapter{kesebangunan bangun datar}

\section{Pengertian kesebangunan dan kongruen}\index{Paragraphs of Text}

Kesebangunan dan kekongruenan biasanya digunakan untuk membandingkan dua buah bangun datar (atau lebih) dengan bentuk yang sama. dua buah bangun datar dapat dikatakan sebangun apabila panjang setiap sisi pada kedua bangun datar tersebut memiliki nilai perbandingan yang sama. sedangkan kongruen memiliki konsep yang lebih mendetail, apabila dua buah (atau lebih) bangun datar memiliki bentuk, ukuran, serta besar sudut yang sama barulah mereka dapat disebut sebagai bangun datar yang kongruen.Perhatikan gambar berikut:

\includegraphics[width=3cm,height=3cm]{Kesebangunan.jpg}


Kesebangunan Pada Persegi Panjang

Perhatikan gambar dua buah persegi panjang di bawah ini.keduanya merupakan bangun datar yang sebangun karena memiliki kesamaan sifat yang dapat dijelaskan sebagai berikut:

\includegraphics[width=3cm,height=3cm]{persegi.jpg}


\textbf{1.Perbandingan antara sisi terpanjang dengan sisi terpendek memiliki nilai yang sama.}

Perbandingan sisi terpanjang PQ dengan sisi terpendek QR  = 39 : 13  = 1 : 3
Perbandingan sisi terpanjang KL dengan sisi terpendek LM   = 24 : 8    = 1 : 3
Perbandingan sisi terpanjang RS dengan sisi terpendek QP   = 39 : 13  = 1 : 3
Perbandingan sisi terpanjang MN dengan sisi terpendek NK = 24 : 8    = 1 : 3

Dari perhitungan diatas dapat dilihat bahwa sisi terpanjang dan terpendek pada kedua persegi panjang diatas  memiliki perbandingan yang sama yaitu 1 : 3.


\textbf{2.Besar sudut pada kedua persegi panjang tersebut memiliki nilai yang sama besar.}

Sudut P = Sudut K; Sudut Q = Sudut L; Sudut R = Sudut M; Sudut S = Sudut N

Karena kedua persegi panjang tersebut hanya memiliki bentuk dan sudut yang sama besar namun tidak memiliki ukuran yang sama, maka dua bangun datar tersebut tidak bisa disebut kongruen.

\textbf{Contoh Soal Kesebangunan pada Persegi Panjang}

Ada dua buah persegi panjang dengan ukuran yang berbeda ABCD dan KLMN. Persegi panjang ABCD memiliki panjang 16cm dan lebar 4cm. Bila persegi panjang ABCD sebangun dengan persegi panjang KLMN yang memiliki panjang 32cm, maka berapakah lebar dari persegi panjang KLMN?

Karena kedua persegi panjang tersebut sebangun, maka berlaku rumus:

AB/KL = BC/LM
16/32 = 4/LM
   LM = 32x4/16
   LM = 124/16
   LM = 8 cm

Maka lebar dari persegi panjang KLMN adalah 8 cm.


Kesebangunan pada Segitiga
Kesebangunan pada segitiga agak lebih sulit dicapai karena ada tiga buah sisi yang harus sama perbandingannya. 

Contoh segitiga yang sebangun:


\includegraphics[width=3cm,height=3cm]{segitiga.jpg}


Segitiga tersebut dapat dikatakan sebangun karena perbandingan sisi-sisinya sama besar:

Sisi AC sesuai dengan sisi PR = AC/PR = 4/2 = 2/1
Sisi AB sesuai dengan sisi PQ = AB/PQ = 8/4 = 2/1
Sisi BC sesuai dengan sisi QR = BC/QR = 6/3 = 2/1

Maka AC/PR = AB/PQ = BC/QR = 2/1


Besar sudut yang bersesuaian memiliki besar yang sama:

Sudut A = sudut P; sudut B = sudut Q; sudut C = sudut R

\textbf{Contoh Soal Kesebangunan pada Persegi Panjang}


\includegraphics[width=3cm,height=3cm]{soal.jpg}


Diketahui segitiga ABC sebangun dengan segitiga KLM, maka berapakah panjang LM dan MK?

Jawab:

AB/KL = BC/LM
18/6  = 15/LM
   3  = 15/LM
   LM = 15/3
   LM = 5 cm

Dari hasil tersebut kita dapat mengetahui bahwa perbandingan sisi pada kedua segitiga tersebut adalah:

18 : 6 = 3 : 1
15 : 5 = 3 : 1
12 : MK = 3 : 1
MK = 12/3
MK = 4 cm
\\

\textbf{Contoh Kesebangunan pada Trapesium}

Perhatikan gambar di bawah ini!

\includegraphics[width=3cm,height=3cm]{soal1.jpg}

Buktikan bahwa,

Soal 6 Rumus

Jika DC = 20 cm, AB = 34 cm, DE = 9 cm dan AE = 15 cm, tentukan EF!

Pembahasan Untuk membuktikan rumus yang ditentukan, kita harus menggambar garis DH yang sejajar dengan garis BC, seperti berikut.
Karena garis EG sejajar dengan garis AH, maka segitiga DEG sebangun dengan segitiga DAH. Akibatnya,

\includegraphics[width=3cm,height=3cm]{rumus1.jpg}

Untuk DC = 20 cm, AB = 34 cm, DE = 9 cm dan AE = 15 cm, maka

\includegraphics[width=3cm,height=3cm]{rumus2.jpg}

Jadi, diperoleh panjang EF adalah 25,25 cm.

\chapter{Kekongruenan Antar Bangun Datar}

\section{Pendahuluan}\index{Paragraphs of Text}

Definisi kekongruenan tidak lepas dari kesebangunan karena kekongruenan
merupakan kasus khusus kesebangunan. Jadi definisinya sebagai berikut.
Dua segibanyak (polygon) dikatakan kongruen jika ada korespondensi satu-satu
antara titik-titik sudut kedua segi banyak tersebut sedemikian hingga berlaku: 

1. sudut-sudut yang bersesuaian sama besar, dan

2. semua perbandingan panjang sisi-sisi yang bersesuaian adalah satu.

Syarat kedua ini dapat diringkas menjadi 2`. sisi-sisi yang bersesuaian sama panjang. 

%------------------------------------------------

\section{Contoh}\index{Contoh}
\includegraphics[width = 8cm, height= 5cm]{Pictures/1.png}
 
Pada gambar di atas telah dibuat korespondensi satu-satu antar titik-titik sudut pada kedua bangun sehingga sudut-sudut yang bersesuaian sama besar dan sisi-sisi yang bersesuaian sama panjang Berarti (sesuai definisi) dapat disimpulkan segiempat
ABCD kongruen dengan segiempat EFGH atau ditulis segiempat ABCD $latex\cong $ EFGH.

Sekali lagi, perhatikan bahwa korespondensi yang menjadikan dua bangun datar kongruen tidak terpengaruh oleh posisi kedua bangun. Jadi sekali telah ditemukan korespondensi satu-satu antar kedua bangun maka posisi apapun tetap kongruen. 

\includegraphics[width = 8cm, height= 5cm]{Pictures/2.png}

Perhatikan gambar di atas. Kedua bangun pada posisi I, II, III, mupun IV tetap
kongruen walaupun posisi kedua bangun tersebut berubah-ubah. Jika dicermati lebih
lanjut, keempat posisi itu mewakili proses translasi, refleksi, rotasi, dan kombinasi
dari ketiganya. Secara bahasa sederhana, dua bangun dikatakan kongruen jika kedua
bangun tersebut sama dalam hal bentuk dan ukurannya. 

\paragraph{}


Selanjutnya perhatikan segiempat dan segilima berikut. 

\includegraphics[width = 8cm, height= 5cm]{Pictures/3.png}

Berdasar gambar di atas, segiempat dapat disusun dari dua segitiga dan segilima
dapat disusun dari tiga segitiga. Secara umum segi-n dapat disusun dari n – 2 segitiga.
Hal tersebut merupakan gambaran bahwa setiap segibanyak dapat disusun dari segitiga-segitiga. Oleh karena itu sifat-sifat kesebangunan dan kekongruenan pada
segitiga perlu untuk dibicarakan secara khusus. 

%------------------------------------------------



%----------------------------------------------------------------------------------------
%	CHAPTER 2
%----------------------------------------------------------------------------------------

\chapter{Teorema}

\section{Teorema}\index{Teorema}

Secara sederhana sesuai dengan pengertian kekongruenan, dua segitiga dikatakan
kongruen jika sudut-sudut yang bersesuaian sama besar dan sisi-sisi yang bersesuaian
sama panjang. Ada satu postulat dan tiga teorema yang terkait dengan kekongruenan
segitiga. Kita ingat bahwa postulat tidak dibuktikan sedangkan teorema perlu
dibuktikan. Tetapi pada modul ini kita tidak membahas bukti teorema karena telah
dibahas pada modul BERMUTU tahun sebelumnya. 

\subsection{Postulat kekongruenan s.sd.s (sisi-sudut-sisi}\index{Teorema!Postulat kekongruenan s.sd.s (sisi-sudut-sisi)}


\begin{theorem}[Postulat kekongruenan s.sd.s (sisi-sudut-sisi)]

Diberikan dua segitiga $\vartriangle $ABC dan $vartriangle $DEF dimana m$\angle$A = m$\angle$D, AB = DF maka $\vartriangle $ABC $\cong$ $\vartriangle $DEF
\end{theorem}
\includegraphics[width = 8cm, height= 4cm]{Pictures/4.png}
\subsection{Teorema kekongruenan sd.s.sd (sudut-sisi-sudut)}\index{Theorems!Teorema kekongruenan sd.s.sd (sudut-sisi-sudut)}
\begin{theorem}
Diberikan dua segitiga $\vartriangle $ABC dan $vartriangle $DEF dimana m$\angle$A = m$\angle$D, AC = DF, m$\angle$A = m$\angle$D maka $\vartriangle $ABC $\cong$ $\vartriangle $DEF
\end{theorem}
\includegraphics[width = 8cm, height= 4cm]{Pictures/5.png}

%------------------------------------------------

\subsection{Teorema Teorema kekongruenan s.s.s (sisi-sisi-sisi)}\index{Theorems!Teorema kekongruenan s.s.s (sisi-sisi-sisi)}
\begin{theorem}
Diberikan dua segitiga $\vartriangle $ABC dan $vartriangle $DEF dimana, AB = DE,  m$\angle$A = m$\angle$D,dan  m$\angle$C = m$\angle$F , BC = EF  maka $\vartriangle $ABC $\cong$ $\vartriangle $DEF
\end{theorem}
\includegraphics[width = 8cm, height= 4cm]{Pictures/6.png}

\subsection{Teorema kekongruenan s.sd.sd (sisi-sudut-sudut)}\index{Theorems!Teorema kekongruenan s.sd.sd (sisi-sudut-sudut)}
\begin{theorem}
Diberikan dua segitiga $\vartriangle $ABC dan $vartriangle $DEF dimana, AB = DE, AC = DF,dan , BC = EF  maka $\vartriangle $ABC $\cong$ $\vartriangle $DEF
\end{theorem}
\includegraphics[width = 8cm, height= 4cm]{Pictures/7.png}

\section{Definitions}\index{Definitions}

This is an example of a definition. A definition could be mathematical or it could define a concept.

\begin{definition}[Definition name]
Given a vector space $E$, a norm on $E$ is an application, denoted $||\cdot||$, $E$ in $\mathbb{R}^+=[0,+\infty[$ such that:
\begin{align}
& ||\mathbf{x}||=0\ \Rightarrow\ \mathbf{x}=\mathbf{0}\\
& ||\lambda \mathbf{x}||=|\lambda|\cdot ||\mathbf{x}||\\
& ||\mathbf{x}+\mathbf{y}||\leq ||\mathbf{x}||+||\mathbf{y}||
\end{align}
\end{definition}

%------------------------------------------------

\section{Notations}\index{Notations}

\begin{notation}
Given an open subset $G$ of $\mathbb{R}^n$, the set of functions $\varphi$ are:
\begin{enumerate}
\item Bounded support $G$;
\item Infinitely differentiable;
\end{enumerate}
a vector space is denoted by $\mathcal{D}(G)$. 
\end{notation}

%------------------------------------------------

\section{Remarks}\index{Remarks}

This is an example of a remark.

\begin{remark}
The concepts presented here are now in conventional employment in mathematics. Vector spaces are taken over the field $\mathbb{K}=\mathbb{R}$, however, established properties are easily extended to $\mathbb{K}=\mathbb{C}$.
\end{remark}

%------------------------------------------------

\section{Corollaries}\index{Corollaries}

This is an example of a corollary.

\begin{corollary}[Corollary name]
The concepts presented here are now in conventional employment in mathematics. Vector spaces are taken over the field $\mathbb{K}=\mathbb{R}$, however, established properties are easily extended to $\mathbb{K}=\mathbb{C}$.
\end{corollary}

%------------------------------------------------

\section{Propositions}\index{Propositions}

This is an example of propositions.

\subsection{Several equations}\index{Propositions!Several Equations}

\begin{proposition}[Proposition name]
It has the properties:
\begin{align}
& \big| ||\mathbf{x}|| - ||\mathbf{y}|| \big|\leq || \mathbf{x}- \mathbf{y}||\\
&  ||\sum_{i=1}^n\mathbf{x}_i||\leq \sum_{i=1}^n||\mathbf{x}_i||\quad\text{where $n$ is a finite integer}
\end{align}
\end{proposition}

\subsection{Single Line}\index{Propositions!Single Line}

\begin{proposition} 
Let $f,g\in L^2(G)$; if $\forall \varphi\in\mathcal{D}(G)$, $(f,\varphi)_0=(g,\varphi)_0$ then $f = g$. 
\end{proposition}

%------------------------------------------------

\section{Examples}\index{Examples}

This is an example of examples.

\subsection{Equation and Text}\index{Examples!Equation and Text}

\begin{example}
Let $G=\{x\in\mathbb{R}^2:|x|<3\}$ and denoted by: $x^0=(1,1)$; consider the function:
\begin{equation}
f(x)=\left\{\begin{aligned} & \mathrm{e}^{|x|} & & \text{si $|x-x^0|\leq 1/2$}\\
& 0 & & \text{si $|x-x^0|> 1/2$}\end{aligned}\right.
\end{equation}
The function $f$ has bounded support, we can take $A=\{x\in\mathbb{R}^2:|x-x^0|\leq 1/2+\epsilon\}$ for all $\epsilon\in\intoo{0}{5/2-\sqrt{2}}$.
\end{example}

\subsection{Paragraph of Text}\index{Examples!Paragraph of Text}

\begin{example}[Example name]
\lipsum[2]
\end{example}

%------------------------------------------------

\section{Exercises}\index{Exercises}

This is an example of an exercise.

\begin{exercise}
This is a good place to ask a question to test learning progress or further cement ideas into students' minds.
\end{exercise}

%------------------------------------------------

\section{Problems}\index{Problems}

\begin{problem}
What is the average airspeed velocity of an unladen swallow?
\end{problem}

%------------------------------------------------

\section{Vocabulary}\index{Vocabulary}

Define a word to improve a students' vocabulary.

\begin{vocabulary}[Word]
Definition of word.
\end{vocabulary}

%----------------------------------------------------------------------------------------
%	PART
%----------------------------------------------------------------------------------------

\part{Part Two}

%----------------------------------------------------------------------------------------
%	CHAPTER 3
%----------------------------------------------------------------------------------------
\chapterimage{chapter_head_2.pdf} % Chapter heading image

\chapter{Statistika}

\section{Ukuran Pemusatan Data}\index{Ukuran Pemusatan Data}

Ukuran pemusatan data disebut juga sembarang ukuran yang menunjukkan pusat sekumpulan data, yang telah diurutkan dari angka yang terkecil sampai terbesar atau sebaliknya dari angka yang terbesar sampai terkecil. Beberapa fungsi dari ukuran pemusatan data adalah untuk membandingkan dua data atau contoh, karena sangat sulit untuk membandingkan banyaknya anggota dari masing-masing anggota populasi atau banyaknya anggota data contoh. Nilai ukuran pemusatan ini dibuat sehingga dapat mewakili seluruh nilai pada data yang bersangkutan.

Ukuran pemusatan yang sering digunakan adalah mean, modus, dan median. Nilai tengah (mean) akan sangat dipengaruh nilai banyaknya data. Median yang sangat beragam sulit dalam penggunaan parameter populasi. Dan modus hanya digunakan untuk data ukuran yang besar.

Salah satu ukuran yang paling penting untuk menggambarkan suatu distribusi data adalah nilai pusat data pengamat. Setiap pengukuran aritmatika yang ditujukan untuk menggambarkan suatu nilai yang mewakili nilai pusat atau nilai sentral dari suatu gugus data (himpunan pengamatan) dikenal sebagai ukuran tendensi sentral. Biasanya Ukuran pemusatan data sering kali digunakan agar data yang diperoleh mudah untuk dipahami oleh siswa. Ukuran pemusatan data debagi menjadi mean yang digunakan untuk mengetahui nilai rata rata pada setiap himpunan angka, median digunakan untuk mengetahui suatu nilai tengah suatu himpunan angka, dan modus adalah data yang sering muncul.


%------------------------------------------------

\section{Mean}\index{Mean}

Mean yaitu suatu nilai rata rata dan di dapatkan dari sekumpulan data adalah jumlah seluruh data dibagi banyaknya data. Dengan mengetahui mean suatu data, maka variasi data yang lain akan mudah diperkirakan.atau juga dapat disebut suatu metode yang sering digunakan untuk menggambarkan ukuran suatu data. Mean dapat dihitung dengan menjumlahkan seluruh nilai data pengukuran dan dibagi dengan banyaknya data yang digunakan. Definisi tersebut di nyatakan dengan persamaan sebagai berikut:

\begin{theorem}[Mean]
rumus mencari nilai rata-rata:
\begin{align}
Sampel\\
& \bar{X}=\frac{x_{1}+x_{2}+x_{3}+......+x_{n}}{n}=\sum_{i=0}^{n}\frac{x_{i}}{n}\\
Populasi\\
& \bar{\mu}=\frac{x_{1}+x_{2}+x_{3}+......+x_{n}}{n}=\sum_{i=0}^{n}\frac{x_{i}}{n}
\end{align}
\end{theorem}

Keterangan

$\sum$ = lambang penjumlahan semua gugus data pengamatan 

n = banyaknya sampel data 

N = banyaknya data populasi 

$\bar x$ = nilai rata-rata sampel

$\mu$ = nilai rata-rata populasi

\subsection{Distribusi frekuensi}\index{Theorems!Several Equations}
Rata-rata yang dihitung berdasarkan data yang sudah ditata dalam bentuk tabel distribusi frekuensi dan dapat ditentukan dengan menggunakan formula / rumus rumus yang sama dengan formula untuk menghitung nilai rata-rata dari data yang sudah dikelompokkan atau data yang terdistribusi, dengan rumus sebagai berikut:

\begin{theorem}[Mean]
rumus mencari nilai distribusi frekuensi:
\begin{align}
& \bar{X} = \frac{\sum f_{i}x_{i}}{\sum f_{i}} 
\end{align}
\end{theorem}

Keterangan
 
$ \sum $ = lambang penjumlahan semua gugus data

$ f_{i} $ = frekuensi data ke-i

$ \bar x $= nilai rata-rata sampel

%------------------------------------------------

\section{Median}\index{Median}

Median merupakan nilai tengah dari sekumpulan data yang telah diurutkan dari angka terkecil sampai ke angka terbesar. Median ditentukan berdasarkan jumlah data, dengan jumlah data yang ganjil maka mediannya memiliki nilai tengah dari data yang telah diurutkan, dan dengan jumlah data genap maka mediannya adalah mean / rataan dari dua bilangan yang ditengah data yang sudah diurutkan

\begin{theorem}[Mean]
rumus mencari nilai rata-rata:
\begin{align}
untuk n ganjil\\
& Me =x_{\frac{1}{2}(n+1)} \\
Untuk n genap\\
& Me =\frac{x_{\frac{n}{2}}+x_{\frac{n}{2}+1}}{2}
\end{align}
\end{theorem}

Keterangan 

$x_{\frac{n}{2}}$ = data pada urutan ke-$\frac{n}{2}$ setelah diurutkan


\section{Modus}\index{Modus}

Modus adalah data yang sering muncul atau data yang memiliki jumlah frekuensi paling banyak. Sebuah data dapat dikatakan tidak memiliki modus ketika seluruh data yang muncul memiliki frekuensi yang sama atau dapat disebut sebuah data memiliki modus lebih dari satu.
Untuk data yang ditampilkan dalam bentuk tabel distribusi frekuensi berkelompok, dapat digunakan menentukan letak modus dengan cara melihat kelas interval yang mempunyai frekuensi paling besar.Bila data mempunyai satu modus dapat disebut unimodal dan data yang memiliki dua modus disebut bimodal, sedangkan jika data mempunyai modus yang lebih dari dua disebut multimodal. Modus dapat dilambangkan dengan Mo

\begin{theorem}[Mean]
rumus mencari nilai rata-rata:
\begin{align}
untuk n ganjil\\
& Mo = T_{b}+(\frac{s_{1}}{s_{1}+s_{2}})i
\end{align}
\end{theorem}

Keterangan 

$Mo$ = Modus

$T_{b}$ = Tepi bawah dari kelas modus

$s_{1}$ = Selisih frekuensi kelas modus dengan frekuensi kelas sebelum kelas modus

$s_{2}$ = Selisih frekuensi kelas modus dengan frekuensi kelas sesudah kelas modus

$i$ = panjang kelas interval

%----------------------------------------------------------------------------------------
%	BIBLIOGRAPHY
%----------------------------------------------------------------------------------------

\chapter*{Bibliography}
\addcontentsline{toc}{chapter}{\textcolor{ocre}{Bibliography}}
\section*{Books}
\addcontentsline{toc}{section}{Books}
\printbibliography[heading=bibempty,type=book]
\section*{Articles}
\addcontentsline{toc}{section}{Articles}
\printbibliography[heading=bibempty,type=article]

%----------------------------------------------------------------------------------------
%	INDEX
%----------------------------------------------------------------------------------------

\cleardoublepage
\phantomsection
\setlength{\columnsep}{0.75cm}
\addcontentsline{toc}{chapter}{\textcolor{ocre}{Index}}
\printindex

%----------------------------------------------------------------------------------------

\end{document}
